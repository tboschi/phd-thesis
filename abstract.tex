\clearpage

\section*{Abstract}

We are living a pivotal moment for neutrino physics.
A new generation of experiments is about to begin and will extend our understanding of neutrinos. %% review
Very large scale experiments, like Hyper-Kamiokande, will collect unprecedented statistics and %
will constrain oscillation parameters to high precision: the CP violation phase, the octant of $\theta_{23}$, %
and the mass hierarchy are likely to be determined.
Many are the experimental difficulties behind a successful megaton water Cherenkov detector, %
but improvements in photodetection technologies luckily allow such an ambitious project.
One of the most important challenges is to keep systematical uncertainties under control, %
so as they do not dominate over statistical errors.
Assessing the impact of the systematics on the overall sensitivity of the experiment is a fundamental requirement %
to the final success of Hyper-Kamiokande.

Thanks to powerful accelerator facilities, future long baseline experiments, such as DUNE, will also explore %
the intensity frontier of neutrino physics and study rare phenomena.
%The high statistics these experiments will be exposed to will allow to study rare phenomena.
Numerous extension to the Standard Model (SM) and alternative theories have been introduced to %
explain neutrino masses and mixings.
These new scenarios often predict new physics, the signature of which is accessible to next-generation experiments.
An interesting example comes from low-scale see-saw models, which consider GeV-scale neutral leptons %
coupled to SM particles with suppressed mixing angles.
The near detector system of DUNE is an ideal place for searches of these particles, %
thanks to high exposure that compensate small event rates.

Current neutrino experiments have also joined this new venture; %
Super-Kamiokande has been extensively refurbish in view of a new phase, starting in early 2020, %
in which the detector will turn into a supernova observatory.
This is achieved by doping the water of Super-Kamiokande with gadolinium, %
in order to increase the efficiency of neutron tagging up to 90\,\%.
The use of gadolinium is a novel technique which will be adopted by many existing and planned experiments.
The benefits of improved neutron tagging are not limited just to supernova neutrinos, but %
to a plethora of other studies, such as reactor and atmospheric neutrinos or proton decay.

In this thesis, all of the topics above are addressed.
After a review of SM neutrino physics in \refcha{cha:intro}, the gadolinium-loaded water Cherenkov technique %
is discussed in \refcha{cha:skgd} with particular focus on Super-Kamiokande.
\refcha{cha:cp_hk} deals with CP violation in neutrino oscillation and the potential of Hyper-Kamiokande to %
constraining oscillation parameters.
In \refcha{cha:mass_models} we study a possible Standard Model extension to explain neutrino masses %
and in \refcha{cha:hnl_dune} we evaluate the prospect of DUNE's near detector to searches of new physics.
