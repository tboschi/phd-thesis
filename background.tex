\chapter{Background reduction}
\label{cha:appbackground}

The background evaluation is only performed for the decay channels with an important discovery potential, %
and these are $N \to \nu e^+ e^-$, $\nu e^\pm \mu^\mp$, $\nu \mu^+\mu^-$, $\nu\pi^0$, $e^\mp\pi^\pm$, and $\mu^\mp\pi^\pm$. 
In~order to reject background events, conservative event selection cuts are outlined using the differences %
between kinematic properties of the final state particles from neutrino--nucleon interactions and %
from the rare HNL in-flight decays.
Simulations of signal events with a given mass inside either the LArTPC or the MPD are input to a %
channel-specific algorithm that discards low energy events and %
defines limits on angular and transverse momentum distributions.
The algorithm aims at keeping an integrated signal efficiency $\widehat{W}_d$ greater than 30\,\%, where 
\begin{equation}
    \widehat{W}_d = \int \dd{E} W_d(E)\ ,
\end{equation}
where the signal efficiency $W_d(E)$ is introduced in \refsec{sec:numevt}.
%%BETTER!!!!!!

As an example of the selection process, the results of the background analysis for a heavy neutrino with mass $m_N = 450$\,MeV %
is reported here.
In the following tables the number of background events is reported %
in the form ``$\mathcal{X} \to \mathcal{Y}\ \mathcal{Z}$'', %
where $\mathcal{X}$ is the per mille (\np{e-3}) fraction of background events %
from mis-identification and $\mathcal{Y}$ and $\mathcal{Z}$ are fractions of irreducible background %
after the application of selection cuts to respectively Majorana and Dirac neutrino simulations.
When the value $\np{0.000}$ is shown, less than one background event per million is expected.
The average $\langle\nu\rangle$ is computed by weighting the flux components contribution to the background, %
using the respective interaction rates as weights, reported in Tab.~\ref{tab:rate}.
To obtain the number of background events, each fraction must be multiplied by the number of %
SM neutrino--nucleon interactions expected in the ND during the experiment lifetime.
It is assumed that the $\nu_\tau$ and $\cj{\nu}_\tau$ components are not responsible for background events, %
therefore only the $\nu_e$, $\nu_\mu$, and $\cj{\nu}_\mu$ components are studied.
The last row of the tables shows the integrated efficiency of the selection cuts.
%It is interesting to report that if charge identification were possible, the background events would be %
%drastically reduced. %EXTEND?
%Because of the overwhelming background, the number of signal events are most of the time substantially influenced.

The studied channels are grouped in three categories, which have similar kinematic features: %
two-body decay, which are semi-leptonic, three-body decay channels, which are purely leptonic instead, and %
decays which can be only detected via photon reconstruction.
%We limit our analysis only for events with a total energy greater than 5~GeV, unless stated differently.

\section{Two-body decays}

The two-body decays $N \to e^\pm \pi^\mp$ and $N \to \mu^\pm \pi^\mp$ are the most promising channels for the detection %
of a heavy neutrino, being the decay mode with the highest branching ratios.
%The typical signal is a two charged particle tracks with a common vertex, in a $V^0$-like fashion.
%Muons and pions leave a clean signature in the LArTPC and the MPD, %
%and electrons are easily reconstructed.
Since all final state particles are charged, direct information on the parent particle in easily reconstructed, %
as for instance the mass of the decaying neutrino, which is the invariant mass of the process
\begin{equation}
    m_N^2 = s = m_\ell^2 + m_\pi^2 + 2E_\ell E_\pi - 2|\vb{p}_\ell| |\vb{p}_\pi| \cos \theta\ ,
\end{equation}
where $\theta$ is the opening angle between the lepton and the pion.
In a two-body decay, the two particles are emitted back-to-back in the neutrino reference frame, %
so in the laboratory frame the relative position on the perpendicular plane is mostly preserved %
and $(\phi_\ell - \phi_\pi)$ is expected to be close to $\pm \pi$.
Despite these distinctive signatures, these two channels are the ones with most background events, %
coming from charged-current interactions of $\nu_e$, $\nu_\mu$, and $\cj{\nu_\mu}$ in which %
additional pions can be easily emitted in coherent or deep inelastic scatterings.
Background events typically peak at low energies and present more isotropic angular distributions.
Therefore, a tight energy threshold on the energies of the charge particles is imposed to accept %
70\,\% of the signal events and a threshold on the energy of the reconstructed neutrino is defined %
by 90\,\% of the retained events.
A cut is also placed on the reconstructed $m_N$ to retain 80\,\% of signal events, %
as well as an upper limit on the transverse momenta and angles to the beamline %
and a lower and an upper limit on the separation angle between the charged particles.
After the cuts are applied, the background events are reduced up to a factor of 2500, %
and the signal efficiency are $\sim$35\,\% for the electronic channel and $\sim$40\,\% %
for the muonic channel, with little difference (respectively 1\,\% and 3\,\%) %
between Dirac or Majorana selection windows.

\begin{center}
\smallskip
	\small
	\begin{tabular}{cr@{~}c@{~~}cr@{~}c@{~~}c}
		%%%%%%
	\toprule

	& \multicolumn{3}{c}{$N\to e^\mp \pi^\pm$}		& \multicolumn{3}{c}{$N\to \mu^\mp \pi^\pm$}	\\

	\cmidrule(lr){2-4} \cmidrule(lr){5-7}   

	& & Majorana		& Dirac	 & & Majorana	& Dirac	\\

	\cmidrule(lr){2-4} \cmidrule(lr){5-7} 

	$\nu_e$         &\np{19.090}~~$\to$ & \np{0.015} & \np{0.015}	&\np{ 0.007}~~$\to$ & \np{0.000} & \np{0.000}	\\
	$\nu_\mu$       &\np{ 0.027}~~$\to$ & \np{0.000} & \np{0.000}	&\np{25.030}~~$\to$ & \np{0.011} & \np{0.012}	\\
	$\cj{\nu}_\mu$  &\np{ 0.025}~~$\to$ & \np{0.000} & \np{0.000}	&\np{29.822}~~$\to$ & \np{0.046} & \np{0.053}	\\

	\cmidrule(lr){2-4} \cmidrule(lr){5-7}

	$\langle\nu\rangle$		&\np{ 0.239}~~$\to$ & \np{0.000} & \np{0.000}	&\np{24.302}~~$\to$ & \np{0.013} & \np{0.014}	\\

	\cmidrule(lr){2-4} \cmidrule(lr){5-7}

	$\widehat{W}_{\ell\pi}$&		& 36.4\,\%	& 35.2\,\%	&		& 43.3\,\%	& 40.2\,\% \\

	\bottomrule
		%%%%%%
	\end{tabular}
\medskip
\end{center}

\section{Three-body decays}

The three-body decays studied are $N\to \nu e^- e^+$, $N\to \nu e^\mp \mu^\pm$, and $N\to \nu \mu^- \mu^+$.
The event selection in this case is more challenging compared to two-body decays event, due the loss of the light neutrino %
which precludes the reconstruction of the decaying HNL, and so cuts as rigorous cannot be defined.
However, since two charged leptons are needed to identify these channels, the resulting background rate, %
from mis-identified photons (from $\pi^0$ decays) and long-track pions, is low.
Even in this case, only high energy events are considered, but with a lower threshold on the charged lepton energies.
The invariant mass of the two leptons has as upper limit $m_N$ and this constrain helps with reducing the background.
Lower and upper limits are also defined for the transverse momenta, as well as separation angles from the beamline.

The background events are reduced from a factor of 40 up to a factor of 200, with the selection requirements %
for Dirac neutrinos being more effective.
The signal efficiency results to be better (6--8\,\% better) for Majorana neutrinos in the $N\to \nu e^- e^+$ and $\mu^-\mu^+$ %
channels, whereas the Dirac neutrino have give a better efficiency in the $N\to e^\mp \mu^\pm$ channel.
High efficiency and low background make these three channel competitive for HNL discovery, despite having %
lower branching ratio and so weaker sensitivity.
%\enlargethispage{\baselineskip}

\begin{center}
\smallskip
	\small
	\begin{tabular}{cr@{~}c@{~~}cr@{~}c@{~~}cr@{~}c@{~~}c}
	%\begin{tabular}{cr@{~}r@{\ |\ }rr@{~}r@{\ |\ }rr@{~}r@{\ |\ }r}
		%%%%%%
	\toprule

	%& \multicolumn{3}{c}{$N\to e^\mp \pi^\pm$}		& \multicolumn{3}{c}{$N\to \mu^\mp \pi^\pm$}	\\
 & \multicolumn{3}{c}{$N\to \nu e^- e^+$}	& \multicolumn{3}{c}{$N\to \nu e^\mp \mu^\pm$}	& \multicolumn{3}{c}{$N\to \nu \mu^- \mu^+$} \\

	\cmidrule(lr){2-4} \cmidrule(lr){5-7}  \cmidrule(lr){8-10} 

	& & Majorana		& Dirac	 & & Majorana	& Dirac & & Majorana & Dirac	\\

	\cmidrule(lr){2-4} \cmidrule(lr){5-7}    \cmidrule(lr){8-10}

	$\nu_e$         &\np{0.190}~~$\to$ & \np{0.003} & \np{0.002}  &\np{0.078}~~$\to$ & \np{0.002} & \np{0.002}  &\np{0.000}~~$\to$ & \np{0.000} & \np{0.000} \\
	$\nu_\mu$       &\np{0.193}~~$\to$ & \np{0.001} & \np{0.000}  &\np{0.092}~~$\to$ & \np{0.000} & \np{0.000}  &\np{0.081}~~$\to$ & \np{0.001} & \np{0.001} \\
	$\cj{\nu}_\mu$  &\np{0.224}~~$\to$ & \np{0.003} & \np{0.002}  &\np{0.160}~~$\to$ & \np{0.000} & \np{0.000}  &\np{0.090}~~$\to$ & \np{0.008} & \np{0.006} \\
                                                                                                                                                                  
	\cmidrule(lr){2-4} \cmidrule(lr){5-7}    \cmidrule(lr){8-10}
	$\langle\nu\rangle$		&\np{0.168}~~$\to$ & \np{0.001} & \np{0.000}  &\np{0.090}~~$\to$ & \np{0.000} & \np{0.000}  &\np{0.022}~~$\to$ & \np{0.000} & \np{0.000}\\

	\cmidrule(lr){2-4} \cmidrule(lr){5-7}    \cmidrule(lr){8-10}

	$\widehat{W}_{\nu\ell\ell}$	&	& 63.4\,\%	& 55.4\,\%	&	& 68.6\,\%	& 71.2\,\% &	& 74.0\,\%	& 68.4\,\%	\\

	\bottomrule
		%%%%%%
	\end{tabular}
\end{center}

\section{EM--detected decays}

The semi-leptonic decay $N \to \nu \pi^0$ may only be identified by a correct photon reconstruction, %
since the neutral pion decays almost 100\,\% of the time in two photons.
This particle is produced in NC1$\pi^0$ interactions and deep inelastic scattering interactions.
Background events occur if only two final state photons from the neutral pion decay %
are above detection threshold and properly reconstructed with an invariant mass equal to $m_{\pi^0}$.
The energy of the reconstructed pion is the best discriminant against background events, %
thanks to their high energy.
Lower and upper limits can be placed on the $\pi^0$ transverse momentum and angle with the beamline, %
but also a threshold on the energy of the photons as well as an upper limit on their angular distributions %
help in defining the kinematics of the event.
The residual background for this channel is the highest among the ones studied: only reduction factors up to 130 can be achieved, %
with a notable difference between selection cuts for Majorana and Dirac HNL decays (the latter ones being more strict).
The signal efficiency efficiency is $\sim$46\,\% for Majorana and $\sim$42\,\% for Dirac.
It is, however, one of the decay modes with the highest branching ratio, %
and with advanced and dedicated techniques~\cite{Ankowski:2008aa, Back:2012wc}
the background rejection can be improved.

\begin{center}
\smallskip
	\small
	%\begin{tabular}{cr@{\ }r@{\ |\ }r}
	\begin{tabular}{cr@{~}c@{~~}c}
		%%%%%%
	\toprule

	& \multicolumn{3}{c}{$N\to \nu \pi^0$}	\\

	\cmidrule(lr){1-4}

	& & Majorana		& Dirac	 \\

	\cmidrule(lr){2-4} 
	%\cmidrule(lr){2-2} \cmidrule(lr){3-4}

	$\nu_e$         &\np{4.135}~~$\to$ & \np{0.058}	& \np{0.048}	\\
	$\nu_\mu$       &\np{5.862}~~$\to$ & \np{0.053}	& \np{0.039}	\\
	$\cj{\nu}_\mu$  &\np{7.428}~~$\to$ & \np{0.179}	& \np{0.138}	\\

	\cmidrule(lr){2-4} 
	%\cmidrule(lr){2-2} \cmidrule(lr){3-4}

	$\langle\nu\rangle$		&\np{5.797}~~$\to$ & \np{0.061}	& \np{0.045}	\\

	\cmidrule(lr){2-4} 
	%\cmidrule(lr){2-2} \cmidrule(lr){3-4}

	$\widehat{W}_{\nu\pi^0}$	& & 46.3\,\%	& 42.3\,\%	 \\

	\bottomrule
		%%%%%%
	\end{tabular}
	\medskip
\end{center}
