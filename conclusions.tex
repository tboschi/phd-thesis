\clearpage
\chapter{Conclusions}
\label{sec:conclusions}

One of the most compelling discoveries of the past few years is evidence of neutrino oscillation.
The main consequence is that at least two neutrino states have %
a non-zero mass below the eV scale, although Dirac mass terms for neutrinos are not permitted or difficult to justify %
in the Standard Model.
Addressing this insufficiency together with others, for example matter-antimatter asymmetry or dark matter, %
requires an extension of the current theory.
There are many complementary fields revolving around neutrino oscillation %
which need careful consideration in order to study this phenomenon in depth and either prove or disprove the Standard Model.
The hope is that from accurate and precise measurements new hints of physics beyond the Standard Model (BSM) will appear %
and bring about a better comprehension of theoretical models.
This thesis has explored just a fraction of the vast realm of topics in neutrino physics, %
focusing mainly on next-generation neutrino experiments.
Some of the existing experiments will renovate and broaden their capability %
by adopting new detection methods, such as Super-Kamiokande and gadolinium capture.
At the same time new experiments, such as Hyper-Kamiokande, will %
push familiar techniques to the limit to increase the precision of oscillation physics.
Other upcoming experiments, such as DUNE, will implement state-of-the-art technologies on a large scale %
not only to study Standard Model physics, but also to perform complementary searches of BSM physics.

The benefits of gadolinium for water Cherenkov detectors have been explained in \refcha{cha:skgd}.
Super-Kamiokande is about to start a new phase in which a hundred tons of gadolinium sulphate %
will be dissolved in the water tank.
This will allow the detector to improve neutron tagging efficiency up to 90\,\%, with a Gd concentration of 0.2\,\%, %
transforming the experiment into a supernova observatory %
and improving all the ongoing analyses.
%The ongoing analyses, such as atmospheric and beam oscillation studies, solar neutrino surveys, and proton decay searches, %
%can be refined by a controlled neutron background.
There are many technical challenges when dealing with a Gd-loaded Cherenkov detector, %
one of which being the monitoring of gadolinium concentration in water.
A new method involving UV absorption spectroscopy is being developed with promising results.
It allows for an almost continuous monitoring and it reachs a precision of $\sim$1\,\% %
on the full load concentration of 0.2\,\%.
The device can also be used to monitor the initial loading of 0.02\,\% with good sensitivity.
Regardless of gadolinium-doping, californium-252 is being studied as a new way of %
performing a more accurate neutron calibration in water Cherenkov detectors.

The next-generation water Cherenkov detector, Hyper-Kamiokande, has been introduced in \refcha{cha:cp_hk}.
Thanks to unprecedented statistics and upgraded instrumentation, Hyper-Kamiokande has the potential %
to study a plethora of topics related to neutrino physics, with particular emphasis on neutrino oscillation.
It is predicted that oscillation parameters will be constrained with high precision, %
eventually determining the CP violation phase in the lepton sector.
This task can be achieved also thanks to a deep knowledge of systematic errors %
introduced at the near and far detectors.
In this thesis, the sensitivity to oscillation parameters has been assessed employing %
the state-of-the-art T2K systematic model.
A combined fitting framework is employed to calculate the likelihood function of predicted event samples.
It is found that Hyper-Kamiokande can exclude maximal violation of CP at full statistics %
with more than $6\sigma$ of significance.
Furthermore, the total sensitivity is not substantially affected by variations of the nominal systematic model.
This study, however, is preliminary and still ongoing.

In \refcha{cha:mass_models}, the origin and lightness of the neutrino mass has been addressed %
in the framework of low-scale seesaw mechanisms.
The addition of an arbitrary number of heavy neutral leptons is the simplest extension of the Standard Model %
and it is accompanied with a diverse and rich phenomenology.
Some textures of the mass matrix, such as the one of the inverse seesaw, %
allows Majorana or pseudo-Dirac heavy neutrinos with experimentally accessible masses.
These new particles could be searched for by direct production in beam dump experiments, %
in which the expected signature would be a decay in-flight into Standard Model particles.
In this thesis, the phenomenological consequences of Majorana and Dirac states have been thoroughly investigated %
in light of searches of neutrino decays.
The differential decay rates and production scale factors have been computed using %
the helicity-spinor formalism and are provided decomposed by helicity states.
These results, first published in \refref{Ballett:2019bgd}, set the groundwork for novel phenomenological analyses %
with future experiments.

%this thesis has been focused on
For instance, the near detector complex of DUNE is capable of performing BSM searches, %
as discussed in \refcha{cha:hnl_dune}.
Although the main goal of the experiment is precision oscillation physics, %
the near detector proves to be an ideal candidate to search for heavy neutrino decays, %
given the intense neutrino beam and exceptional reconstruction capabilities.
If at least one extra neutral state exists with a mass from few MeV to the GeV, %
the new singlet would be produced in the beam from mixing-suppressed meson and lepton decays.
Thanks to the high energy of the beam, neutrino masses up to 2\,GeV have also been tested %
by simulating $D_s$ meson production at the target.
A background study was performed on decay channels with good detection prospects, %
defined by high branching ratios and clean detector signatures.
Combining the scaled flux components with the decay probabilities and signal efficiencies, %
the 90\,\% C.L.\ sensitivity of the near detector has been estimated to all accessible channels.
Finally, a random matrix scan of different inverse seesaw realisations was performed %
to define regions of parameter space allowed by the considered seesaw realisation.
It is found that the near detector can extend current limits on mixing parameters, %
reaching regions of interests for neutrino mass models.
If a discovery is made, some considerations can be drawn upon the nature of the new fermionic states.


In conclusion, neutrino physics has made giant leaps thanks to both impressive experimental efforts %
and remarkable theoretical progress.
This thesis has given an account of some of the necessary work needed to progress on both %
the experimental and theoretical side.
%There are several possible mechanisms that determine the neutrino masses and the oscillation parameters.
Next-generation neutrino experiments will produce new results that will prove crucial for grasping the underlying %
principles of particle physics.
