\clearpage
\chapter{Conclusions}
\label{sec:conclusions}

One of the most compelling discoveries of the past few years is evidence of neutrino oscillation.
The main consequence is that at least two neutrino states have %
a non-zero mass below the eV scale, although Dirac mass terms for neutrinos are not permitted or difficult to justify %
in the Standard Model.
Addressing this insufficiency together with others, for example matter-antimatter asymmetry or dark matter, %
requires an extension of the current theory.
The hope is that from accurate and precise measurements new hints of physics beyond the Standard Model (BSM) will appear %
and bring about a better comprehension of theoretical models.
This thesis has explored just a fraction of the vast realm of topics in neutrino physics, %
focusing mainly on next-generation neutrino experiments.
Some of the existing experiments will renovate and broaden their capabilities %
by adopting new detection methods, such as Super-Kamiokande and gadolinium capture.
At the same time new experiments, such as Hyper-Kamiokande, will %
push familiar techniques to the limit to increase the precision of oscillation physics.
Other upcoming experiments, such as DUNE, will implement state-of-the-art technologies on a large scale %
not only to study Standard Model physics, but also to perform complementary searches of BSM physics.

The benefits of gadolinium for water Cherenkov detectors have been explained in \refcha{cha:skgd}.
Super-Kamiokande (SK) is about to start a new phase in which a hundred tons of gadolinium sulphate %
will be dissolved in the water tank.
This will allow the detector to improve neutron tagging efficiency up to 90\,\%, with a Gd concentration of 0.2\,\%, %
transforming the experiment into a supernova observatory %
and improving all the ongoing analyses.
%The ongoing analyses, such as atmospheric and beam oscillation studies, solar neutrino surveys, and proton decay searches, %
%can be refined by a controlled neutron background.
There are many technical challenges when dealing with a Gd-loaded Cherenkov detector, %
one of which being the monitoring of gadolinium concentration in water.
A new method involving UV absorption spectroscopy is being developed with promising results, %
as shown in \refcha{cha:skgd}.
Gadolinium presents strong absorption lines in the region between 270\,nm and 275\,nm %
and their intensity is directly proportional to the amount of gadolinium in aqueous solution.
This technique allows for an almost continuous monitoring of the concentration, %
compared to the current spectroscopy technique in place which is executed with a monthly frequency.
The precision reached with the UV absorption spectroscopy is around $\sim$1\,\% %
on the full load concentration of 0.2\,\%, using a 100\,cm water sample.
The same device was able to reach competitive sensitivity also on the initial loading concentration of 0.02\,\%.
Regardless of gadolinium-doping, an improved method for neutron calibration of the detector was also investigated.
A californium-252 source has some advantages compared to the currently employed Americium-Beryllium (Am-Be), %
among which the possibility of estimating the source activity indirectly.
Using a GEANT4 simulation, it was found that the same calibration device used by SK %
%, consisting of a 5\,cm cube of BGO scintillator,
has already the optimal shape to be used with \tapi{252}Cf instead.
The Am-Be neutron source could then be replaced without further modifications of the device, %
and a more accurate neutron calibration can be achieved thanks to the features of spontaneous fission events in californium-252.

The next-generation water Cherenkov detector, Hyper-Kamiokande (HK), has been introduced in \refcha{cha:cp_hk}.
Thanks to unprecedented statistics and upgraded instrumentation, Hyper-Kamiokande has the potential %
to study a plethora of topics related to neutrino physics, %
with particular emphasis on neutrino oscillation and CP violation in the lepton sector.
%It is predicted that oscillation parameters will be constrained with high precision, %
%eventually determining the CP violation phase in the lepton sector.
%This task can be achieved also thanks to a deep knowledge of systematic errors %
%introduced at the near and far detectors.
In this thesis, the sensitivity to oscillation parameters has been assessed %
using a combined fit between atmospheric and beam samples.
The atmospheric events for HK are simulated by scaling SK Monte Carlo atmospheric data %
and they are classified as fully-contained, partially contained, or upward-going muons events, for a total of 2224 bins.
The beam-related events are classified as electron- or muon-like one ring events, %
for the beam in neutrino mode or antineutrino mode and they are arranged in four distributions with 87 energy bins each.
These spectra are predicted from a neutrino beam flux simulation tuned with near detector constraints.
Once a \emph{true} combination of oscillation parameters is chosen, the $\chi^2$ between observed, i.e.\ true, %
and expected events is calculated in the way explained in \refcha{cha:cp_hk}.
The framework employed is capable of beam and atmospheric combined fits.
The likelihood also includes effects from systematic uncertainties.
%After the explanation of the likelihood used in the study, the systematic error models employed are introduced.
Although the atmospheric model is still preliminary, the state-of-the-art T2K systematic model %
is adopted for the beam component instead.
It consists of 119 systematics encompassing beam, cross-section, and far detector parameters.
Just considering the beam sample, it is found that Hyper-Kamiokande can exclude maximal %
violation of CP at full statistics with more than $6\sigma$ of significance.
Some variations of the beam systematic model are also being investigated in order to understand %
how the experimental sensitivity is affected by certain model parameters. 
One of these, the energy scale error, might be treated too naively in the fitting framework %
and so slight modifications to both the error model and the likelihood should be considered.
This study, however, is preliminary and still ongoing.


In \refcha{cha:mass_models}, the origin and lightness of the neutrino mass has been addressed %
in the framework of low-scale seesaw mechanisms.
The addition of an arbitrary number of heavy neutral leptons is the simplest extension of the Standard Model %
and it is accompanied with a diverse and rich phenomenology.
Some textures of the mass matrix, such as the one of the inverse seesaw, %
predict Majorana or pseudo-Dirac heavy neutrinos with experimentally accessible masses.
These new particles could be searched for by direct production in beam dump experiments, %
in which the expected signature would be an in-flight decay into Standard Model particles.
In this thesis, the phenomenological consequences of Majorana and Dirac states have been thoroughly investigated %
in light of searches of neutrino decays.
The differential decay rates and production scale factors have been computed using %
the helicity-spinor formalism and are provided decomposed by helicity states.
It was shown that Dirac and Majorana neutrinos have different total decay width in neutral current processes %
and measuring the rate could be an actual way of determining the nature of the initial state.
Interesting differences appear between Majorana and Dirac neutrinos once the role of helicity is considered, %
and this could be also exploited to determine the nature of the heavy singlet fermion.
The effect of helicity emerges in the differential decay rate leading to different %
distributions of final state particles. %
%and if the initial state is Majorana or (pseudo-)Dirac.
For example, if the HNL are Majorana, two-body decays present an isotropic distribution for both helicity states, %
or, if Dirac, the angular distribution has a dependency proportional to $A\pm B \cos\theta$, %
with the sign depending on the helicity state.
An effective evaluation of the heavy neutrino flux was also carried out %
which is not polarised to a single helicity state differently from a light neutrino flux.
The production modes of nearly-sterile neutrinos are sensitive to their helicity state, %
thanks to mass effects which lead to the enhancement of certain channels with respect to light neutrinos.
The two components of the neutrino flux behave therefore differently thanks to the dependency of decay distribution on the helicity.
These results, first published in \refref{Ballett:2019bgd}, set the groundwork for novel phenomenological analyses %
of future experiments.

%this thesis has been focused on
For instance, the near detector complex of DUNE is capable of performing BSM searches, %
as discussed in \refcha{cha:hnl_dune}.
Although the main goal of the experiment is precision oscillation physics, %
the near detector proves to be an ideal candidate to search for heavy neutrino decays, %
given the intense neutrino beam and exceptional reconstruction capabilities.
If at least one extra neutral state exists with a mass from few MeV to the GeV, %
the new singlet would be produced in the beam thanks to suppressed mixings by meson and lepton decays.
Thanks to the high energy of the beam, neutrino masses up to 2\,GeV have also been tested %
by simulating $D_s$ meson production at the target.
A background study was performed on decay channels with good detection prospects, %
defined by high branching ratios and clean detector signatures.
Due to the ND vicinity to the beam target, it is fundamental to suppress the overwhelming %
number of SM neutrino--nucleon interactions, which constitute the background for the rare signal of HNL decays.
Most of the time, both the reconstruction of hadronic activity at the vertex %
and the multiplicity of final state particles are enough to distinguish between signal and background, %
reducing the latter down to~$\lesssim5$\,\%.
To further reduce unwanted events, simple kinematic cuts are applied thanks to the very forward distribution of %
decay in-flight events, additionally suppressing the background events to less than \np{5e-5} of the original number.
The rejection prescriptions are tuned to maintain an acceptable signal efficiency, which is between $\sim$40\,\% down to $\sim$20\,\%.
For all the other channels, no background study was performed, mainly because the final state particles %
are vector mesons which present experimentally challenging and specific signatures, the study of which %
was out of the scope of this work.
Combining the scaled flux components with the decay probabilities and signal efficiencies, %
the 90\,\% C.L.\ sensitivity of the near detector has been estimated to all accessible channels, %
for both single and dominant mixings.
For masses between 0.3 and 0.5\,GeV, the ND can probe mixing elements below \np{e-9} in most cases, %
reaching \np{e-10}, especially with two-body semi-leptonic channels for both $|U_{e N}|^2$ and $|U_{\mu N}|^2$.
Thanks to the $D_s$ meson production, neutrino masses above 0.5\, and up to 2\,GeV become accessible, %
as well as production and decay modes purely sensitive to the tau mixing.
In this case, the sensitivity does not exceed \np{e-8} for the electronic and muonic channels and %
\np{5e-6} for the tauonic channel.
Finally, a random matrix scan of different inverse seesaw realisations was performed %
to define regions of allowed parameter space.
It is found that the near detector can extend current limits on mixing parameters, %
reaching regions of interest for neutrino mass models.
If a discovery is made, some considerations can be drawn upon the nature of the new fermionic states.
The charges of the final state particles can help determine whether the lepton number %
is conserved in the event: any event clearly violating lepton number would be a manifestation of %
the Majorana nature of the decaying HNL.


In conclusion, neutrino physics has made giant leaps thanks to both impressive experimental efforts %
and remarkable theoretical progress.
This thesis has given an account of some of the necessary work needed to progress on both %
the experimental and theoretical side.
%There are several possible mechanisms that determine the neutrino masses and the oscillation parameters.
Next-generation neutrino experiments will produce new results that will prove crucial for grasping the underlying %
principles of particle physics.
% ??
There are many complementary fields revolving around neutrino oscillation %
which need careful consideration in order to study this phenomenon in depth and either prove or disprove the Standard Model.
