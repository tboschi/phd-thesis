\clearpage
\chapter{Conclusions}
\label{sec:conclusions}

Adding an arbitrary number of heavy neutral fermions is the simplest extension of the Standard Model which allows to address the neutrino mass origin.
These models are accompanied with a diverse and rich phenomenology, %
which can be tested by the next-generation neutrino experiments.
This is the case of low-scale seesaw mechanisms, such as the inverse seesaw %
which, depending on the realisation, allows Majorana or pseudo-Dirac heavy neutrinos with experimentally accessible masses.
In this paper, we have thoroughly investigated the phenomenological consequences of Majorana and Dirac states %
in light of searches of neutrino decays in beam dump experiments.
Production and decay modes have been computed using the helicity-spinor formalism, %
and all the formulae for differential decay rates and production scale factors are provided, for the first time, decomposed by helicity states.
We find agreement with previous studies, and hopefully settle down the dispute on different results.

We have shown that Dirac and Majorana neutrinos have different total decay width in NC processes %
and, in principle, measuring the rate could be a way of determining the nature of the initial state.
We put a lot of stress on the role of the helicity in these type of signatures: %
interesting differences appear between Majorana and Dirac neutrinos, which could be also %
exploited to determine the nature of the heavy singlet fermion.
%Despite the fact that the total decay rate of a massive neutrino cannot be affected by its helicity state, %
The effect of the heavy neutrino helicity appears in the differential decay rate leading to different %
distributions of final state particles. %
%and if the initial state is Majorana or (pseudo-)Dirac.
For example, if the HNL are Majorana, two-body decays present an isotropic distribution for both helicity states, %
or, if Dirac, the angular distribution has a dependency proportional to $A\pm B \cos\theta$, %
with the sign depending on the helicity state.
We have also developed an effective evaluation of the heavy neutrino flux which, differently from a light neutrino flux, %
is not polarised to a single helicity state.
The production modes of a nearly-sterile neutrinos are sensitive to its helicity state, %
due to mass effects which can lead to enhancement of certain channels with respect to light neutrinos.
The two components of the neutrino flux behave therefore differently thanks to the dependency of decay distribution on the helicity.
%This can have 
%With good statistics and collecting data in both neutrino and antineutrino mode, %
%the fermionic nature of the neutrinos could be identified by the study of the distribution of their decays~\cite{Balantekin:2018ukw}.
%However the last point is left to future studies.

We have studied the prospects for production and detection of HNL at the ND of the DUNE experiment.
The ND will be exposed to an intense neutrino beam and its exceptional reconstruction capabilities make it %
an ideal candidate for searches of heavy neutrino decays.
If at least one extra neutral state exists with a mass from few MeV to the GeV, %
the new singlet would be produced in the beam from mixing-suppressed meson and lepton decays.
It can subsequently decay inside the ND to the channels listed in \reftab{tab:decays}.
%We have estimated the performance of the ND to the detection of such signals, from evaluating %
%the HNL production at the LBNF and the decay probability at the ND.
Thanks to the high energy of the beam, we have considered the possibility of testing neutrino masses heavier than the kaon mass.
We have carried out a simulation of $D_s$ meson production and decay, extending the analysis up to neutrino masses of 2\,GeV. %
More importantly, this has also allowed us to put constraints on $|U_{\tau N}|^2$ mixing, which is weakly bounded.
%While $\nu_taus$ from $D$ decays are
%irrelevant for standard oscillation physics, 
%but remarkable for its BSM physics.

A background study was performed on decay channels with good detection prospects, %
defined by high branching ratios and clean detector signatures.
%The final states of these channels $\nu e^+ e^-$, $\nu e^\pm \mu^\mp$, $\nu \mu^+ \mu^-$, $\nu \pi^0$, $e^\mp \pi^\pm$, and $\mu^\mp \pi^\pm$.
Due to the ND vicinity to the beam target, it is fundamental to suppress the overwhelming %
number of SM neutrino--nucleon interactions, which constitute the background for the rare signal of HNL decays.
Reconstruction of hadronic activity at the vertex and the multiplicity of final state particles are %
most of the time enough to distinguish between signal and background, reducing the latter down to~$\lesssim5$\,\%.
To further reduce unwanted events, simple kinematic cuts are applied thanks to the very forward distribution of %
decay in-flight events, additionally suppressing the background events to less than \np{5e-5} of the original number.
The rejection prescriptions are tuned to maintain an acceptable signal efficiency, which is between $\sim$40\,\% down to $\sim$20\,\%.
For all the other channels, no background study was performed, mainly because the final state particles %
are vector mesons which present experimentally challenging and specific signatures, the study of which %
was out of the scope of this work.
Combining the scaled flux components with the decay probabilities and signal efficiencies, %
we estimate the 90\,\% C.L.\ sensitivity of DUNE ND to all accessible channels, %
for both single and two dominant mixings.
For masses between 0.3 and 0.5\,GeV, the ND can probe mixing elements below \np{e-9} in most cases, %
reaching $< \np{e-10}$, especially with two-body semi-leptonic channels for both $|U_{e N}|^2$ and $|U_{\mu N}|^2$.
Thanks to the $D_s$ meson production, neutrino masses above 0.2\, and up to 2\,GeV become accessible, %
as well as production and decay modes purely sensitive to the tau mixing.
In this case, the sensitivity does not exceed \np{e-8} for the electronic and muonic channels and %
\np{5e-6} for the tauonic channel. We point out that a large fraction of these parameters fall in the region relevant for the production of the baryon asymmetry via the ASR leptogenesis mechanism. 

Finally, we performed a random matrix scan of different ISS realisations to define regions of parameter space %
allowed by the model under consideration.
We identify three possible minimal cases that can provide good HNL candidates and at the same time address the lightness of the neutrino masses.
The first two correspond to an ISS\,(2,2) and an ISS\,(2,3) scenarios in which the heavy neutrino is part of the lightest pseudo-Dirac pair.
The third case is when strong LNV perturbations in a ISS\,(2,3) realisation give the Weyl state a mass accessible by the experiment.
%	In this case, the ISS(2,3) realisation predicts the existence of a fourth light state, which could potentially explain the short baseline anomalies.
%	None of the matrices generated, however, can provide both a neutrino viable for this purpose and a candidate for searches of heavy neutrino decays.
%	The third scenario is also an ISS(2,3) realisation in which strong LNV parameters are introduced and the Weyl state
%	and an ISS(2,3) scenario in which the HNL is the Weyl state, with a mass accessible by the DUNE ND decay search.
We make sure that the matrices generated are in agreement with oscillation data on neutrino masses and satisfy the constraint %
imposed by other experiments on unitarity and lepton number violation.
%The resulting points of the parameter space are then set over against the combined sensitivity of the channels with good detection prospect.
We stress that DUNE will mostly---but not exclusively---be sensitive to pseudo-Dirac states.
In the region with strongest sensitivity, which is for masses just below 0.5\,GeV for $|U_{e N}|^2$ and %
below 0.4\,GeV for $|U_{\mu N}|^2$, the ND starts intersecting regions of the parameter space %
valid for a type I seesaw realisation or Majorana states in the ISS\,(2,3) scenario.
This might have consequences for the signal and analysis strategies adopted by the collaboration, %
according to the different topology of distribution between Majorana and pseudo-Dirac neutrinos.
\enlargethispage{\baselineskip}
In case of a discovery, some considerations can be drawn upon the nature of the new fermionic states.
