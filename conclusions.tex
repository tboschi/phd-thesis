\clearpage
\chapter{Conclusions}
\label{sec:conclusions}

In the past years, neutrino physics has made giant leaps thanks to both impressive experimental efforts %
and remarkable theoretical progress.
One of the most compelling discoveries is the evidence of neutrino oscillation, %
which hints at beyond the Standard Model physics.
There are many complementary fields revolving around this phenomenon which need careful considerations  %
in order to study neutrino oscillation in depth.
This thesis has explored just a fraction of the vast realm of topics in neutrino physics, %
focusing mainly on next-generation neutrino experiments.
Some of the existing experiments will renovate to extend their physics %
reach by adopting new detection methods, such as Super-Kamiokande and gadolinium capture.
At the same time, new experiments, like Hyper-Kamiokande, will %
push familiar techniques to the limit to increase the current precision to oscillation physics.
Other upcoming experiments, such as DUNE, will implement rising technologies on a large scale %
not only to study Standard Model physics, but also to perform complementary searches of novel physics.

In \refcha{cha:skgd}, the benefits of gadolinium for water Cherenkov detectors have been discussed.
Super-Kamiokande is about to start a new phase in which hundred tons of gadolinium sulphate %
will be dissolved in the water tank.
This will allow the detector to improve neutron tagging efficiency up to 90\,\%, with a Gd concentration of 0.2\,\%, %
transforming the experiment into a \emph{de facto} supernova observatory.
The ongoing analyses, such as atmospheric and beam oscillation studies, solar neutrino surveys, and proton decay searches, %
can be refined by a controlled neutron background.
There are many technical challenges when dealing with a Gd-loaded Cherenkov detector, %
one of which being monitoring the concentration of gadolinium in water.
A new method involving UV absorption spectroscopy is being developed with promising results.
It allows for an almost continuous monitoring and it can reach a precision of $\sim$1\,\% on the concentration.
Regardless of gadolinium-doping, californium-252 is being studied as a new way of %
performing a more accurate neutron calibration in water Cherenkov detectors.

The next-generation water Cherenkov detector, Hyper-Kamiokande, has been introduced in \refcha{cha:cp_hk}.
Thanks to unprecedented statistics and upgraded instrumentation, Hyper-Kamiokande has the potential %
to study a plethora of topics related to neutrino physics, with particular emphasis on neutrino oscillation.
It is predicted that oscillation parameters will be constrained with high precision, %
eventually determining the CP violation phase in the lepton sector, $\delta_\text{CP}$.
This task can be achieved also thanks to a deep comprehension of systematic errors %
introduced at the near and at the far detector.
In this thesis, the sensitivity to oscillation parameters has been assessed employing %
the state-of-the-art T2K systematic model.
A combined fitting framework is employed to calculate the likelihood function of predicted event samples.
It is found that Hyper-Kamiokande can exclude maximal violation of CP at full statistics %
with more than $6\sigma$ of significance.
Furthermore, it is found the total sensitivity is not substantially affected by variations of the nominal systematic model.
This study, however, is preliminary and still ongoing.

The most outstanding consequence of neutrino oscillation is that at least two neutrino states have %
a non-zero mass below the eV scale.
In \refcha{cha:mass_models}, the origin and lightness of the neutrino mass has been addressed %
in the framework of low-scale seesaw mechanisms.
The addition of an arbitrary number of heavy neutral leptons is the simplest extension of the Standard Model %
and it is accompanied with a diverse and rich phenomenology.
Some textures of the mass matrix, such as the one of the inverse seesaw, %
allows Majorana or pseudo-Dirac heavy neutrinos with experimentally accessible masses.
These new particles could be searched for by direct production in beam dump experiments, %
in which the expected signature would be a decay in-flight into Standard Model particles.
In this thesis, the phenomenological consequences of Majorana and Dirac states have been thoroughly investigated %
in light of searches of neutrino decays.
The differential decay rates and production scale factors have been computed using %
the helicity-spinor formalism and are provided decomposed by helicity states.

Another the DUNE experiment is capable of performing precision oscillation physics, %
the work of this thesis however has been focused on the near detector complex of DUNE %
which is an ideal candidate to perform searches of heavy neutrino decays, as discussed in \refcha{cha:hnl_dune}.
This is possible thanks to an intense neutrino beam and exceptional reconstruction capabilities.
If at least one extra neutral state exists with a mass from few MeV to the GeV, %
the new singlet would be produced in the beam from mixing-suppressed meson and lepton decays.
Thanks to the high energy of the beam, neutrino masses up to 2\,GeV have also been tested %
by simulating $D_s$ meson production at the target.
A background study was performed on decay channels with good detection prospects, %
defined by high branching ratios and clean detector signatures.
Combining the scaled flux components with the decay probabilities and signal efficiencies, %
the 90\,\% C.L.\ sensitivity of the near detector has been estimated to all accessible channels.
Finally, a random matrix scan of different inverse seesaw realisations was performed %
to define regions of parameter space allowed by the model under consideration.
It is found that the near detector can extend current limits on mixing parameters, %
reaching regions of interests for neutrino mass models.
If a discovery is made, some considerations can be drawn upon the nature of the new fermionic states.


%In conclusion, the state of neutrino oscillation physics has come a long way in a relatively
%short time, with only a handful of unknowns left, all expected to be resolved with the
%next generation of experiments. This thesis has documented some of the essential work
%going towards the optimisation of the sensitivities of these experiments. But despite
%the experimental progress, on the theoretical side there are many potential explanations
%for what the mechanism is that determines the parameters of neutrino oscillations, and
%where the neutrino mass comes from to enable the phenomenon to occur. To make
%progress in this area, with the eventual hope of finding a successor to the Standard
%Model of particle physics, the measurements of the next generation of experiments will
%be essential.
