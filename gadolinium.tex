\clearpage
\chapter{Continuous gadolinium measurement}


In the next phase of the Super-Kamiokande experiment, gadolinium sulfate Gd\tped{2}(SO\tped{4})\tped{3} will be added to the detector,
which will improve the ability of the detector to identify neutrons, and therefore low energy $\nu_e$ %
through inverse beta decay (IBD).

\begin{figure}
	\centering
	\begin{fmffile}{ibd}
		\begin{fmfgraph*}(80,50)
			\fmfset{arrow_len}{3mm}
			\fmfleft{i2,i1}
			\fmfright{o2,o1}
			\fmf{fermion}{v1,i1}
			\fmf{fermion}{o1,v1}
			\fmf{boson, l.d = 1pt, l.s=left, label=$\!\!W^-\!\!$}{v1,v2}
			\fmf{fermion}{i2,v2}
			\fmf{fermion}{v2,o2}
			\fmfv{l=$\cj{\nu}_e$, l.d=2pt}{i1}
			\fmfv{l=$e^+$, l.d=2pt}{o1}
			\fmfv{l=$p$, l.d=3pt}{i2}
			\fmfv{l=$n$, l.d=3pt}{o2}
			\fmffreeze
			\begin{fmfgroup}
				\fmfset{arrow_len}{1.8mm}
				\fmfpen{0.6thick}
				\fmfi{fermion}{vpath (__i2,__v2) shifted (thick*(0,-2))}
				\fmfi{fermion}{vpath (__i2,__v2) shifted (thick*(0,-4))}
				\fmfi{fermion}{vpath (__v2,__o2) shifted (thick*(0,-2))}
				\fmfi{fermion}{vpath (__v2,__o2) shifted (thick*(0,-4))}
			\end{fmfgroup}
		\end{fmfgraph*}
	\end{fmffile}
	\caption{Inverse beta decay.}
\end{column}

One motivation to detect those neutrons is the search for relic SN neutrinos also know as
Diffuse Supernova Neutrino Background (DSNB) from all previous ccSNe in the universe. The
energy of these neutrinos should be around a few tens of MeV. The largest cross section in this
search corresponds to inverse beta decay (IBD): ν¯e + p → e
+ +n which is two orders of magnitude
larger than νe elastic scattering. Theoretical predictions provide with several limits but although
SK has the best world limit [1], DSNB has not been detected yet. The reason for that are the
current irreducible backgrounds: at low energies by invisible muon decay (muons below Cherenkov
threshold that decay into electrons) and at high energies by atmospheric neutrinos. The lower
energy threshold is defined by our ability to remove spallation events. Because we can detect the
positron with high efficiency only, this search is severely limited by these backgrounds. However,
if we could detect neutrons efficiently, these backgrounds could be reduced significantly and thus,
making the detection of the DSNB possible for the first time.

The isotopes \tapi{155}Gd and \tapi{157}Gd present very high cross-sections to thermal neutron capture, %
respectively \np{6.074e4}\,b and \np{2.537e5}\,b.
Hydrogen alone has a cross section of \np{3.321e-1}\,b.

Addition of water-soluble gadolinium salt Gd\tped{2}(SO\tped{4})\tped{3} %
enables a water Cherenkov detector to efficient identification of low energy anti-neutrinos, %
one of the main component of the Supernova Relic Neutrino (SNR) signal.

Capture of a thermal neutron on \textbf{hydrogen} has a cross section of %
$\np{332.6}\pm\np{0.7}$~mb, %
characteristic time of $\sim200$~\textmu s in water, emitting $\np{2.2}$~MeV gammas.
Capture on gadolinium has a cross section of $\np{49e3}$~b, emitting~$\sim\np{8}$~MeV gammas.
Characteristic time of $\sim30$~\textmu s for \np{0.2}\,\% Gd-water solution. 

The concentration of Gd in water affects the \textbf{efficiency} and \textbf{timing} of neutron captures, %
antineutrino measurements depends on this.

It is fundamental to measure the concentration \textbf{regularly}: %
it can change in time inside the tank (\textbf{temperature} and \textbf{flow} dependency).

Measure absorption lines of Gd, around 275~nm.
Use a UV source and a spectrometer.
Extract the absorbance $A$.
Absorption is directly proportional to Gd\%.


The automated Gadolinium absorbance detector (GAD) is progressing well.
The first prototype based on a 10 cm cell for 0.2$\%$ Gadolinium sulphate doping achieved %
a $3\%$ concentration measurement resolution ($1\%$ neutron capture efficiency) %
at the target concentration in comparison to the $3.5\%$ currently achieved via a manual mass spectrometer method.

This device was intended for deployment in EGADS in Japan but the Gadolinium sulphate %
concentration was changed to $0.02\%$ in preparation for Super-K phase 1 loading.
Therefore work progressed to building the version 2 prototype of GAD for %
sensitivity to $0.02\%$, by increasing the volume to $\approx 1$ meter.

This prototype has been successfully built and tested and shown a concentration measurement %
resolution of $1\%$ at both $0.02\%$ and $0.2\%$ doping.
Figure~\ref{fig:gad2} shows the new system and Figure~\ref{fig:gad2perf} shows the performance of the new system.
