\clearpage
\chapter{Continuous gadolinium measurement}

In the next phase of the Super-Kamiokande experiment, a gadolinium salt compound Gd\tped{2}(SO\tped{4})\tped{3} %
will be added to the detector~\cite{}, which will improve the ability of the detector to identify neutrons.
Neutrons are emitted in inverse beta decay (IBD) process, $\cj{\nu}_e + p \to e^+ + n$.
The current technique to tag neutrons in SK is by detecting capture on hydrogen nuclei.
Capture of a thermal neutron on hydrogen has a cross section of $\np{332.6}\pm\np{0.7}$~\,b, %
characteristic time of $\sim200$\,\textmu s in water, with the release of $\np{2.2}$\,MeV gammas.
After the prompt Cherenkov radiation from the charged lepton, the secondary signal from an electron, %
compton-scattered by the 2.2\,MeV photon, is looked for in a long time window.
However, in water what matters are the electrons Compton-scattered above the Cerenkov threshold by relatively hard gammas.
The detectable light following neutron capture on Gd (in thin foils) in possible discrete counters was carefully simulated %
for the SNO heavy water Cerenkov detector.
The equivalent single electron energy was found to peak at about 5 MeV, and range over 3--8\,MeV.
This spread reflects the intrinsic variation in the gamma cascades and the detector energy
resolution (the simulation had 5 photoelectrons per MeV of electron energy, compared to 6 in SK-I)~\cite{hep-ph/0309300}.
The energy threshold for SK is 5\,MeV, and the search of hydrogen-tagged neutrons is not efficient (20\%).
a long time window in which to look for the hydrogen signal, 
Capture on gadolinium has a cross section of $\np{49e3}$~b, which de-excites emitting a $\sim\np{8}$~MeV gamma-ray cascade.
By adding 0.2\,\% by mass of the Gd salt as Gd\tped{2}(SO\tped{4})\tped{3}, the characteristic time of capture is %
$\sim30$~\textmu s and an efficiency of 90\,\% can be achieved.
The isotopes \tapi{155}Gd and \tapi{157}Gd present very high cross-sections to thermal neutron capture, %
respectively \np{6.074e4}\,b and \np{2.537e5}\,b.
Hydrogen alone has a cross section of \np{3.321e-1}\,b.

One of the main motivation to detect neutrons is to be able to detect relic supernova neutrinos, %
also know as Diffuse Supernova Neutrino Background (DSNB), produced from all previous core-collapse supernova expolosions in the Universe.
This measurement will provide very important information on, for example, the mean energy spectrum %
of core collapse supernovae and the star formation rate of the universe.
This measurement is largely affected by irreducible backgrounds, like invisible muon decay at low energies---%
muons below Cherenkov threshold that decay into electrons---and at high energies by atmospheric neutrinos.
The lower energy threshold is defined by our ability to remove spallation events and solar neutrinos, %
that can be excluded effectively with neutron tagging.
The energy of these neutrinos should be around a few tens of MeV.
The largest cross section in this search corresponds to IBD, %
which is two orders of magnitude larger than $\nu_e$ elastic scattering.
The predictions provide with several limits but although SK has the best world limit, %
DSNB has not been detected yet.
However, if we could detect neutrons efficiently, these backgrounds could be reduced significantly and thus, %
making the detection of the DSNB possible for the first time.

Another motivation is searches of galactic supernova burst.
When a massive star at the end of its life collapses to a neutron star, %
it radiates almost all of its binding energy in the form of neutrinos, %
most of which have energies in the range 10 to 30\,MeV, and are emitted over a timescale of several tens of seconds.
These neutrinos are released just after core collapse, whereas the photon signal may take hours %
or days to emerge from the stellar envelope.
SK will acquire a huge number of neutrino events if the supernova is inside out galaxy, %
providing much information about early stages of the core-collapse process, %
its spectrum and time profile.
Start that are at the onset of their collapse, just after carbon ignition, will release most of its energy through neutrinos.
During the silicon burning stage, the neutrino luminosity is eight orders of magnitude less than %
in the peak at core-collapse, but while the later lasts just a few seconds, %
the Si burning phase takes several days.
During the Si burning phase roughly 1\,\% of the total energy of core-collapse are emitted through
``pre-supernova'' neutrinos with a monotonically increasing rate.

Improved neutron tagging will also be helpful to understand atmospheric and accelerator neutrino interactions and final states.
Also the separation of neutrino and antineutrino at the GeV scale will benefit from neutron tagging.
Finally, the background in proton decay searches can be reduced, since it requires no neutrons to appear in the final state.

%Addition of water-soluble gadolinium salt Gd\tped{2}(SO\tped{4})\tped{3} %
%enables a water Cherenkov detector to efficient identification of low energy anti-neutrinos, %
%one of the main component of the Supernova Relic Neutrino (SNR) signal.
%
%Capture of a thermal neutron on hydrogen has a cross section of $\np{332.6}\pm\np{0.7}$~mb, %
%characteristic time of $\sim200$~\textmu s in water, emitting $\np{2.2}$~MeV gammas.
%Capture on gadolinium has a cross section of $\np{49e3}$~b, emitting~$\sim\np{8}$~MeV gammas.
%Characteristic time of $\sim30$~\textmu s for \np{0.2}\,\% Gd-water solution. 

\begin{figure}
	\centering
	\begin{fmffile}{ibd}
		\begin{fmfgraph*}(100,75)
			\fmfset{arrow_len}{3mm}
			\fmfleft{i2,i1}
			\fmfright{o2,o1}
			\fmf{fermion}{v1,i1}
			\fmf{fermion}{o1,v1}
			\fmf{boson, l.d = 1pt, l.s=left, label=$\!\!W^-\!\!$}{v1,v2}
			\fmf{fermion}{i2,v2}
			\fmf{fermion}{v2,o2}
			\fmfv{l=$\cj{\nu}_e$, l.d=2pt}{i1}
			\fmfv{l=$e^+$, l.d=2pt}{o1}
			\fmfv{l=$p$, l.d=3pt}{i2}
			\fmfv{l=$n$, l.d=3pt}{o2}
			\fmffreeze
			\begin{fmfgroup}
				\fmfset{arrow_len}{1.8mm}
				\fmfpen{0.6thick}
				\fmfi{fermion}{vpath (__i2,__v2) shifted (thick*(0,-2))}
				\fmfi{fermion}{vpath (__i2,__v2) shifted (thick*(0,-4))}
				\fmfi{fermion}{vpath (__v2,__o2) shifted (thick*(0,-2))}
				\fmfi{fermion}{vpath (__v2,__o2) shifted (thick*(0,-4))}
			\end{fmfgroup}
		\end{fmfgraph*}
	\end{fmffile}
	\smallskip
	\caption{Inverse beta decay.}
\end{figure}


\section{Gadolinium sulfate}

In the original proposal by John Beacom and Mark Vagins, gadolinium trichloride GdCl\tped{3} was proposed, %
but the current full-scal plan has steered on gadolinium sulfate Gd\tped{2}(SO\tped{4})\tped{3} %
as the salt of choice (other candidate could be gadolinium nitrate Gd(NO\tped{3})\tped{3}.
There are a few requirements which a Gd compound must meet in order to become a good candidate for a full scale test.
The compound must be water soluble, and it should not be too difficult to dissolve large amounts.
The above three candidates can all be dissolved fairly easily, with the GdCl3 and Gd(NO3)3 only
needing stirring to fully dissolve, while the Gd2(SO4)3, can be forced into solution with the addition of %
a small amount of sulfuric acid, about 380 ml of acid for 28 kg of Gd2(SO4)3 in 14 tons of water (test done at EGADS).
So, this solubility requirement does not immediately rule out any of these three compounds.
Next, if the compound is to be used inside a very large water Cherenkov detector, such as SK, it must be safe for
the detectors components and tank, and it should be a non-toxic chemical, so it may be easily put into existing detectors.
While none of the above salts are toxic, they do have very different corrosive properties, due to the different anions of the molecule.
The nitrate and the sulfate turn out to be non-corrosive, and do not seem to affect detectors components much, if at all.
An extensive soak test study was carried out in Japan, with each of the 31 different materials inside the SK detector %
being soaked in both pure water and Gd2(SO4)3 solution.
Only the rubber used for the PMT pressure housing showed any difference between the pure water and the Gd2(SO4)3 solution, %
with further studies showing the rubber is unaffected by the Gd solution, if the temperature is kept below 15\,${}^\circ$C, %
which it is in the case of SK.
However, the chloride is corrosive, so for this reason the GdCl3 has been ruled out as a full scale test candidate.
Last the Gd compound solution must maintain a high level of optical transparency, %
so that light can be seen from the opposite side of the detector.
Although the GdCl3 has fairly good optical properties, it has been ruled out by the above.
Gd(NO3)3 is opaque in the UV region of the electromagnetic (EM) spectrum, for wavelengths less than 350\,nm, %
and this unfortunately is where a large fraction of Cherenkov light is detected in SK PMTs, %
so it cannot be a good full scale test candidate either.
This leaves Gd2(SO4)3, and it turns out to have good transparency in the UV and optical regions of the EM spectrum, %
leaving it as the best candidate for the EGADS test facility.
This is in fact the choice that has been made by the EGADS group, and the remaining studies described in this paper %
have been done using, or under the assumption that, Gd2(SO4)3 will be used.
One last exceptionally nice quality about this compound is today it costs only 5 US dollars per kilogram, %
meaning it would only be about 500,000 US dollars to dope SK [11], a price which is on the scale of such a major experiment.~\cite{1201.1017}.

\section{Gadolinium purity}

\section{Gadolinium concentration}

The concentration of Gd in water affects the efficiency and timing of neutron captures, antineutrino measurements depends on this.
It is fundamental to measure the concentration regularly, as this can change in time inside the tank.
On such a large scale, it is not a trivial task to predict temperature and flow dependency of the dissolved Gd.
At the moment, the Evaluating Gadolinium's Action on Detector Systems (EGADS) keeps track of the Gd concentration
with a Zeeman atomic absorption spectrometer, located near the experiment.
Water samples are collected from the EGADS tank, diluted and atomised inside the AAS machine.
The amount of Gd is then compared to known samples of Gd loaded water to determine the concentration.

We propose an alternative method, which still uses atomic absorption lines of Gd, but in solution in water.
Gadolinium presents strong emission/absorption lines in the UV region~\cite{NIST}.
Using a UV source and a spectrometer it would be possible to measure the absorption by gadolinium dissolved in water.
The absorption in directly proportional to the amount of gadolinium and therefore the concentration, %
where absorption is defined as
\begin{equation}
	\mathcal{A}(x) = \log_{10} \frac{I_0 (x)}{I(x)}\ ,
\end{equation}
where $I_0(x)$ is the reference intensity of the light source at wavelength $x$ and $I(x)$ %
is the measured intensity at the same wavelength.
The reference intensity would be using a pure water sample.

Using two absorption peaks the measurement could be improved by taking the difference of the absorption, %
as any effect which is wavelength independent is removed.
\begin{equation}
	\Delta \mathcal{A} = \mathcal{A}(x_1) - \mathcal{A}(x_2) = %
	\log_{10} \frac{I_0 (x_1)}{I(x_1)} - \log_{10} \frac{I_0 (x_2)}{I(x_2)}\ ,
\end{equation}

The automated Gadolinium absorbance detector (GAD) is progressing well.
The first prototype based on a 10 cm cell for 0.2$\%$ Gadolinium sulphate doping achieved %
a $3\%$ concentration measurement resolution ($1\%$ neutron capture efficiency) %
at the target concentration in comparison to the $3.5\%$ currently achieved via a manual mass spectrometer method.

This device was intended for deployment in EGADS in Japan but the Gadolinium sulphate %
concentration was changed to $0.02\%$ in preparation for Super-K phase 1 loading.
Therefore work progressed to building the version 2 prototype of GAD for %
sensitivity to $0.02\%$, by increasing the volume to $\approx 1$ meter.

This prototype has been successfully built and tested and shown a concentration measurement %
resolution of $1\%$ at both $0.02\%$ and $0.2\%$ doping.
