\chapter{Experimental techniques for neutrino detections}


Being electrically neutral and uncoloured particles, neutrinos can only interact through weak interactions.
For this reason, coupled with the small cross-sections typical of the weak force, the study of neutrino results %
in a challenging task.
Direct observation is unfeasible, thus detection must rely on weak interactions with matter, where %
their SU(2) charged counterparts are either produced or scattered, by CC or NC interactions respectively.
The physics is mediated by the lagrangians in Eq.~\ref{eq:cc} and~\ref{eq:nc}.

Large active volumes have to be employed, such that a significant number of neutrino can be considered and %
interaction probability is thus increased.
These apparatus are often built underground to shield the detector from cosmic rays and other barckground radiation.
Apart from providing matter to interact with, at the same time these volumes must be capable of detecting %
the scattered charged leptons.
Many are the materials or substances that can be used, like chlorine, gallium, solid or liquid scintillators.

One of the most promising techniques is to combine liquid argon with time projection chambers.
As with most other liquefied noble gases, argon has a high scintillation light yield %
(ca 51~photons/keV[arXiv:1004.0373]), is transparent to its own scintillation light, and is relatively easy to purify.
Compared to xenon, argon is also cheaper and has a distinct scintillation time profile which allows the separation %
of electronic recoils from nuclear recoils.

A more dated and better-known technology is the \emph{water Cherenkov} method, where the detector is used to %
record the Cherenkov light produced when the particles pass through pass through tanks full of purified water. 

\section{Water cherenkov}
\label{sec:wch}

The speed of light in vacuum is a universal constant, $c$, and it is a physical limit of the propagation %
of information, as stated by the special theory of Relativity.
However, when in a medium, light may travel at speed significantly less than $c$.
This reduction of speed depends on the relative permittivity, $\varepsilon$, of the material in which light is %
propagating.
Because of the non-zero real part of the dielectric constant, the electromagnetic (EM) field is modified and %
the phase velocity of light changes into
\begin{equation}
	\label{eq:light}
	v_P = \frac{c}{\sqrt{\varepsilon(\lambda)}} = \frac{c}{n(\lambda)}\,,
\end{equation}
where $n(\lambda) > 1$ is the \emph{refractive index} of the medium and %
depends on the wavelenght (energy) of the wave.

A charged particles moving at a constant velocity in a dielectric medium disrupts the local electromagnetic field, %
by deforming its molecules and temporarily polarising the material.
The dipoles are restored almost instantaneously and thus become impulsive sources of EM waves.
If the velocity of the passing particle, $v = \beta c$, is less than the speed of the light in the medium %
as expressed in Eq.~\ref{eq:light}, i.e.\ $\beta < 1/n$, then the total energy flux of the excited %
field is zero and EM waves are not irradiated.
On the contrary, if $\beta > 1/n$, the perturbance left by the passage of the particle is such that %
the energy is released coherently.
The result is that the field is different from zero in a cone coaxial with respect to the direction of %
the charged particle, whose direction is opposite to the particle motion.
As far as the photons are concerned, these are emitted coherently to a fixed angle with respect %
to the particle motion.
With the help of Fig.~\ref{fig:cherenkov}, it is easy to find that:
\begin{align}
	\sin \alpha &= \frac{1}{\beta n}\,\\
	\cos \theta &= \frac{1}{\beta n}\,.
\end{align}
where $\alpha$ is the apex angle of the cone and $\theta$ is the photon angle with respect to the particle direction.
For an ultra-relativistic particle, for which $\beta \sim 1$, there is a maximum angle of emission, given by:
\begin{equation}
	\cos \theta_{\mathrm{MAX}} = \frac{1}{n}\,.
\end{equation}

The phenomenon is called \emph{Cherenkov effect}, and it occurs everytime a charged particle passes through a %
dielectric medium at a speed:
\begin{equation}
	\label{eq:cherenkov}
	\beta > \frac{1}{n}\,.
\end{equation}
According to the theory of electromagnetic waves, a charged particle moving uniformly does not irradiate %
and this proves that the Cherenkov radiation is not related with Bremsstrahlung. 

This condition can be expressed in terms of the particle energy, given that $E^2 = p^2+m^2$ and %
$\beta = p/E$\footnote{For this calculation, the convention $c = 1$ is adopted.}.
The threshold becomes:
\begin{equation}
	\label{eq:ch_Eth}
	\frac{E}{m} > \frac{1}{\sqrt{1-1/n^2}}\,,
\end{equation}
with $m$ the mass of the charged particle.

The radiation is emitted in the visible and near visible regions of the EM spectrum, for which $ n > 1$.
A real medium is always dispersive and radiation is restricted to those frequencies bands for which %
$n(\nu) > \frac{1}{\beta}$.
In the x-ray region, for instance, $n(\nu)$ is always less than one and radiation is forbidden at this energies, %
because Eq.~\ref{eq:cherenkov} cannot be satisfied.

Truly, coherent emission of light needs two more conditions to be fulfilled:
\begin{itemize}
	\item the length of the track of the particle in the medium should be large compared with the wavelength, %
		$\lambda$, of the radiation in question, otherwise diffraction effects will become dominant;
	\item the velocity of the particle must be constant during its passage through the medium, or, %
		to be more specific, the differences in the times for particle to traverse successive $\lambda$ distances %
		should be small compared with the period $\frac{\lambda}{c}$ of the emitted light.
\end{itemize}

The number of photons emitted by a charged particle of charge $Ze$ per unit path length and per unit %
energy interval, or equivalently to $\lambda$, of the photons is equal to:
\begin{equation}
	\label{eq:ch_ph}
	\frac{\mathrm{d}^2N}{\mathrm{d}x\mathrm{d}\lambda} = \frac{2\pi\alpha Z^2}{\lambda^2} %
	\bigg(1-\frac{1}{\beta^2 n^2(\lambda)} \bigg)\,.
\end{equation}
This means that the greater part of Cherenkov photons are emitted in the ultraviolet range, because of the %
proportionality to $1/\lambda^2$.

\emph{Cherenkov detectors} take advantage of this effect, detecting the light produced by charge particles.
A large volume of transparent material, such as water, ice, or liquid scintillator, can be %
surrounded by lightsensitive detectors in order to capture the Cherenkov radiation.
This technique is largely used in neutrino detection, since they cannot be detected directly: the charged lepton, %
yeilded in CC or NC interactions, is observed.
From the light collected, it is possibile to reconstruct information on the interaction, such as the velocity %
of the charged particle, which is somehow related to the energy of the incident neutrino, or the position %
of the interaction vertex.
If the charged lepton drop under the Cherenkov threshold, the light is emitted in the shape of a ring, which %
further data can be inferred from.
Not every neutrino energy allows the production of a charged lepton\footnote{For instance, the CCQE process %
	$\bar\nu_e+p\rightarrow n+e^+$ has a energy threshold of \np{1.81}~MeV and the interaction %
	$\nu_\mu+n\rightarrow p+\mu^-$ has the threshold of \np{110.16}~MeV, because of the muon mass.}, %
but only MeV-scale neutrinos can be observed in a Cherenkov detector.


\section{Gadolinium neutron capture}
\label{sec:gdcap}

Current multi-kiloton scale water Cherenkov detectors, like Super-Kamiokande (SK), have provided %
many clues in the experimental understanding of the neutrino, be it originated in solar, atmospheric, or %
accelerator reactions.
However, in spite of the large lifetime of the experiment, some analyses are still limited by statistical uncertainty, %
and would benefit from increasing exposure.
Other analyses suffer from background contamination, as in the case of the supernova relic neutrinos (SRN) search, %
and would benefit more from the development of new background suppression techniques.
This hindrance can be overcome by studying the yield of neutrons in neutrino interactions, such as the %
\emph{inverse beta decay} (the antineutrino CCQE scattering).
It would allow a handle on antineutrinos rate, and possibly a method of background reduction for other studies.

Since neutrons are chargeless, they cannot interact with matter by means of the Coulomb force, %
which dominates the energy loss mechanisms for charged particles, described by the Bethe formula.
Neutrons can interact with nuclei in various way, depending on the energy:
\begin{itemize}
	\item elastic and inelastic scattering;
	\item transmutation;
	\item neutron activation;
	\item spallation reaction;
	\item neutron-induced fission;
\end{itemize}

As a result of the interaction, the neutron may either be absorbed, or change its energy and direction significantly.
In this way the average energy of a neutron beam can be completely or partly reduced up to thermal energies, %
close to \np{0.025}~eV.
In this range of energy, the neutron presents a different and often much larger effective neutron absorption %
cross-section for a given nuclide, compared to, for instance, fast neutrons, hence \emph{thermalisation} can %
result in a \emph{neutron activation} process.
This occurs when atomic nuclei capture free thermal neutrons, creating heavier nuclei, often in an excited state.
The excited nucles decays almost instantaneously emitting usually gamma rays.

The neutron energy distribution is adopted to the Maxwellian distribution known for thermal motion.
The time required by the thermalisation of neutrons follows an exponential, and the time constant is largely %
studied, [ref needed], amongst all the thermalisation in water.
It was found that neutron thermalisation in water has a time constant of 5$\mu$s [fujino, sumita, shiba].
Neutrons can be captured by either the hydrogen or the oxygen.
Free neutron will capture on a hydrogen nucleus, releasing a 2.2 MeV gamma.
In SK, for instance, this gamma results in about seven photo-electrons, and thus only detectable with $\simeq$20\,\% %
efficiency.

Gadolinium-157 has the highest thermal neutron capture cross-section among any stable %
nuclides: 259,000 barns.
Dissolving gadolinium compounds in water could considerably increase the neutron capture probability.
The neutron in water thermalises quickly and can thus be captured by a Gd nucleus with a probability of 90\,\%.
Upon capturing a neutron the Gd emits 3-4 gamma rays having a total energy of about 8 MeV.
The time and spatial correlation of the positron and neutron capture events ($20~\mu$s and 4~cm) %
can significantly reduce the backgrounds, and hence enhance the nu e signal events.
Even moderately energetic neutrons ranging from tens to hundreds of MeV will quickly lose energy %
by collisions with free protons and oxygen nuclei in water. 
Once thermalised, the neutrons undergo radiative capture, combining with a nearby nucleus to %
produce a more tightly bound final state, with excess energy released in a gamma-ray ( ) cascade. 
Gd-doped water enhances the capture cross-section compared to pure water %
(49,000 barns compared with 0.3 barns on a free proton) and, since the cascade happens %
at higher energies (8 MeV vs 2.2 MeV), it produces enough optical light to be reliably detected in %
a large target volume.


\section{Neutrino Production}
\label{sec:prod}

Numerous are the neutrino sources at the reach of neutrino experiments.
Neutrinos are produced in CC interactions, which can happen in nuclear reaction, as for \emph{solar} %
or \emph{reactor} neutrinos, or in cosmic rays impacts with the Earth's atmosphere, %
conveying energetic \emph{atmospheric} neutrinos.

Artificial neutrinos are also yielded in high-energy proton accelerators.
Accelerator neutrino beams are fundamental instrumental discovery tools in particle physics, in that more control %
less variables are involved.
Neutrino beams are derived from the decays of charged $\pi$ and $K$ mesons, which in turn are created from %
proton beams striking thick nuclear targets.
The precise selection and manipulation of the $\pi/K$ beam control the energy spectrum and type of neutrino beam.

The $\pi^{\pm}$ mesons have a mass of \np{139.6}~MeV and a mean lifetime of \np{2.6e-8}~s.
The primary decay mode of a pion, with a branching fraction of \np{99,9877}\,\%, is a leptonic %
decay into a muon and a muon neutrino:

%%
\begin{minipage}[c][3cm][c]{0.5\textwidth}
	\centering
	\begin{align}
		\pi^+ &\rightarrow \mu^+ + \nu_\mu \\
		\pi^- &\rightarrow \mu^- + \bar{\nu}_\mu
	\end{align}
\end{minipage}
%
\begin{minipage}[c][3cm][c]{0.5\textwidth}
	\centering
	\begin{fmffile}{pion_muon}
		\begin{fmfgraph*}(80,50)
			\fmfleft{i2,i1}
			\fmfright{o2,o1}
			\fmf{fermion}{i1,v1,i2}
			\fmf{photon}{v1,v2}
			\fmf{fermion}{o1,v2,o2}
			\fmf{photon, label=$W^\pm$}{v1,v2}
			\fmflabel{$\pi$}{v1}
			\fmflabel{$u,d$}{i1}
			\fmflabel{$\bar{d},\bar{u}$}{i2}
			\fmflabel{$\mu^\pm$}{o1}
			\fmflabel{$\nu_\mu,\bar{\nu}_\mu$}{o2}
		\end{fmfgraph*}
	\end{fmffile}
\end{minipage}
%%

The second most common decay mode of a pion, is the leptonic decay into an electron and the %
corresponding neutrino, $\pi^\pm \rightarrow \nu_e + e$.
In spite of the considerable differences in the space momentum, this process is suppressed %
with respect to the muonic one.
This effect is called \emph{helicity suppression} and is due to the great mass of the muon %
($m_\mu = \np{105.658}$~MeV) compared to the electron's ($m_e = \np{0.510}$~MeV); this results in a %
stronger helicity-chirality correspondence for the electron rather than for the muon.
Given that the $\pi$ mesons are spinless, neutrinos are left-handed, and antineutrinos are %
right-handed, the muonic channel is preferred because of spin and linear momentum preservation.
The suppression of the electronic decay mode with respect to the muonic one is given %
approximately within radiative corrections by the ratio:
\begin{equation}
	R_\pi = \Bigl ( \frac{m_e}{m_\mu} \Bigr )^2 
	\Bigl (\frac{m_\pi^2 - m_e^2}{m_\pi^2 - m_\mu^2} \Bigr )
	= \np{1.283e-4}
\end{equation}
The measured branching ratio of the electronic decay is indeed $(\np{1.23}\pm{0.02})\times10^{-4}$.

As far as the charged $K$ meson is concerned, it mainly decays in a muon and its correspective neutrino, %
with a branching ratio of \np{63.55}\,\%:

%%
\begin{minipage}[c][3cm][c]{0.5\textwidth}
	\centering
	\begin{align}
		K^+ &\rightarrow \mu^+ + \nu_\mu \\
		K^- &\rightarrow \mu^- + \bar{\nu}_\mu
	\end{align}
\end{minipage}
%
\begin{minipage}[c][3cm][c]{0.5\textwidth}
	\centering
	\begin{fmffile}{kaon_muon}
		\begin{fmfgraph*}(80,50)
			\fmfleft{i2,i1}
			\fmfright{o2,o1}
			\fmf{fermion}{i1,v1,i2}
			\fmf{photon, label=$W^\pm$}{v1,v2}
			\fmf{fermion}{o1,v2,o2}
			\fmflabel{$K$}{v1}
			\fmflabel{$u,s$}{i1}
			\fmflabel{$\bar{s},\bar{u}$}{i2}
			\fmflabel{$\mu^\pm$}{o1}
			\fmflabel{$\nu_\mu,\bar{\nu}_\mu$}{o2}
		\end{fmfgraph*}
	\end{fmffile}
\end{minipage}
%%

The second most frequent decay (\np{20.66}\,\%) is the decay into two pions, $K^{\pm} \rightarrow \pi^0 + \pi^\pm$.
Other decays have a branching ratio of 5\,\% or less and are listed in table Tab.~\ref{tab:kaons}.
On the contrary, the decays of the neutral kaon produce neutrino in few cases.
Because of the oscillation phenomenon given by the mixing between $K^0$ and $\bar K^0$, the neutral kaon has two %
manifestations, the short kaon $K_S$ and the long kaon $K_L$, named after their lifetimes.
While the $K$-short decays only in two pions ($2 \pi^0$ or $\pi^+ + \pi^-$), the $K$-long has a wider variety %
of final state combination, all of them involving three particles.
Among these, neutrinos are produced in the processes:

%%
\begin{minipage}[c][3cm][c]{0.5\textwidth}
	\centering
	\begin{align}
		K^0_L &\rightarrow \pi^\pm + \mu^\mp + \overset{(-)}{\nu}_\mu \\
		K^0_L &\rightarrow \pi^\pm + e^\mp + \overset{(-)}{\nu}_e 
	\end{align}
\end{minipage}
%
\begin{minipage}[c][3cm][c]{0.5\textwidth}
	\centering
	\begin{fmffile}{kaonlong}
		\begin{fmfgraph*}(70,70)
			\fmfleft{h,i}
			\fmfright{k,o3,o2,o1}
			\fmf{fermion}{i,v1,o3}
			\fmf{fermion}{h,w,k}
			\fmf{photon}{v1,v2}
			\fmf{fermion}{o1,v2,o2}
			\fmflabel{$d$}{h}
			\fmflabel{$d$}{k}
			\fmflabel{$s$}{i}
			\fmflabel{$\nu_\mu$}{o3}
			\fmflabel{$e^\pm$}{o2}
			\fmflabel{$\nu_e$}{o1}
		\end{fmfgraph*}
	\end{fmffile}
\end{minipage}
%%

\begin{table}
	\caption{Decay mode for a charged kaon, $K^\pm$, sorted by branching ration (in percent).}
	\label{tab:kaons}
	\[
		\begin{array}{lr}
			\toprule
			\mu^\pm + \overset{(-)}{\nu}_\mu	&	\np{65.55}\pm\np{0.11}	\\
			\midrule
			\pi^\pm + \pi^0			&	\np{20.66}\pm\np{0.08}	\\
			\midrule
			\pi^+ + \pi^\pm + \pi^-		&	\np{5.59}\pm\np{0.04}	\\
			\midrule
			\pi^0 + e^\pm + \overset{(-)}{\nu}_e	&	\np{5.07}\pm\np{0.04}	\\
			\midrule
			\pi^0 + \mu^\pm + \overset{(-)}{\nu}_\mu	&	\np{3.35}\pm\np{0.03}	\\
			\midrule
			\pi^\pm + \pi^0 + \pi^0		&	\np{1.76}\pm\np{0.02}	\\
			\bottomrule
		\end{array}
	\]
\end{table}

Neutrinos are also produced by the decay of muons.
Muons are unstable elementary particles and decay via the weak interaction. 
The dominant decay mode, called \emph{Michel decay}, is also the simplest possible:
because lepton numbers must be conserved, one of the product neutrinos of muon decay %
must be a muonic neutrino and the other an electronic antineutrino, along with an electron, %
because of the charge preservation.
Vice versa, an antimuon decay produces the corresponding antiparticles.
These two decays are:

%%
\begin{minipage}[c][3cm][c]{0.5\textwidth}
	\centering
	\begin{align}
		\label{eq:mupdecay}
		\mu^- &\rightarrow e^- + \bar\nu_e + \nu_\mu \\
		\label{eq:mundecay}
		\mu^+ &\rightarrow e^+ + \nu_e + \bar\nu_\mu
	\end{align}
\end{minipage}
%
\begin{minipage}[c][3.5cm][c]{0.5\textwidth}
	\centering
	\begin{fmffile}{mudecay}
		\begin{fmfgraph*}(70,70)
			\fmfleft{i}
			\fmfright{o3,o2,o1}
			\fmf{fermion}{i,v1,o3}
			\fmf{photon}{v1,v2}
			\fmf{fermion}{o1,v2,o2}
			\fmflabel{$\mu^\pm$}{i}
			\fmflabel{$\nu_\mu$}{o3}
			\fmflabel{$e^\pm$}{o2}
			\fmflabel{$\nu_e$}{o1}
		\end{fmfgraph*}
	\end{fmffile}
\end{minipage}
%%

The neutrino source provided by the muon decay, is more of a nuisance background, because of the long lifetime, %
which give rise to electronic component in neutrino spectrum.
Usually a beam absorbed is located at the end of the decay region of an accelerator line, to stop the hadronic and %
muonic component of the beam, and only an almost pure neutrino beam pointing towards te.

\section{Liquid argon time projection chamber}
