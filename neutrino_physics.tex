%%%%%%%%%%%%%%%%%%%%%%%%%%%%%%%		CHAP 1		%%%%%%%%%%%%%%%%%%%%%%%%%%%%%%%

\clearpage
\chapter{Neutrino in the Standard Model}
\label{cha:intro}

The Standard Model (SM) is a gauge theory that describes the strong, electromagnetic, and weak interactions %
of elementary particles in the framework of quantum field theory.
The theory is based on the local symmetry group 
\begin{equation}
	\label{eq:smgroup}
	\mathrm{SU(3)}_C \otimes \mathrm{SU(2)}_L \otimes \mathrm{U(1)}_Y
\end{equation}
where $C$, $L$ and $Y$ denote color, left-handed chirality and weak hypercharge, respectively.
The gauge group uniquely determines the interactions and the number of %
vector gauge bosons that correspond to the generators of the group.
They are eight massless gluons that mediate strong interactions, %
corresponding to the eight generators of SU(3)$_C$, and four gauge bosons, %
of which three are massive ($W^\pm$ and $Z$) and one is massless, corresponding %
to the three generators of SU(2)$_L$ and one generator of U(1)$_Y$, responsible for %
electroweak interactions.
The symmetry group of the SM fixes the interactions, i.e. the number and properties of the %
vector gauge bosons, with only three independent unknown parameters: the three coupling constants of %
the SU(3)$_C$, SU(2)$_L$, and U(1)$_Y$ groups, all of which must be determined from experiments.
On the contrary, the number and properties of scalar bosons and fermions are left unconstrained, %
except for the fact that they must transform according to the representations of the symmetry group, %
while the fermion representations must lead to the cancellation of quantum anomalies.
The known elementary fermions are divided in two categories, quarks and
leptons, according to the scheme:
\begin{center}
	\small
	\begin{tabular}{lccc}
		\toprule
		\textbf{Generation}	&\textbf{1st}	& \textbf{2nd}	& \textbf{3rd}	\\
		\midrule
		\multirow{2}*{Quark} & $u$ 		& $c$		& $t$		\\
		& $d$		& $s$		& $b$		\\
		\midrule
		\multirow{2}*{Letpons}	& $e$ 		& $\mu$		& $\tau$	\\
		& $\nu_e$	& $\nu_\mu$	& $\nu_\tau$	\\
		\bottomrule
	\end{tabular}
\end{center}
and their respective antiparticles.
They are distinguished by the fact that quarks participate in all the interactions % 
whereas leptons participate only in the electroweak interactions.

In the SM, electroweak interactions can be studied separately from strong interactions, %
because the symmetry under the color group is unbroken and there is no mixing %
between the SU(3)$_C$ and the $\mathrm{SU(2)}_L \otimes \mathrm{U(1)}_Y$ sectors.
On the other hand, the Glashow, Salam, and Weinberg theory well explains the group mixing between %
electromagnetic and weak interactions caused by a symmetry breaking process.
This model and the discovery of the predicted $W$ and $Z$ bosons, in addition to the gluon, %
the top, and charm quarks, made the fortune of the Standard Model.
Their redicted properties were experimentally confirmed with good precision and %
the recent discovery of the Higgs Boson is the last crowning achievement of SM.

Despite being the most successful theory of particle physics to date, the SM is actually limited %
in its approximation to reality, in that some clear evidences cannot be explained.
The most outstanding breakthrough is the neutrino oscillations, which was awarded the Nobel Prize in Physics in 2015 %
and has proved that the neutrinos are not all massless, as it is assumed by theory.
Mass terms for the neutrinos can be included in the SM, with the implications of theoretical problems.
Likewise, the SM is unable to provide an explanation of the observed asymmetry between matter and anti-matter.
It was noted by Sakharov that a solution to this puzzle would require some form of C and CP violation %
in the early Universe, along with Baryon number violation and out-of-equilibrium interactions.
These facts suggest that the Standard Model is not a complete theory and additional physics %
Beyond the Standard Model (BSM) is required.

The study of neutrinos is for sure one of the most promising probe to BSM physics and %
is of vital importance to the future development of particle physics, %
in particular through precision measurement of their interactions.
A deep understanding of neutrino interactions, and neutrino-nucleon interactions in particular, %
could lead to a great impact on long-baseline experiments, proton decay search, and supernova detection.

\section{Neutrino interactions}
\label{sec:inter}

Neutrinos are colourless and chargless particles, thus sensitive only to weak interactions.
Because of their nature, these leptons have small cross-sections and are difficult to measure.
All the interactions are described by the the electroweak part of the SM, based on the symmetry group %
$\mathrm{SU(2)}_L \otimes \mathrm{U(1)}_Y$, and are governed by the lagrangian %
$\mathcal{L} = \mathcal{L}^{(CC)} + \mathcal{L}^{(NC)}$.
In fact, neutrinos are mediated by the $W^\pm$ for charged-current (CC) interactions and by the $Z$ boson for %
neutral-current (NC) ones, whose respective lagrangians are:
\begin{align}
	\label{eq:cc}
	\mathcal{L}^{(CC)} &= - \frac{g}{2\sqrt{2}} \big (j^\mu_{W,L}W_\mu +\mathrm{h.c} \big ) \\
	\label{eq:nc}
	\mathcal{L}^{(NC)} &= - \frac{g}{2\cos\vartheta_{\mathrm{W}}} j^\mu_{Z,\nu}Z_\mu\,,
\end{align}
where the two currents are given by
\begin{align}
	j^\mu_{W,L}   &= 2 \sum_{\alpha = e, \mu, \tau} \overline{\nu_{\alpha L}} \gamma^\mu l_{\alpha L} \\
	j^\mu_{Z,\nu}   & = \sum_{\alpha = e, \mu, \tau} \overline{\nu_{\alpha L}} \gamma^\mu \nu_{\alpha L}\,,
\end{align}
and $\vartheta_{\mathrm{W}} = \np{28.7}^\circ$ is the Weinberg angle.

The easiest interaction that can be studied is the neutrino-electron elastic scattering
\begin{equation}
	\nu_\alpha + e^- \rightarrow \nu_\alpha + e^-\,,
\end{equation}
and its antineutrino counterpart\footnote{In terms of Feynam diagrams, the \emph{t}-channel diagram is %
	replaced by the \emph{s}-channel diagram.}.
For the electronic neutrino, both CC and NC interactions are allowed, while for $\alpha = \mu, \tau$ the %
charged-current interactions are forbidden.
The respective Feynman diagrams are shown in Fig.~\ref{fig:nescat} and~\ref{nutscat}.
For low neutrino energies, where the effects of the $W$ and $Z$ propagators can be neglected, %
the above processes are described by the effective charged-current and neutral-current lagrangians
\begin{align}
	\mathcal{L}_\mathrm{eff}(\nu_e e^- \rightarrow \nu_e e^-) &= - \frac{G_F}{\sqrt{2}} %
	[\overline{\nu_e}\gamma^\mu(1-\gamma^5)\nu_e][\bar{e}\gamma_\mu((1+g_V^l)-(1+g_A^l)\gamma^5)e] \\
	\mathcal{L}_\mathrm{eff}(\nu_\alpha e^- \rightarrow \nu_\alpha e^-) &= - \frac{G_F}{\sqrt{2}} %
	[\overline{\nu_\alpha}\gamma^\mu(1-\gamma^5)\nu_\alpha][\bar{e}\gamma_\mu(g_V^l-g_A^l)\gamma^5)e] %
	\quad (\alpha = \mu,\tau)\,.
\end{align}

The differential cross-section with respect to the momentum transfer $Q^2$
\begin{equation}
	\frac{\mathrm{d}\sigma}{\mathrm{d}Q^2} = \frac{G_F^2}{\pi}\bigg[g_1^2 + g_2^2\bigg(1 - %
	\frac{Q^2}{2p_\nu \cdot p_e} \bigg)^2 - g_1 g_2 m_e^2 \frac{Q^2}{2 (p_\nu \cdot p_e)^2} \bigg]\,.
\end{equation}
The quantities $g_1$ and $g_2$ depend on the flavour of the neutrino and related to the vector and axial couplings, %
$g_V$ and $g_A$.
They are:
\begin{align}
	g_1^{\nu_e} &= \frac{1}{2} + \sin^2\vartheta_W \quad , \quad
	g_2^{\nu_e} = \sin^2\vartheta_W \\
	g_1^{\nu_{\mu,\tau}} &= -\frac{1}{2} + \sin^2\vartheta_W \quad , \quad
	g_2^{\nu_{\mu,\tau}} = \sin^2\vartheta_W\,.
\end{align}

The differential cross-section as a function of the electron scattering angle in the laboratory frame is
\begin{equation}
	\begin{split}
		\frac{\mathrm{d}\sigma}{\mathrm{d}\cos\theta} = \sigma_0 \frac{4 E_\nu^2 (m_e+E_\nu)^2 \cos \theta}%
		{\big[(m_e+E_\nu)^2-E_\nu^2 \cos^2 \theta \big]^2} \bigg[&g_1^2 + g_2^2\bigg(1 - %
		\frac{2 m_e E_\nu \cos^2 \theta}{(m_e+E_\nu)^2-E_\nu^2 \cos^2 \theta} \bigg)^2 \\
		- &g_1 g_2 \frac{2m_e^2 \cos^2 \theta}{(m_e+E_\nu)^2-E_\nu^2 \cos^2 \theta} \bigg]\,,
	\end{split}
\end{equation}
where 
\begin{equation}
	\sigma_0 = \frac{2 G_F^2 m_e^2}{\pi}\,.
\end{equation}

For what concern the experiments, neutrino interactions with nucleons are easier to study thanks to the %
much larger cross-section and a more diverse range of processes, despite being less straightforward to %
deal with theoretically.
In general, these processes can be categorised according to the momentum transfer.
At small $Q^2$, elastic interactions dominate and may be brought about by both charged and neutral currents.
When this occurs via neutral currents, all flavour of neutrinos and anti-neutrinos can scatter off %
both neutrons and protons in what is referred to as ``NC elastic'' scattering.
The process is:
\begin{align}
	\nu_l + N \rightarrow \nu_l + N\,,
	\bar\nu_l + N \rightarrow \bar\nu_l + N\,,
\end{align}

Once neutrinos acquire sufficient energy they can also undergo the analogous charged current interactions, %
called ``quasi-elastic'', due to the fact that the recoiling nucleon changes its charge and mass transfer occurs.
The processes are
\begin{align}
	\nu_l + n &\rightarrow p + l^-\,\\
	\bar\nu_l + p &\rightarrow n + l^+\,,
\end{align}
with $l=e, \mu, \tau$.
For the muonic neutrino with energy below one GeV, the CCQE is the dominant interaction, event though the %
cross-section plateaus at higher energies, as the available $Q^2$ increases: it becomes increasingly unlikely %
for the nucleon to remain intact.

The physics behind the CC quasi-elastic processes is more complicated.
The differential cross-section for the scattering in the laboratory frame is given by
\begin{equation}
	\label{eq:cc_xsec_q}
	\frac{\mathrm{d} \sigma_{CC}}{\mathrm{d}Q^2} = \frac{G_F^2 |V_{ud}|^2 m_N^4}{8\pi (p_\nu \cdot p_N)^2} %
	\bigg [A(Q^2) \pm B(Q^2) \frac{s-u}{m_N^2} + C(Q^2) \frac{(s-u)^2}{m_N^4} \bigg]\,,
\end{equation}
where the plus sign refers to the $N = n$ interactions, while the minus sign to $N = p$.

\begin{equation}
	\label{eq:cc_xsec_t}
	\frac{\mathrm{d} \sigma_{CC}}{\mathrm{d}\cos\theta} = -\frac{G_F^2 |V_{ud}|^2 m_N^2}{4\pi} \frac{p_l}{E_\nu} %
	\bigg [A(Q^2) \pm B(Q^2) \frac{s-u}{m_N^2} + C(Q^2) \frac{(s-u)^2}{m_N^4} \bigg]\,,
\end{equation}

The functions $A(Q^2)$, $B(Q^2)$, and $C(Q^2)$ depends on the nucleon form-factors in the following way:
\begin{align}
	\begin{split}
		\label{eq:A(Q)}
		A &= \frac{m_l^2+Q^2}{m_N^2} \bigg\{ \bigg(1+\frac{Q^2}{4m_N^2}\bigg) G_A^2 - \bigg(1-\frac{Q^2}{4m_N^2}\bigg) %
		\bigg(F_1^2 - \frac{Q^2}{4m_N^2}F_2^2 \bigg) +\frac{Q^2}{m_N^2} F_1 F_2 \\
		&\qquad- \frac{m_l^2}{4m_N^2} %
		\bigg[ (F_1+F_2)^2+(G_A+2G_P)^2-\frac{1}{4}\bigg(1+\frac{Q^2}{4m_N^2}\bigg) G_P^2 \bigg] \bigg\}\, 
	\end{split}\\
	\label{eq:B(Q)}
	B &= \frac{Q^2}{m_N^2} G_A (F_1+F_2)\,\\
	\label{eq:C(Q)}
	C &= \frac{1}{4} \big (G_A^2 +F_1^2+\frac{Q^2}{4m_N^2}F_2^2\big)\,.
\end{align}

The form factors $F_1(Q^2)$, $F_2(Q^2)$, $G_A(Q^2)$, and $G_P(Q^2)$ are called, respectively, \emph{Dirac}, %
\emph{Pauli}, \emph{axial}, and \emph{pseudoscalar} weak charged-current form factors of the nucleon.
These funtions of $Q^2$ describe the spatial distributions of electric charge and current inside the nucleon %
and thus are intimately related to its internal structure.

CCQE interactions are particularly important to neutrino physics for mainly two reasons:
\begin{itemize}
	\item measurements of the differential cross-section in Eq.~\ref{eq:cc_xsec_q} give information on the %
		nucleon form-factors, which are difficult to measure; 
	\item their nature as two-body interactions enable the kinematics to be completely reconstructed, %
		and hence the initial neutrino energy determined which is critical for measuring the oscillation parameters.
\end{itemize}

In fact, if the target nucleon is at rest, at least compared to the neutrino energy, %
then this can be calculated as:
\begin{equation}
	E_\nu = \frac{m_n E_l + \frac{1}{2}\big ( m_p^2-m_n^2-m_l^2)}{m_n - E_l+p_l \cos \theta_l}\,,
\end{equation}
where the measurement of the momentum, $p_l$ and the angle with respect to the neutrino, $\theta_l$, of the %
outgoing charged lepton are only required.

Similar calculations can be made for the NCQE scatterings.
The cross-sections have the same form as the CC cross-sections in Eq.~\ref{eq:cc_xsec_q} and ~\ref{eq:cc_xsec_t}, %
without the mixing term $|V_{ud}|^2$ and with the proper nucleon form factors.
Since the values of the electromagnetic form factors, $F_1$ and $F_2$, are reasonably well known and the part %
in Eq.~\ref{eq:A(Q)} containing $G_P$ can be often neglected, thanks to the different mass magnitudes of %
leptons and nucleons, the axial form factor, $G_A$, can be determined through measurements of the charged-current %
quasielastic scattering processes.
On the contrary, measurements of the neutral-current elastic scattering cross-section give information %
on the \emph{strange} form factors of the nucleon, whose main contribute comes from the strange quark.


The low $Q^2$ region also presents an inelastic scattering contribution mostly affected by resonance production, %
where the nucleon is excited into a baryonic resonance before decaying.
At high $Q^2$, inelastic scattering is dominated by deep inelastic scattering (DIS), because the neutrino can scatter %
directly off a constituent quark, fragmenting the original nucleon.
In between these extreme scenarios, an additional contribution comes from interactions where the hadronic %
system is neither completely fragmented nor forms a recognisable resonance.
These interactions are referred to as ``shallow inelastic scattering'', and there is no clear model for dealing %
with them.
