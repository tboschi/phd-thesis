%%%%%%%%%%%%%%%%%%%%%%%%%%%%%%%		CHAP 1		%%%%%%%%%%%%%%%%%%%%%%%%%%%%%%%

\clearpage
\chapter{Neutrinos in the Standard Model and Beyond}
\label{cha:intro}

The Standard Model (SM) is a renormalisable Yang-Mills theory~\ref{} that describes the strong, electromagnetic, and weak interactions %
of elementary particles in the framework of quantum field theory.
It is based on the local gauge symmetry group 
\begin{equation}
	\label{eq:smgroup}
	\text{SU(3)}_C \otimes \text{SU(2)}_L \otimes \text{U(1)}_Y
\end{equation}
where $C$, $L$ and $Y$ denote respectively color, left-handed chirality and weak hypercharge.
The gauge group uniquely determines the interactions and the number of %
vector gauge bosons that correspond to the generators of the group.
%
The electroweak subgroup $\text{SU(2)}_L \otimes \text{U(1)}_Y$ undergoes a spontaneous symmetry breaking process %
out of which three of the four vector bosons acquire mass ($W^\pm$ and $Z$~bosons) and the last one, the photon, remains massless.
The colour section is unbroken and does not mix with the electroweak sector: %
the generators of the algebra of $\text{SU(3)}_C$ corresponds to eight massless gluons.
%
%%In the SM, electroweak interactions can be studied separately from strong interactions, %
%%because the symmetry under the color group is unbroken and there is no mixing %
%%between the SU(3)$_C$ and the $\mathrm{SU(2)}_L \otimes \mathrm{U(1)}_Y$ sectors.
%%On the other hand, the Glashow, Salam, and Weinberg theory well explains the group mixing between %
%%electromagnetic and weak interactions caused by a symmetry breaking process.
%%They are eight massless gluons that mediate strong interactions, %
%%corresponding to the eight generators of SU(3)$_C$, and four gauge bosons, %
%%of which three are massive ($W^\pm$ and $Z$) and one is massless, corresponding %
%%to the three generators of SU(2)$_L$ and one generator of U(1)$_Y$, responsible for %
%%electroweak interactions.
Since the number and properties of the gauge bosons is determined by the SM group, %
the only independent parameters left are the coupling constants of the interactions, which can be determined by the experiments.
The spontanous breaking symmetry requires at least one scalar boson thanks to the Higgs mechanism.
The recent discovery of the Higgs boson is the crowning achievemnt of the SM~\ref{}.
%The symmetry group of the SM fixes the interactions, i.e. the number and properties of the %
%vector gauge bosons, with only three independent unknown parameters: the three coupling constants of %
%the SU(3)$_C$, SU(2)$_L$, and U(1)$_Y$ groups, all of which must be determined from experiments.
On the contrary, the number and properties of scalar bosons and fermions are left unconstrained, %
except for the fact that they must transform according to the representations of the symmetry group, %
while the fermion representations must lead to the cancellation of quantum anomalies.

The known elementary fermions are divided in two categories, quarks and
leptons.
They are distinguished by the fact that quarks participate in all the interactions % 
whereas leptons participate only in the electroweak interactions.
\begin{center}
	\small
	\begin{tabular}{lccc}
		\toprule
		\textbf{Generation}	&\textbf{1st}	& \textbf{2nd}	& \textbf{3rd}	\\
		\midrule
		\multirow{2}*{Quark} & $u$ 		& $c$		& $t$		\\
		& $d$		& $s$		& $b$		\\
		\midrule
		\multirow{2}*{Letpons}	& $e$ 		& $\mu$		& $\tau$	\\
		& $\nu_e$	& $\nu_\mu$	& $\nu_\tau$	\\
		\bottomrule
	\end{tabular}
\end{center}

%In the SM, electroweak interactions can be studied separately from strong interactions, %
%because the symmetry under the color group is unbroken and there is no mixing %
%between the SU(3)$_C$ and the $\mathrm{SU(2)}_L \otimes \mathrm{U(1)}_Y$ sectors.
%On the other hand, the Glashow, Salam, and Weinberg theory well explains the group mixing between %
%electromagnetic and weak interactions caused by a symmetry breaking process.
%This model and the discovery of the predicted $W$ and $Z$ bosons, in addition to the gluon, %
%the top, and charm quarks, made the fortune of the Standard Model.
%Their redicted properties were experimentally confirmed with good precision and %
%the recent discovery of the Higgs Boson is the last crowning achievement of SM.

Despite being the most successful theory of particle physics to date, the SM is actually limited %
in its approximation to reality, in that some clear evidences cannot be explained.
The most outstanding breakthrough is the neutrino oscillations, which was awarded the Nobel Prize in Physics in 2015 %
and has proved that the neutrinos are not all massless, as it is assumed by theory.
Mass terms for the neutrinos can be included in the SM, with the implications of theoretical problems.
Likewise, the SM is unable to provide an explanation of the observed asymmetry between matter and anti-matter.
It was noted by Sakharov that a solution to this puzzle would require some form of C and CP violation %
in the early Universe, along with Baryon number violation and out-of-equilibrium interactions.
These facts suggest that the Standard Model is not a complete theory and additional physics %
Beyond the Standard Model (BSM) is required.

The study of neutrinos is for sure one of the most promising probe to BSM physics and %
is of vital importance to the future development of particle physics, %
in particular through precision measurement of their interactions.
A deep understanding of neutrino interactions, and neutrino-nucleon interactions in particular, %
could lead to a great impact on long-baseline experiments, proton decay search, and supernova detection.
Since the SM is a renormalizable theory, even its quantum corrections are insensitive to the physics beyond the SM.
Because of this reason, the SM is pheomenologically very successful and so far has been able to describe all the known
phenomena, except for the indications in favor of neutrino oscillations that we will discuss in the following chapters.

\section{Electroweak sector}
\label{sec:inter}

The electroweak (EW) sector of the SM is formed by the direct product of the weak isospin group $\text{SU(2)}_L$ and %
the hypercharge group $\text{U(1)}_Y$.
The two groups are connected by the Gell-Mann--Nishijima relation which connects the $I_3$ component of the %
weak isospin operator and the hypercharge operator $Y$  with the charge operator $Q$ as
\begin{equation}
	\label{eq:gellmann}
	Q = I_3 + \frac{Y}{2}\ .
\end{equation}
Left-handed chiral components of the fermion fields form doublets under $\text{SU(2)}_L$
\begin{equation}
	\label{eq:doublets}
	L_{\alpha L} = \mqty(\nu_{\alpha L} \\ \ell_{\alpha L}) \quad, \quad
	Q_{\alpha L} = \mqty(q^U_{\alpha L} \\ q^D_{\alpha L}) \ ,
\end{equation}
where $q^U$ and $q^D$ represent respectively the up quark and down quark families.
The right-handed fields, $\ell_{\alpha R}$, $q^U_{\alpha R}$, and $q^D_{\alpha R}$, transform as singlets.
Note that the right-handed components of the neutrino fields, $\nu_{\alpha R}$, are not historically considered in the SM %
because the neutrinos were believed massless until recent times~\ref{}.
As such, neutrinos are assumed to be massless in the SM.
The EW lagrangian is therefore the most general renormalizable Lagrangian invariant %
under the local symmetry $\text{SU(2)}_L \otimes \text{U(1)}_Y$:
\begin{align}
	\label{eq:ew_lagrangian}
	\mathcal{L} =\  &i\, \cj{L}_{\alpha L} \sh{D} L_{\alpha L} + i\, \cj{Q}_{\alpha L} \sh{D} Q_{\alpha L} + %
			 i\, \cj{\ell}_{\alpha R} \sh{D} \ell_{\alpha R} + i\, \cj{q^D}_{\alpha R} \sh{D} q^D_{\alpha R} + %
			 i\, \cj{q^U}_{\alpha R} \sh{D} q^U_{\alpha R} \notag \\ 
			&-\frac{1}{4} B_{\mu\nu} B^{\mu\nu} - \frac{1}{4} \vb{A}_{\mu\nu} \vb{A}^{\mu\nu} 
		      + (D_\mu H)^\dagger (D^\mu H) - \mu^2 H^\dagger H - \lambda (H^\dagger H)^2  \notag \\
		      &-\qty(Y^\ell_{\alpha \beta}\,\cj{L}_{\alpha L} H \ell_{\beta R} 
		      +Y^D_{\alpha \beta}\,\cj{Q}_{\alpha L} H q^D_{\beta R} 
		      +Y^U_{\alpha \beta}\,\cj{Q}_{\alpha L} \tilde{H} q^U_{\beta R} +\ \text{h.c.})\ ,
\end{align}
where the covariant derivative is 
\begin{equation}
	\label{eq:covariant}
	D_\mu = \pd_\mu + i g\, \vb{A}_\mu \cdot \vb{I} + i g'\, B_\mu \frac{Y}{2}\ ,
\end{equation}
which satisfies gauge invariance, %
and $\tilde{H} = i \sigma_2 H^*$ is the cojugate Higgs field. % thanks to the transformations
It is important to note that Dirac mass terms for fermion fields other than neutrinos %
are anyway forbidden by the gauge symmetry.
These terms will become manifest once the symmetry is broken through the Higgs mechanism (\refeq{sec:eq_higgs}).
%\begin{align}
%	\label{eq:gauge}
%	\vb{A}_\mu \cdot \vb{I} \longmapsto 
%\end{align}
The vector boson fields $\vb{A}^\mu = (A_1^\mu, A_2^\mu, A_3^\mu)$ and $B^\mu$ corresponds respectively %
to the three generators $\vb{I} = (I_1, I_2, I_3)$ of the $\text{SU(2)}_L$ group %
and the generator $Y$ of the $\text{U(1)}_Y$ group.
The $\text{SU(2)}_L$ generators are $I_a = \sigma_a / 2$, where $\sigma_a$ are the Pauli matrices, %
and thus satify the commutation relation
\begin{equation}
	\label{eq:generators}
	[I_a, I_b] = i \varepsilon_{a b c} I_c\ ,
\end{equation}
with $\varepsilon_{a b c}$ the Levi-Civita tensor.

\subsection{Electroweak interactions}
\label{sec:ew_interactions}

Expanding the covariant derivative and ignoring the kinetic terms, we can retrieve the interaction term %
for the leptonic sector
\begin{equation}
	\label{eq:interaction}
	\mathcal{L}_\text{int} = -\frac{1}{2} \sum_\alpha %
		\mqty(\cj{\nu}_{\alpha L} & \cj{\ell}_{\alpha L}) %
		\mqty( g \sh{A}_3 - g' \sh{B} & g(\sh{A}_1 - i \sh{A}_2) \\
		       g(\sh{A}_1 + i \sh{A}_2) & - g \sh{A}_3 - g' \sh{B}  ) %
		\mqty(\nu_{\alpha L} \\ \ell_{\alpha L} ) + g' \cj{\ell}_{\alpha R} \sh{B} \ell_{\alpha R}\ .
\end{equation}
Defining the combinations
\begin{align}
	\label{eq:fields}
	W^\mu &= \flatfrac{\qty(A_1^\mu - i A_2^\mu)}{\sqrt{2}} \\
	Z^\mu &= \cos \vartheta_\text{W} A_3^\mu - \cos \vartheta_\text{W} B^\mu \\
	A^\mu &= \sin \vartheta_\text{W} A_3^\mu + \cos \vartheta_\text{W} B^\mu \ ,
\end{align}
the electromagnetic field $A^\mu$ is expressed as a rotation of $A_e^\mu$ and $B^\mu$, thus recovering QED; %
the new field $Z^\mu$ also describes a neutral current process.
The lagrangian in \refeq{eq:interaction} can therefore be divided into two parts, %
$\mathcal{L}_\text{int} = \mathcal{L}^{(CC)} + \mathcal{L}^{(NC)}$, %
describing charged-current (CC) and neutral-current (NC) interactions.
These are
\begin{align}
	\label{eq:lagrangian_cc}
	\mathcal{L}_\text{CC} &= - \frac{g}{2\sqrt{2}} \sum_\alpha \cj{\nu}_{\alpha} \sh{W} (1-\gamma^5) \ell_{\alpha} + \text{h.c} \\
	\label{eq:lagrangian_nc}
	\mathcal{L}_\text{NC} &= - \frac{g}{2\cos\vartheta_\text{W}} %
		\sum_\alpha \qty[ \cj{\nu}_\alpha \sh{Z} (g^\nu_V - g^\nu_A \gamma^5) \nu_\alpha + %
				\cj{\ell}_\alpha \sh{Z} (g^\nu_v-g^\nu_A\gamma^5) \ell_\alpha] \notag \\
		&\hphantom{=} + g\sin\vartheta_\text{W} \sum_\alpha \cj{\ell}_\alpha \sh{A} \ell_\alpha \ .
\end{align}
where $g'$ has been re-written in terms of $g$ and $\vartheta_\text{W}$ by setting to zero the coupling of neutrinos %
to the electromagnetic field for neutrinos, which gives
\begin{equation}
	g \sin \vartheta_\text{W} = g' \cos \vartheta_\text{W}\ .
\end{equation}
Another important relation comes from the charged lepton couplings with the electromagnetic field which must coincide %
the QED lagrangian: we have that $g \sin\vartheta_\text{W} = q_e$ and so $g^2 + g'^2 = q_e^2$.
The~constants $g_V$ and $g_A$, introduced in \refeq{eq:lagrangian_nc}, can be defined for any fermion $f$ as
\begin{align}
	g_V^f &= I_3^f - 2 q^f \sin^2 \vartheta_\text{W} \\
	g_A^f &= I_3^f\ .
\end{align}
Thanks to this notation, the interaction lagrangian for the quark sector can be written in the same form of %
\refeqs{eq:lagrangian_cc}{eq:lagrangian_nc}, being careful of substituting the lepton fields with the quark fields.

\subsection{Higgs mechanism}
\label{sec:ew_higgs}

In the EW lagrangian in \refeq{eq:ew_lagrangian}, the Higgs $H$ is a complex scalar field and an $SU(2)_L$ doublet %
\begin{equation}
	\label{eq:higgs}
	H(x) = \mqty( H^+(x) \\ H^0(x) ) \ ,
\end{equation}
the potential of which can spontaneously break if $\lambda > 0$ and $\mu^2 < 0$, in
\begin{equation}
	\label{eq:higgs_potential}
	V(H) = \mu^2 H^\dagger H + \lambda (H^\dagger H)^2 \ .
\end{equation}
Defining
\begin{equation}
	\label{eq:vev}
	v \equiv \sqrt{- \frac{\mu^2}{\lambda}}\ ,
\end{equation}
the potential $V(H)$ shows a minimum for $H^\dagger H = v^2 / 2$, which %
corresponds to the lowest energy state, or vacuum.
In general, fermion and non-zero spin boson fields must have a vanishing vacuum expectation value (vev), %
so as to to preserve the Lorentz symmetries of space and time.
The same applies to charged scalar fields, as the vacuum is electrically chargeless.
On the other hand, neutral scalar fields can have a nonzero value in vacuum, and so the vev %
of the Higgs field could be given by
\begin{equation}
	\expval{H} = \frac{1}{\sqrt{2}} \mqty( 0 \\ v )\ .
\end{equation}
This value spontaneously breaks the EW group $\text{SU(2)}_L \times \text{U(1)}_Y$, %
but it remains invariant under the gauge transformations from the $\text{U(1)}_Q$ group, %
with $Q$ from \refeq{eq:gellmann}, which guarantees the existence of a massless gauge boson %
associated with the photon.
We can expand the scalar field around its vev and by choosing the unitary gauge three of the four real scalar fields %
can be rotated away being unphysical, simplifying to
\begin{equation}
	\label{eq:vev_higgs}
	\expval{H} = \frac{1}{\sqrt{2}} e^{} \mqty( 0 \\ v + h(x) )\ .
\end{equation}
Using the definition of the EW fields in \refeq{eq:fields}, %
the covariant derivative of \refeq{eq:covariant} applied to the Higgs is 
\begin{equation}
	\label{eq:d_higgs}
	D_\mu H(x) = \frac{1}{\sqrt{2}} %
		\mqty ( i \frac{g}{\sqrt{2}} W_\mu (x) \qty[v + h(x)] \\ %
			\pd_\mu h(x) - i \frac{g}{2\cos \vartheta_\text{W}} Z_\mu(x) [v+h(x)] )\ .
\end{equation}
The lagrangian terms with the Higgs field therefore becomes 
\begin{align}
	\label{eq:all_higgs}
	\mathcal{L}_\text{Higgs} =&\ \frac{1}{2} (\pd h)^2 - v^2 \lambda h^2 - \lambda h^3 - \frac{\lambda}{4} h^4 + %
			\frac{g^2 v^2}{4} W^\dagger_\mu W^\mu + \frac{g^2 v^2}{8\cos^2\vartheta_\text{W}} Z_\mu Z^\mu \notag \\
			& +\frac{g^2 v}{2} W^\dagger_\mu W^\mu h +\frac{g^2 v}{4\cos^2\vartheta_\text{W}} Z_\mu Z^\mu h 
			 +\frac{g^2}{4} W^\dagger_\mu W^\mu h^2 +  \frac{g^2}{8\cos^2\vartheta_\text{W}} Z_\mu Z^\mu h^2\ .
\end{align}
The second term of the first line is a mass term for the scalar fields, %
from which the mass of the Higgs boson is determined to be $m_H = v \sqrt(2\lambda) = \sqrt{-2 \mu^2}$, %
a value recentely measured~\ref{}.
The fifth and sixth terms represent the mass terms for the $W$ and $Z$ bosons, namely
\begin{equation}
	m_W = \frac{gv}{2} \quad, \quad m_Z = \frac{gv}{2\cos\vartheta_\text{W}}\ ,
\end{equation}
and the following parameter
\begin{equation}
	\rho = \frac{m_W^2}{m_Z^2 \cos^2\vartheta_\text{W}}
\end{equation}
is predicted to be $\rho = 1$ in the SM: experimentally it is measured to be $\rho = 0.999999$.
The other terms of \ref{eq:all_higgs} describe self-interactions of the Higgs and %
interactions with the $W$ and $Z$ vector bosons.

Applying the same expansion of \refeq{eq:vev_higgs} to the Yukawa terms of the SM lagrangian, %
we obtain a Dirac mass terms and trilinear couplings of the fermions with the Higgs.
There is no principle by which the Yukawa coupling matrices $Y_{\alpha \beta}$ should be diagonal \emph{a priori}, %
but without a diagonal matrix the fermion masses are not defined.
Being a generic complex matrix, the diagonalization can be achived by a biunitary transformation
\begin{equation} 
	V_L^\dagger\ \frac{v}{\sqrt{2}} Y'_{\alpha \beta}\ V_R = \frac{v}{\sqrt{2}} Y_{\alpha} = m_\alpha\, \delta_{\alpha \beta}\\ ,
\end{equation} 
where $V_L$ and $V_R$ are unitary matrices.
This transformation acts on the fermionic fields as
\begin{align}
	\ell_L = V_L^\dagger \ell_L' \quad , \quad \ell_R = V_R^\dagger \ell_R' \\ 
	\ell_Q = V_L^\dagger \ell_L' \quad , \quad \ell_R = V_Q^\dagger \ell_R' \\ 
\end{align}

After diagonalising the Yukawa coupling $Y_{\alpha \beta} = y_\alpha \delta^\alpha_\beta$, we have in the lepton sector
\begin{equation}
	\label{eq:lepton_mass}
	\mathcal{L}_{H, L} = - \sum_\alpha \frac{y_\alpha^\ell v}{\sqrt{2}}\ \cj{\ell}_\alpha\, \ell_\alpha %
			     - \sum_\alpha \frac{y_\alpha^\ell}{\sqrt{2}}\   \cj{\ell}_\alpha\, \ell_\alpha\,h \ ,
\end{equation}
where the mass of the charged leptons is given by 
\begin{equation}
	\label{eq:dirac_mass}
	m_\alpha = \frac{y_\alpha^\ell\ v}{\sqrt{2}}\ .
\end{equation}
The same is found in the quark sector, where the right Yukawa couplings should be used with the corresponding diagonalisation:
\begin{equation}
	\label{eq:quark_mass}
	\mathcal{L}_{H, Q} = - \sum_\alpha \qty(\frac{y_\alpha^D v}{\sqrt{2}}\ \cj{q}^D_\alpha\, q^D_\alpha %
					      + \frac{y_\alpha^U v}{\sqrt{2}}\ \cj{q}^U_\alpha\, q^U_\alpha) %
			     - \sum_\alpha \qty(\frac{y_\alpha^D}{\sqrt{2}}\ \cj{q}^D_\alpha\, q^D_\alpha\,h %
			     		      + \frac{y_\alpha^U}{\sqrt{2}}\ \cj{q}^U_\alpha\, q^U_\alpha\,h) \ ,
\end{equation}
with masses
\begin{equation}
	\label{eq:dirac_mass}
	m_\alpha = \frac{y_\alpha^D\ v}{\sqrt{2}} \quad \text{and} \quad  %
	m_\beta = \frac{y_\beta^U\ v}{\sqrt{2}} 
\end{equation}
for $\alpha = d,\,s,\,b$ and $\beta = u,\,c,\,t$.


\section{Neutrino oscillations}
\label{sec:neutrino_oscillations}



\section{Neutrino interactions}
\label{sec:neutrino_interactions}

The easiest interaction that can be studied is the neutrino-electron elastic scattering
\begin{equation}
	\nu_\alpha + e^- \rightarrow \nu_\alpha + e^-\,,
\end{equation}
and its antineutrino counterpart\footnote{In terms of Feynam diagrams, the \emph{t}-channel diagram is %
	replaced by the \emph{s}-channel diagram.}.
For the electronic neutrino, both CC and NC interactions are allowed, while for $\alpha = \mu, \tau$ the %
charged-current interactions are forbidden.
The respective Feynman diagrams are shown in Fig.~\ref{fig:nescat} and~\ref{nutscat}.
For low neutrino energies, where the effects of the $W$ and $Z$ propagators can be neglected, %
the above processes are described by the effective charged-current and neutral-current lagrangians
\begin{align}
	\mathcal{L}_\mathrm{eff}(\nu_e e^- \rightarrow \nu_e e^-) &= - \frac{G_F}{\sqrt{2}} %
	[\overline{\nu_e}\gamma^\mu(1-\gamma^5)\nu_e][\bar{e}\gamma_\mu((1+g_V^l)-(1+g_A^l)\gamma^5)e] \\
	\mathcal{L}_\mathrm{eff}(\nu_\alpha e^- \rightarrow \nu_\alpha e^-) &= - \frac{G_F}{\sqrt{2}} %
	[\overline{\nu_\alpha}\gamma^\mu(1-\gamma^5)\nu_\alpha][\bar{e}\gamma_\mu(g_V^l-g_A^l)\gamma^5)e] %
	\quad (\alpha = \mu,\tau)\,.
\end{align}

The differential cross-section with respect to the momentum transfer $Q^2$
\begin{equation}
	\frac{\mathrm{d}\sigma}{\mathrm{d}Q^2} = \frac{G_F^2}{\pi}\bigg[g_1^2 + g_2^2\bigg(1 - %
	\frac{Q^2}{2p_\nu \cdot p_e} \bigg)^2 - g_1 g_2 m_e^2 \frac{Q^2}{2 (p_\nu \cdot p_e)^2} \bigg]\,.
\end{equation}
The quantities $g_1$ and $g_2$ depend on the flavour of the neutrino and related to the vector and axial couplings, %
$g_V$ and $g_A$.
They are:
\begin{align}
	g_1^{\nu_e} &= \frac{1}{2} + \sin^2\vartheta_W \quad , \quad
	g_2^{\nu_e} = \sin^2\vartheta_W \\
	g_1^{\nu_{\mu,\tau}} &= -\frac{1}{2} + \sin^2\vartheta_W \quad , \quad
	g_2^{\nu_{\mu,\tau}} = \sin^2\vartheta_W\,.
\end{align}

The differential cross-section as a function of the electron scattering angle in the laboratory frame is
\begin{equation}
	\begin{split}
		\frac{\mathrm{d}\sigma}{\mathrm{d}\cos\theta} = \sigma_0 \frac{4 E_\nu^2 (m_e+E_\nu)^2 \cos \theta}%
		{\big[(m_e+E_\nu)^2-E_\nu^2 \cos^2 \theta \big]^2} \bigg[&g_1^2 + g_2^2\bigg(1 - %
		\frac{2 m_e E_\nu \cos^2 \theta}{(m_e+E_\nu)^2-E_\nu^2 \cos^2 \theta} \bigg)^2 \\
		- &g_1 g_2 \frac{2m_e^2 \cos^2 \theta}{(m_e+E_\nu)^2-E_\nu^2 \cos^2 \theta} \bigg]\,,
	\end{split}
\end{equation}
where 
\begin{equation}
	\sigma_0 = \frac{2 G_F^2 m_e^2}{\pi}\,.
\end{equation}

For what concern the experiments, neutrino interactions with nucleons are easier to study thanks to the %
much larger cross-section and a more diverse range of processes, despite being less straightforward to %
deal with theoretically.
In general, these processes can be categorised according to the momentum transfer.
At small $Q^2$, elastic interactions dominate and may be brought about by both charged and neutral currents.
When this occurs via neutral currents, all flavour of neutrinos and anti-neutrinos can scatter off %
both neutrons and protons in what is referred to as ``NC elastic'' scattering.
The process is:
\begin{align}
	\nu_l + N \rightarrow \nu_l + N\,,
	\bar\nu_l + N \rightarrow \bar\nu_l + N\,,
\end{align}

Once neutrinos acquire sufficient energy they can also undergo the analogous charged current interactions, %
called ``quasi-elastic'', due to the fact that the recoiling nucleon changes its charge and mass transfer occurs.
The processes are
\begin{align}
	\nu_l + n &\rightarrow p + l^-\,\\
	\bar\nu_l + p &\rightarrow n + l^+\,,
\end{align}
with $l=e, \mu, \tau$.
For the muonic neutrino with energy below one GeV, the CCQE is the dominant interaction, event though the %
cross-section plateaus at higher energies, as the available $Q^2$ increases: it becomes increasingly unlikely %
for the nucleon to remain intact.

The physics behind the CC quasi-elastic processes is more complicated.
The differential cross-section for the scattering in the laboratory frame is given by
\begin{equation}
	\label{eq:cc_xsec_q}
	\frac{\mathrm{d} \sigma_{CC}}{\mathrm{d}Q^2} = \frac{G_F^2 |V_{ud}|^2 m_N^4}{8\pi (p_\nu \cdot p_N)^2} %
	\bigg [A(Q^2) \pm B(Q^2) \frac{s-u}{m_N^2} + C(Q^2) \frac{(s-u)^2}{m_N^4} \bigg]\,,
\end{equation}
where the plus sign refers to the $N = n$ interactions, while the minus sign to $N = p$.

\begin{equation}
	\label{eq:cc_xsec_t}
	\frac{\mathrm{d} \sigma_{CC}}{\mathrm{d}\cos\theta} = -\frac{G_F^2 |V_{ud}|^2 m_N^2}{4\pi} \frac{p_l}{E_\nu} %
	\bigg [A(Q^2) \pm B(Q^2) \frac{s-u}{m_N^2} + C(Q^2) \frac{(s-u)^2}{m_N^4} \bigg]\,,
\end{equation}

The functions $A(Q^2)$, $B(Q^2)$, and $C(Q^2)$ depends on the nucleon form-factors in the following way:
\begin{align}
	\begin{split}
		\label{eq:A(Q)}
		A &= \frac{m_l^2+Q^2}{m_N^2} \bigg\{ \bigg(1+\frac{Q^2}{4m_N^2}\bigg) G_A^2 - \bigg(1-\frac{Q^2}{4m_N^2}\bigg) %
		\bigg(F_1^2 - \frac{Q^2}{4m_N^2}F_2^2 \bigg) +\frac{Q^2}{m_N^2} F_1 F_2 \\
		&\qquad- \frac{m_l^2}{4m_N^2} %
		\bigg[ (F_1+F_2)^2+(G_A+2G_P)^2-\frac{1}{4}\bigg(1+\frac{Q^2}{4m_N^2}\bigg) G_P^2 \bigg] \bigg\}\, 
	\end{split}\\
	\label{eq:B(Q)}
	B &= \frac{Q^2}{m_N^2} G_A (F_1+F_2)\,\\
	\label{eq:C(Q)}
	C &= \frac{1}{4} \big (G_A^2 +F_1^2+\frac{Q^2}{4m_N^2}F_2^2\big)\,.
\end{align}

The form factors $F_1(Q^2)$, $F_2(Q^2)$, $G_A(Q^2)$, and $G_P(Q^2)$ are called, respectively, \emph{Dirac}, %
\emph{Pauli}, \emph{axial}, and \emph{pseudoscalar} weak charged-current form factors of the nucleon.
These funtions of $Q^2$ describe the spatial distributions of electric charge and current inside the nucleon %
and thus are intimately related to its internal structure.

CCQE interactions are particularly important to neutrino physics for mainly two reasons:
\begin{itemize}
	\item measurements of the differential cross-section in Eq.~\ref{eq:cc_xsec_q} give information on the %
		nucleon form-factors, which are difficult to measure; 
	\item their nature as two-body interactions enable the kinematics to be completely reconstructed, %
		and hence the initial neutrino energy determined which is critical for measuring the oscillation parameters.
\end{itemize}

In fact, if the target nucleon is at rest, at least compared to the neutrino energy, %
then this can be calculated as:
\begin{equation}
	E_\nu = \frac{m_n E_l + \frac{1}{2}\big ( m_p^2-m_n^2-m_l^2)}{m_n - E_l+p_l \cos \theta_l}\,,
\end{equation}
where the measurement of the momentum, $p_l$ and the angle with respect to the neutrino, $\theta_l$, of the %
outgoing charged lepton are only required.

Similar calculations can be made for the NCQE scatterings.
The cross-sections have the same form as the CC cross-sections in Eq.~\ref{eq:cc_xsec_q} and ~\ref{eq:cc_xsec_t}, %
without the mixing term $|V_{ud}|^2$ and with the proper nucleon form factors.
Since the values of the electromagnetic form factors, $F_1$ and $F_2$, are reasonably well known and the part %
in Eq.~\ref{eq:A(Q)} containing $G_P$ can be often neglected, thanks to the different mass magnitudes of %
leptons and nucleons, the axial form factor, $G_A$, can be determined through measurements of the charged-current %
quasielastic scattering processes.
On the contrary, measurements of the neutral-current elastic scattering cross-section give information %
on the \emph{strange} form factors of the nucleon, whose main contribute comes from the strange quark.

The low $Q^2$ region also presents an inelastic scattering contribution mostly affected by resonance production, %
where the nucleon is excited into a baryonic resonance before decaying.
At high $Q^2$, inelastic scattering is dominated by deep inelastic scattering (DIS), because the neutrino can scatter %
directly off a constituent quark, fragmenting the original nucleon.
In between these extreme scenarios, an additional contribution comes from interactions where the hadronic %
system is neither completely fragmented nor forms a recognisable resonance.
These interactions are referred to as ``shallow inelastic scattering'', and there is no clear model for dealing %
with them.

\section{Neutrino sources}
\label{sec:prod}

Numerous are the neutrino sources at the reach of neutrino experiments.
Neutrinos are produced in CC interactions, which can happen in nuclear reaction, as for \emph{solar} %
or \emph{reactor} neutrinos, or in cosmic rays impacts with the Earth's atmosphere, %
conveying energetic \emph{atmospheric} neutrinos.

Artificial neutrinos are also yielded in high-energy proton accelerators.
Accelerator neutrino beams are fundamental instrumental discovery tools in particle physics, in that more control %
less variables are involved.
Neutrino beams are derived from the decays of charged $\pi$ and $K$ mesons, which in turn are created from %
proton beams striking thick nuclear targets.
The precise selection and manipulation of the $\pi/K$ beam control the energy spectrum and type of neutrino beam.

The $\pi^{\pm}$ mesons have a mass of \np{139.6}~MeV and a mean lifetime of \np{2.6e-8}~s.
The primary decay mode of a pion, with a branching fraction of \np{99,9877}\,\%, is a leptonic %
decay into a muon and a muon neutrino:

%%
\begin{minipage}[c][3cm][c]{0.5\textwidth}
	\centering
	\begin{align}
		\pi^+ &\rightarrow \mu^+ + \nu_\mu \\
		\pi^- &\rightarrow \mu^- + \bar{\nu}_\mu
	\end{align}
\end{minipage}
%
\begin{minipage}[c][3cm][c]{0.5\textwidth}
	\centering
	\begin{fmffile}{pion_muon}
		\begin{fmfgraph*}(80,50)
			\fmfleft{i2,i1}
			\fmfright{o2,o1}
			\fmf{fermion}{i1,v1,i2}
			\fmf{photon}{v1,v2}
			\fmf{fermion}{o1,v2,o2}
			\fmf{photon, label=$W^\pm$}{v1,v2}
			\fmflabel{$\pi$}{v1}
			\fmflabel{$u,d$}{i1}
			\fmflabel{$\bar{d},\bar{u}$}{i2}
			\fmflabel{$\mu^\pm$}{o1}
			\fmflabel{$\nu_\mu,\bar{\nu}_\mu$}{o2}
		\end{fmfgraph*}
	\end{fmffile}
\end{minipage}
%%

The second most common decay mode of a pion, is the leptonic decay into an electron and the %
corresponding neutrino, $\pi^\pm \rightarrow \nu_e + e$.
In spite of the considerable differences in the space momentum, this process is suppressed %
with respect to the muonic one.
This effect is called \emph{helicity suppression} and is due to the great mass of the muon %
($m_\mu = \np{105.658}$~MeV) compared to the electron's ($m_e = \np{0.510}$~MeV); this results in a %
stronger helicity-chirality correspondence for the electron rather than for the muon.
Given that the $\pi$ mesons are spinless, neutrinos are left-handed, and antineutrinos are %
right-handed, the muonic channel is preferred because of spin and linear momentum preservation.
The suppression of the electronic decay mode with respect to the muonic one is given %
approximately within radiative corrections by the ratio:
\begin{equation}
	R_\pi = \Bigl ( \frac{m_e}{m_\mu} \Bigr )^2 
	\Bigl (\frac{m_\pi^2 - m_e^2}{m_\pi^2 - m_\mu^2} \Bigr )
	= \np{1.283e-4}
\end{equation}
The measured branching ratio of the electronic decay is indeed $(\np{1.23}\pm{0.02})\times10^{-4}$.

As far as the charged $K$ meson is concerned, it mainly decays in a muon and its correspective neutrino, %
with a branching ratio of \np{63.55}\,\%:

%%
\begin{minipage}[c][3cm][c]{0.5\textwidth}
	\centering
	\begin{align}
		K^+ &\rightarrow \mu^+ + \nu_\mu \\
		K^- &\rightarrow \mu^- + \bar{\nu}_\mu
	\end{align}
\end{minipage}
%
\begin{minipage}[c][3cm][c]{0.5\textwidth}
	\centering
	\begin{fmffile}{kaon_muon}
		\begin{fmfgraph*}(80,50)
			\fmfleft{i2,i1}
			\fmfright{o2,o1}
			\fmf{fermion}{i1,v1,i2}
			\fmf{photon, label=$W^\pm$}{v1,v2}
			\fmf{fermion}{o1,v2,o2}
			\fmflabel{$K$}{v1}
			\fmflabel{$u,s$}{i1}
			\fmflabel{$\bar{s},\bar{u}$}{i2}
			\fmflabel{$\mu^\pm$}{o1}
			\fmflabel{$\nu_\mu,\bar{\nu}_\mu$}{o2}
		\end{fmfgraph*}
	\end{fmffile}
\end{minipage}
%%

The second most frequent decay (\np{20.66}\,\%) is the decay into two pions, $K^{\pm} \rightarrow \pi^0 + \pi^\pm$.
Other decays have a branching ratio of 5\,\% or less and are listed in table Tab.~\ref{tab:kaons}.
On the contrary, the decays of the neutral kaon produce neutrino in few cases.
Because of the oscillation phenomenon given by the mixing between $K^0$ and $\bar K^0$, the neutral kaon has two %
manifestations, the short kaon $K_S$ and the long kaon $K_L$, named after their lifetimes.
While the $K$-short decays only in two pions ($2 \pi^0$ or $\pi^+ + \pi^-$), the $K$-long has a wider variety %
of final state combination, all of them involving three particles.
Among these, neutrinos are produced in the processes:

%%
\begin{minipage}[c][3cm][c]{0.5\textwidth}
	\centering
	\begin{align}
		K^0_L &\rightarrow \pi^\pm + \mu^\mp + \overset{(-)}{\nu}_\mu \\
		K^0_L &\rightarrow \pi^\pm + e^\mp + \overset{(-)}{\nu}_e 
	\end{align}
\end{minipage}
%
\begin{minipage}[c][3cm][c]{0.5\textwidth}
	\centering
	\begin{fmffile}{kaonlong}
		\begin{fmfgraph*}(70,70)
			\fmfleft{h,i}
			\fmfright{k,o3,o2,o1}
			\fmf{fermion}{i,v1,o3}
			\fmf{fermion}{h,w,k}
			\fmf{photon}{v1,v2}
			\fmf{fermion}{o1,v2,o2}
			\fmflabel{$d$}{h}
			\fmflabel{$d$}{k}
			\fmflabel{$s$}{i}
			\fmflabel{$\nu_\mu$}{o3}
			\fmflabel{$e^\pm$}{o2}
			\fmflabel{$\nu_e$}{o1}
		\end{fmfgraph*}
	\end{fmffile}
\end{minipage}
%%

\begin{table}
	\caption{Decay mode for a charged kaon, $K^\pm$, sorted by branching ration (in percent).}
	\label{tab:kaons}
	\[
		\begin{array}{lr}
			\toprule
			\mu^\pm + \overset{(-)}{\nu}_\mu	&	\np{65.55}\pm\np{0.11}	\\
			\midrule
			\pi^\pm + \pi^0			&	\np{20.66}\pm\np{0.08}	\\
			\midrule
			\pi^+ + \pi^\pm + \pi^-		&	\np{5.59}\pm\np{0.04}	\\
			\midrule
			\pi^0 + e^\pm + \overset{(-)}{\nu}_e	&	\np{5.07}\pm\np{0.04}	\\
			\midrule
			\pi^0 + \mu^\pm + \overset{(-)}{\nu}_\mu	&	\np{3.35}\pm\np{0.03}	\\
			\midrule
			\pi^\pm + \pi^0 + \pi^0		&	\np{1.76}\pm\np{0.02}	\\
			\bottomrule
		\end{array}
	\]
\end{table}

Neutrinos are also produced by the decay of muons.
Muons are unstable elementary particles and decay via the weak interaction. 
The dominant decay mode, called \emph{Michel decay}, is also the simplest possible:
because lepton numbers must be conserved, one of the product neutrinos of muon decay %
must be a muonic neutrino and the other an electronic antineutrino, along with an electron, %
because of the charge preservation.
Vice versa, an antimuon decay produces the corresponding antiparticles.
These two decays are:

%%
\begin{minipage}[c][3cm][c]{0.5\textwidth}
	\centering
	\begin{align}
		\label{eq:mupdecay}
		\mu^- &\rightarrow e^- + \bar\nu_e + \nu_\mu \\
		\label{eq:mundecay}
		\mu^+ &\rightarrow e^+ + \nu_e + \bar\nu_\mu
	\end{align}
\end{minipage}
%
\begin{minipage}[c][3.5cm][c]{0.5\textwidth}
	\centering
	\begin{fmffile}{mudecay}
		\begin{fmfgraph*}(70,70)
			\fmfleft{i}
			\fmfright{o3,o2,o1}
			\fmf{fermion}{i,v1,o3}
			\fmf{photon}{v1,v2}
			\fmf{fermion}{o1,v2,o2}
			\fmflabel{$\mu^\pm$}{i}
			\fmflabel{$\nu_\mu$}{o3}
			\fmflabel{$e^\pm$}{o2}
			\fmflabel{$\nu_e$}{o1}
		\end{fmfgraph*}
	\end{fmffile}
\end{minipage}
%%

The neutrino source provided by the muon decay, is more of a nuisance background, because of the long lifetime, %
which give rise to electronic component in neutrino spectrum.
Usually a beam absorbed is located at the end of the decay region of an accelerator line, to stop the hadronic and %
muonic component of the beam, and only an almost pure neutrino beam pointing towards te.

