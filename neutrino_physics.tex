%%%%%%%%%%%%%%%%%%%%%%%%%%%%%%%		CHAP 1		%%%%%%%%%%%%%%%%%%%%%%%%%%%%%%%

\clearpage
\chapter{Neutrinos in the Standard Model and Beyond}
\label{cha:intro}

The Standard Model (SM) is a renormalisable Yang-Mills theory~\ref{} that describes the strong, electromagnetic, and weak interactions %
of elementary particles in the framework of quantum field theory.
It is based on the local gauge symmetry group 
\begin{equation}
	\label{eq:smgroup}
	\text{SU(3)}_C \otimes \text{SU(2)}_L \otimes \text{U(1)}_Y
\end{equation}
where $C$, $L$ and $Y$ denote respectively colour, left-handed chirality and weak hyper-charge.
The gauge group uniquely determines the interactions and the number of %
vector gauge bosons that correspond to the generators of the group.
%
The electroweak subgroup $\text{SU(2)}_L \otimes \text{U(1)}_Y$ undergoes a spontaneous symmetry breaking process %
out of which three of the four vector bosons acquire mass ($W^\pm$ and $Z$~bosons) and the last one, the photon, remains massless.
The colour section is unbroken and does not mix with the electroweak sector: %
the generators of the algebra of $\text{SU(3)}_C$ corresponds to eight massless gluons.
%
%%In the SM, electroweak interactions can be studied separately from strong interactions, %
%%because the symmetry under the color group is unbroken and there is no mixing %
%%between the SU(3)$_C$ and the $\mathrm{SU(2)}_L \otimes \mathrm{U(1)}_Y$ sectors.
%%On the other hand, the Glashow, Salam, and Weinberg theory well explains the group mixing between %
%%electromagnetic and weak interactions caused by a symmetry breaking process.
%%They are eight massless gluons that mediate strong interactions, %
%%corresponding to the eight generators of SU(3)$_C$, and four gauge bosons, %
%%of which three are massive ($W^\pm$ and $Z$) and one is massless, corresponding %
%%to the three generators of SU(2)$_L$ and one generator of U(1)$_Y$, responsible for %
%%electroweak interactions.
Since the number and properties of the gauge bosons is determined by the SM group, %
the only independent parameters left are the coupling constants of the interactions, which can be determined by the experiments.
The spontaneous breaking symmetry requires at least one scalar boson thanks to the Higgs mechanism.
The recent discovery of the Higgs boson is the crowning achievement of the SM~\ref{}.
%The symmetry group of the SM fixes the interactions, i.e. the number and properties of the %
%vector gauge bosons, with only three independent unknown parameters: the three coupling constants of %
%the SU(3)$_C$, SU(2)$_L$, and U(1)$_Y$ groups, all of which must be determined from experiments.
On the contrary, the number and properties of scalar bosons and fermions are left unconstrained, %
except for the fact that they must transform according to the representations of the symmetry group, %
while the fermion representations must lead to the cancellation of quantum anomalies.

The known elementary fermions are divided in two categories, quarks and
leptons.
They are distinguished by the fact that quarks participate in all the interactions % 
whereas leptons participate only in the electroweak interactions.
\begin{center}
	\small
	\begin{tabular}{lccc}
		\toprule
		\textbf{Generation}	&\textbf{1st}	& \textbf{2nd}	& \textbf{3rd}	\\
		\midrule
		\multirow{2}*{Quark} & $u$ 		& $c$		& $t$		\\
		& $d$		& $s$		& $b$		\\
		\midrule
		\multirow{2}*{Leptons}	& $e$ 		& $\mu$		& $\tau$	\\
		& $\nu_e$	& $\nu_\mu$	& $\nu_\tau$	\\
		\bottomrule
	\end{tabular}
\end{center}

%In the SM, electroweak interactions can be studied separately from strong interactions, %
%because the symmetry under the color group is unbroken and there is no mixing %
%between the SU(3)$_C$ and the $\mathrm{SU(2)}_L \otimes \mathrm{U(1)}_Y$ sectors.
%On the other hand, the Glashow, Salam, and Weinberg theory well explains the group mixing between %
%electromagnetic and weak interactions caused by a symmetry breaking process.
%This model and the discovery of the predicted $W$ and $Z$ bosons, in addition to the gluon, %
%the top, and charm quarks, made the fortune of the Standard Model.
%Their redicted properties were experimentally confirmed with good precision and %
%the recent discovery of the Higgs Boson is the last crowning achievement of SM.

Despite being the most successful theory of particle physics to date, the SM is actually limited %
in its approximation to reality, in that some clear evidences cannot be explained.
The most outstanding breakthrough is the neutrino oscillations, which was awarded the Nobel Prize in Physics in 2015 %
and has proved that the neutrinos are not all massless, as it is assumed by theory.
Mass terms for the neutrinos can be included in the SM, with the implications of theoretical problems.
Likewise, the SM is unable to provide an explanation of the observed asymmetry between matter and anti-matter.
It was noted by Sakharov that a solution to this puzzle would require some form of C and CP violation %
in the early Universe, along with Baryon number violation and out-of-equilibrium interactions.
These facts suggest that the Standard Model is not a complete theory and additional physics %
Beyond the Standard Model (BSM) is required.

The study of neutrinos is for sure one of the most promising probe to BSM physics and %
is of vital importance to the future development of particle physics, %
in particular through precision measurement of their interactions.
A deep understanding of neutrino interactions, and neutrino-nucleon interactions in particular, %
could lead to a great impact on long-baseline experiments, proton decay search, and supernova detection.
Since the SM is a renormalisable theory, even its quantum corrections are insensitive to the physics beyond the SM.
Because of this reason, the SM is phenomenologically very successful and so far has been able to describe all the known
phenomena, except for the indications in favour of neutrino oscillations that we will discuss in the following chapters.

\section{Electroweak sector}
\label{sec:ew_sector}

The electroweak (EW) sector of the SM is formed by the direct product of the weak isospin group $\text{SU(2)}_L$ and %
the hyper-charge group $\text{U(1)}_Y$.
The two groups are connected by the Gell-Mann--Nishijima relation which connects the $I_3$ component of the %
weak isospin operator and the hyper-charge operator $Y$  with the charge operator $Q$ as
\begin{equation}
	\label{eq:gellmann}
	Q = I_3 + \frac{Y}{2}\ .
\end{equation}
Left-handed chiral components of the fermion fields form doublets under $\text{SU(2)}_L$
\begin{equation}
	\label{eq:doublets}
	\bs{L}_L = \mqty(\bs{\nu}_L \\ \bs{\ell}_L) \quad, \quad
	\bs{Q}_L = \mqty(\bs{q}_L^U \\ \bs{q}_L^D) \ ,
\end{equation}
where the left-handed fields in bold-face represent the fermion families
\begin{equation}
	\bs{\nu}_L  = \mqty(\nu_{eL} \\ \nu_{\mu L} \\ \nu_{\tau L}) \quad , \quad
	\bs{\ell}_L = \mqty(e_L \\ \mu_L \\ \tau_L) \quad , \quad
	\bs{q}^D_L  = \mqty(d_L \\ s_L \\ b_L) \quad \text{, and} \quad
	\bs{q}^U_L  = \mqty(u_L \\ c_L \\ t_L)\ .
\end{equation}
%$\bs{q}^U$ and $\bs{q}^D$ represent respectively the up quark and down quark families.
The right-handed fields, instead, transform simply as singlets and they are
\begin{equation}
	\bs{\ell}_R = \mqty(e_R \\ \mu_R \\ \tau_R) \quad , \quad
	\bs{q}^D_R  = \mqty(d_R \\ s_R \\ b_R) \quad \text{, and} \quad
	\bs{q}^U_R  = \mqty(u_R \\ c_R \\ t_R)\ .
\end{equation}
Note that the right-handed components of the neutrino fields, $\nu_{\alpha R}$, are not historically considered in the SM %
because the neutrinos were believed massless until recent times~\ref{}.
As such, neutrinos are assumed to be massless in the SM.
The EW Lagrangian is therefore the most general renormalisable Lagrangian invariant %
under the local symmetry $\text{SU(2)}_L \otimes \text{U(1)}_Y$:
\begin{align}
	\label{eq:ew_lagrangian}
	\mathcal{L}_\text{EW} =\  &i\, \cj{\bs{L}}_L\, \sh{D}\, \bs{L}_L + i\, \cj{\bs{Q}}_L\, \sh{D}\, \bs{Q}_L + %
			 i\, \cj{\bs{\ell}}_R\, \sh{D}\, \bs{\ell}_R + i\, \cj{\bs{q}}^D_R\, \sh{D}\, \bs{q}^D_R + %
			 i\, \cj{\bs{q}}^U_R\, \sh{D}\, \bs{q}^U_R \notag \\ 
			&-\frac{1}{4} B_{\mu\nu} B^{\mu\nu} - \frac{1}{4} \vb{A}_{\mu\nu} \vb{A}^{\mu\nu} 
		      + (D_\mu H)^\dagger (D^\mu H) - \mu^2 H^\dagger H - \lambda (H^\dagger H)^2  \notag \\
		      &- \qty( \cj{\bs{L}}_L\, Y^\ell H \,\bs{\ell}_R 
		      	     + \cj{\bs{Q}}_L\, Y^D    H \,\bs{q}^D_R 
		      	     + \cj{\bs{Q}}_L\, Y^U    \widetilde{H}\, \bs{q}^U_R \ +\ \text{h.c.})\ ,
\end{align}
where the covariant derivative is defined as 
\begin{equation}
	\label{eq:covariant}
	D_\mu = \pd_\mu + i g\, \vb{A}_\mu \cdot \vb{I} + i g'\, B_\mu \frac{Y}{2}\ ,
\end{equation}
and satisfies gauge invariance, %
and $\widetilde{H} = i \sigma_2 H^*$ is the conjugate Higgs field. % thanks to the transformations
It is important to note that Dirac mass terms for fermion fields other than neutrinos %
are anyway forbidden by the gauge symmetry.
These terms will become manifest once the symmetry is broken through the Higgs mechanism (see \refsec{sec:ew_higgs}).
%\begin{align}
%	\label{eq:gauge}
%	\vb{A}_\mu \cdot \vb{I} \longmapsto 
%\end{align}
The vector boson fields $\vb{A}^\mu = (A_1^\mu, A_2^\mu, A_3^\mu)$ and $B^\mu$ corresponds respectively %
to the three generators $\vb{I} = (I_1, I_2, I_3)$ of the $\text{SU(2)}_L$ group %
and the generator $Y$ of the $\text{U(1)}_Y$ group.
The $\text{SU(2)}_L$ generators are $I_a = \sigma_a / 2$, where $\sigma_a$ are the Pauli matrices, %
and thus satisfy the commutation relation
\begin{equation}
	\label{eq:generators}
	[I_a, I_b] = i \varepsilon_{a b c} I_c\ ,
\end{equation}
with $\varepsilon_{a b c}$ the Levi-Civita tensor.

\subsection{Electroweak interactions}
\label{sec:ew_interactions}

Expanding the covariant derivative and ignoring the kinetic terms, we can retrieve the interaction term %
for the leptonic sector
\begin{equation}
	\label{eq:interaction}
	\mathcal{L}_\text{int} = -\frac{1}{2} \sum_\alpha %
		\mqty(\cj{\nu}_{\alpha L} & \cj{\ell}_{\alpha L}) %
		\mqty( g \sh{A}_3 - g' \sh{B} & g(\sh{A}_1 - i \sh{A}_2) \\
		       g(\sh{A}_1 + i \sh{A}_2) & - g \sh{A}_3 - g' \sh{B}  ) %
		\mqty(\nu_{\alpha L} \\ \ell_{\alpha L} ) + g'\, \cj{\ell}_{\alpha R}\, \sh{B} \, \ell_{\alpha R}\ ,
\end{equation}
where $\alpha$ is a family generation index.
Defining the combinations
\begin{align}
	\label{eq:fields}
	W^\mu &= \flatfrac{\qty(A_1^\mu - i A_2^\mu)}{\sqrt{2}} \\
	Z^\mu &= \cos \vartheta_\text{W} A_3^\mu - \cos \vartheta_\text{W} B^\mu \\
	A^\mu &= \sin \vartheta_\text{W} A_3^\mu + \cos \vartheta_\text{W} B^\mu \ ,
\end{align}
the electromagnetic field $A^\mu$ is expressed as a rotation of $A_e^\mu$ and $B^\mu$, thus recovering QED; %
the new field $Z^\mu$ also describes a neutral current process.
The Lagrangian in \refeq{eq:interaction} can therefore be divided into two parts, %
$\mathcal{L}_\text{int} = \mathcal{L}^{(CC)} + \mathcal{L}^{(NC)}$, %
describing charged-current (CC) and neutral-current (NC) interactions.
These are
\begin{equation}
	\label{eq:currents_ccnc}
	\mathcal{L}_\text{CC,L} = -\frac{g}{2\sqrt{2}}\ j^\mu_\text{CC,L} W_\mu + \text{h.c.} 
	\quad \text{and} \quad
	\mathcal{L}_\text{NC,L} = -\frac{g}{2\cos\vartheta_\text{W}}\ j^\mu_\text{NC,L} Z_\mu
		     + g\sin\vartheta_\text{W}\  \cj{\bs{\ell}}\, \sh{A}\, \bs{\ell} \ ,
\end{equation}
where the $W$ and $Z$ vector bosons have been factorised out, leaving the fermionic currents
\begin{align}
	\label{eq:lepton_cc}
	j^\mu_\text{CC,L} &= \cj{\bs{\nu}}\, \gamma^\mu(1-\gamma^5)\, \bs{\ell} \\
	\label{eq:lepton_nc}
	j^\mu_\text{NC,L} &= \cj{\bs{\nu}}\, \gamma^\mu\, (g^\nu_V - g^\nu_A \gamma^5)\, \bs{\nu} + %
		      \cj{\bs{\ell}}\, \gamma^\mu\, (g^\ell_V-g^\ell_A\gamma^5)\, \bs{\ell}\ .
\end{align}
The constant $g'$ has been re-written in terms of $g$ and $\vartheta_\text{W}$ by setting to zero the coupling of neutrinos %
to the electromagnetic field for neutrinos, which gives
\begin{equation}
	g \sin \vartheta_\text{W} = g' \cos \vartheta_\text{W}\ .
\end{equation}
Another important relation comes from the charged lepton couplings with the electromagnetic field which must coincide %
the QED Lagrangian: we have that $g \sin\vartheta_\text{W} = q_e$ and so $g^2 + g'^2 = q_e^2$.
The~constants $g_V$ and $g_A$, introduced in \refeq{eq:lagrangian_nc}, can be defined for any fermion $f$ as
\begin{equation}
	\label{eq:gv_ga}
	g_V^f = I_3^f - 2 q^f \sin^2 \vartheta_\text{W} \quad \text{and} \quad
	g_A^f = I_3^f\ .
\end{equation}
Thanks to this notation, the interaction Lagrangian for the quark sector can be written in the same form of %
\refeq{eq:currents_ccnc}, where the fermionic currents of \refeqs{eq:lepton_cc}{eq:lepton_nc} now become
\begin{align}
	\label{eq:quark_cc}
	j^\mu_\text{CC,Q} &= \cj{\bs{q}}^U\, \gamma^\mu(1-\gamma^5)\, \bs{q}^D \\
	\label{eq:quark_nc}
	j^\mu_\text{NC,Q} &= \cj{\bs{q}}^U\, \gamma^\mu\, (g^U_V - g^U_A \gamma^5)\, \bs{q}^U + %
		      \cj{\bs{q}}^D\, \gamma^\mu\, (g^D_V-g^D_A\gamma^5)\, \bs{q}^D\ .
\end{align}

\subsection{Higgs mechanism}
\label{sec:ew_higgs}

In the EW Lagrangian in \refeq{eq:ew_lagrangian}, the Higgs $H$ is a complex scalar field and an $SU(2)_L$ doublet %
\begin{equation}
	\label{eq:higgs}
	H(x) = \mqty( H^+(x) \\ H^0(x) ) \ ,
\end{equation}
the potential of which can spontaneously break if $\lambda > 0$ and $\mu^2 < 0$, in
\begin{equation}
	\label{eq:higgs_potential}
	V(H) = \mu^2 H^\dagger H + \lambda (H^\dagger H)^2 \ .
\end{equation}
Defining
\begin{equation}
	\label{eq:vev}
	v \equiv \sqrt{- \frac{\mu^2}{\lambda}}\ ,
\end{equation}
the potential $V(H)$ shows a minimum for $H^\dagger H = v^2 / 2$, which %
corresponds to the lowest energy state, or vacuum.
In general, fermion and non-zero spin boson fields must have a vanishing vacuum expectation value (vev), %
so as to to preserve the Lorentz symmetries of space and time.
The same applies to charged scalar fields, as the vacuum is electrically chargeless.
On the other hand, neutral scalar fields can have a nonzero value in vacuum, and so the vev %
of the Higgs field could be given by
\begin{equation}
	\expval{H} = \frac{1}{\sqrt{2}} \mqty( 0 \\ v )\ .
\end{equation}
This value spontaneously breaks the EW group $\text{SU(2)}_L \times \text{U(1)}_Y$, %
but it remains invariant under the gauge transformations from the $\text{U(1)}_Q$ group, %
with $Q$ from \refeq{eq:gellmann}, which guarantees the existence of a massless gauge boson %
associated with the photon.
We can expand the scalar field around its vev and by choosing the unitary gauge three of the four real scalar fields %
can be rotated away being unphysical, simplifying to
\begin{equation}
	\label{eq:vev_higgs}
	\expval{H} = \frac{1}{\sqrt{2}} \mqty( 0 \\ v + h(x) )\ .
\end{equation}
Using the definition of the EW fields in \refeq{eq:fields}, %
the covariant derivative of \refeq{eq:covariant} applied to the Higgs is 
\begin{equation}
	\label{eq:d_higgs}
	D_\mu H(x) = \frac{1}{\sqrt{2}} %
		\mqty ( i \frac{g}{\sqrt{2}} W_\mu (x) \qty[v + h(x)] \\ %
			\pd_\mu h(x) - i \frac{g}{2\cos \vartheta_\text{W}} Z_\mu(x) [v+h(x)] )\ .
\end{equation}
The Lagrangian terms with the Higgs field therefore becomes 
\begin{align}
	\label{eq:all_higgs}
	\mathcal{L}_\text{Higgs} =&\ \frac{1}{2} (\pd h)^2 - v^2 \lambda h^2 - \lambda h^3 - \frac{\lambda}{4} h^4 + %
			\frac{g^2 v^2}{4} W^\dagger_\mu W^\mu + \frac{g^2 v^2}{8\cos^2\vartheta_\text{W}} Z_\mu Z^\mu \notag \\
			& +\frac{g^2 v}{2} W^\dagger_\mu W^\mu h +\frac{g^2 v}{4\cos^2\vartheta_\text{W}} Z_\mu Z^\mu h 
			 +\frac{g^2}{4} W^\dagger_\mu W^\mu h^2 +  \frac{g^2}{8\cos^2\vartheta_\text{W}} Z_\mu Z^\mu h^2\ .
\end{align}
The second term of the first line is a mass term for the scalar fields, %
from which the mass of the Higgs boson is determined to be $m_H = v \sqrt(2\lambda) = \sqrt{-2 \mu^2}$, %
a value recently measured~\ref{}.
The fifth and sixth terms represent the mass terms for the $W$ and $Z$ bosons, namely
\begin{equation}
	m_W = \frac{gv}{2} \quad, \quad m_Z = \frac{gv}{2\cos\vartheta_\text{W}}\ ,
\end{equation}
and the following parameter
\begin{equation}
	\label{eq:magic_ratio}
	\rho = \frac{m_W^2}{m_Z^2 \cos^2\vartheta_\text{W}}
\end{equation}
is predicted to be $\rho = 1$ in the SM: experimentally it is measured to be $\rho = 0.999999$.
The other terms of \ref{eq:all_higgs} describe self-interactions of the Higgs and %
interactions with the $W$ and $Z$ vector bosons.

Applying the same expansion of \refeq{eq:vev_higgs} to the Yukawa terms of the SM Lagrangian, %
we direct coupling between left and right chiral fields and trilinear couplings of the fermions with the Higgs.
We have in the lepton section
\begin{equation}
	\label{eq:lepton_mass}
	\mathcal{L}_{H, L} = - \frac{v}{\sqrt{2}}\ \cj{\bs{\ell}}_L\,Y^\ell\, \bs{\ell}_R %
			     - \frac{1}{\sqrt{2}}\ \cj{\bs{\ell}}_L\,Y^\ell\, \bs{\ell}_R\,h \ + \ \text{h.c.}\ ,
\end{equation}
The same is found in the quark sector:
\begin{equation}
	\label{eq:quark_mass}
	\mathcal{L}_{H, Q} = - \qty(\frac{v}{\sqrt{2}}\ \cj{\bs{q}}_L^D Y^D \bs{q}_R^D %
			          + \frac{v}{\sqrt{2}}\ \cj{\bs{q}}_L^U Y^U \bs{q}_R^U) %
			     - \qty(\frac{1}{\sqrt{2}}\ \cj{\bs{q}}_L^D Y^D \bs{q}_R^D\ h %
			          + \frac{1}{\sqrt{2}}\ \cj{\bs{q}}_L^U Y^U \bs{q}_R^U\ h) \,+\,\text{h.c.}\ .
\end{equation}

The terms in the Lagrangians of \refeqs{eq:lepton_mass}{eq:quark_mass} proportional to $\cj{f}_L f_R = \cj{f} f$ %
are Dirac mass terms for the fermion $f$.
However, there is no principle by which the Yukawa coupling matrices $Y^f$ should be diagonal \emph{a priori}, %
but without a diagonal matrix the fermion masses are not properly defined.
Being a generic complex matrix, the diagonalisation can be performed via a biunitary transformation
\begin{equation} 
	V^{f\dagger}_L\ \qty(\frac{v}{\sqrt{2}} Y^f)\ V^f_R = \frac{v}{\sqrt{2}} \hat{Y}^f_{\alpha} = %
	\text{diag}\qty(\frac{y^f_\alpha v}{\sqrt{2}})\ ,
\end{equation} 
where $V_L$ and $V_R$ are both unitary matrices and the fermion masses are defined by the Yukawa couplings
\begin{equation}
	\label{eq:dirac_mass}
	m^f_\alpha \equiv \frac{y^f_\alpha v}{\sqrt{2}}\ ,
\end{equation}
with $\alpha$ a family generation index.
The biunitary transformation acts on the lepton fields as
\begin{equation}
	\label{eq:lepton_eigenmass}
	\hat{\bs{\ell}}_L = V^{\ell\dagger}_L\, \bs{\ell}_L \quad , \quad \hat{\bs{\ell}}_R = V^{\ell\dagger}_R\, \bs{\ell}_R
\end{equation}
and on the quark fields as 
\begin{align}
	\hat{\bs{q}}^D_L &= V^{D\dagger}_L\, \bs{q}^D_L \quad , \quad \hat{\bs{q}}^D_R = V^{D\dagger}_R\, \bs{q}^D_R \notag \\
	\label{eq:quark_eigenmass}
	\hat{\bs{q}}^U_L &= V^{U\dagger}_L\, \bs{q}^U_L \quad , \quad \hat{\bs{q}}^U_R = V^{U\dagger}_R\, \bs{q}^U_R\ ,
\end{align}
Dropping the ``hat notation'' to indicate mass eigenfields, we obtain in the lepton sector
\begin{equation}
	\label{eq:lepton_diag}
	\mathcal{L}_\text{mass} = -\! \sum_{\alpha=e, \mu, \tau} \frac{y_\alpha^\ell v}{\sqrt{2}}\ \cj{\ell}_\alpha\, \ell_\alpha %
				= -\! \sum_{\alpha=e, \mu, \tau} m_\alpha\ \cj{\ell}_\alpha\, \ell_\alpha \ ,%
\end{equation}
and in the quark sector
\begin{equation}
	\label{eq:quark_diag}
	\mathcal{L}_\text{mass} = -\!\! \sum_{\alpha=d,s,b} \!\qty(\frac{y_\alpha^D v}{\sqrt{2}}\ \cj{q}^D_\alpha\, q^D_\alpha) %
				  -\!\! \sum_{\beta=u,c,t}  \!\qty(\frac{y_\alpha^U v}{\sqrt{2}}\ \cj{q}^U_\beta\,  q^U_\beta) %
				= -\!\! \sum_{\alpha=d,s,b} \!\qty(m_\alpha\ \cj{q}^D_\alpha\, q^D_\alpha) %
				  -\!\! \sum_{\beta=u,c,t}  \!\qty(m_\beta \ \cj{q}^U_\beta\,  q^U_\beta)\ . %
\end{equation}
As noted previously, in the SM there are no mass terms for neutrino fields.

\subsection{Fermion mixing}
\label{sec:fermion_mixing}

The same transformations of \refeqs{eq:lepton_eigenmass}{eq:quark_eigenmass} should be %
equally applied to all the parts of the EW Lagrangian.
Let us start from the quark CC current expressed in \refeq{eq:quark_cc}
\begin{equation}
	\label{eq:real_quark_cc}
	j^\mu_\text{CC,Q} = 2\ \cj{\bs{q}}^U_L\, \gamma^\mu\, \bs{q}^D_L %
			  = 2\ \cj{\hat{\bs{q}}}^U_L V^{U\dagger}_L\, \gamma^\mu\, V_L^D\hat{\bs{q}}^D_L % 
			  = 2\ \cj{\hat{\bs{q}}}^U_L\, \gamma^\mu\, V\, \hat{\bs{q}}^D_L\ ,
\end{equation}
where the unitary matrix $V = V^{U\dagger}_L V^D_L$, called Cabibbo-Kobayashi-Maskawa (CKM) matrix, %
describes the mixing between quark fields in weak interaction processes when initial and final states %
represent particles with definite masses.
The same mixing matrix, however, does not appear in the NC current of \refeq{eq:quark_nc}; %
defining the couplings $2 g_L^f = g_V^f + g_A^f$ and $2 g_R^f = g_V^f - g_A^f$, the current with mass eigenstates %
becomes
\begin{align}
	\label{eq:real_quark_nc}
	j^\mu_\text{NC,Q} &= 2 g^U_L\ \cj{\bs{q}}^U_L\, \gamma^\mu\, \bs{q}^U_L %
			   + 2 g^U_R\ \cj{\bs{q}}^U_R\, \gamma^\mu\, \bs{q}^U_R %
			   + 2 g^D_L\ \cj{\bs{q}}^D_L\, \gamma^\mu\, \bs{q}^D_L %
			   + 2 g^D_R\ \cj{\bs{q}}^D_R\, \gamma^\mu\, \bs{q}^D_R \notag \\
			  &= 2 g^U_L\ \cj{\hat{\bs{q}}}^U_L V^{U\dagger}_L\, \gamma^\mu\, V^U_L \hat{\bs{q}}^U_L %
			   + 2 g^U_R\ \cj{\hat{\bs{q}}}^U_R V^{U\dagger}_R\, \gamma^\mu\, V^U_R \hat{\bs{q}}^U_R %
			   + 2 g^D_L\ \cj{\hat{\bs{q}}}^D_L V^{D\dagger}_L\, \gamma^\mu\, V^D_L \hat{\bs{q}}^D_L %
			   + 2 g^D_R\ \cj{\hat{\bs{q}}}^D_R V^{D\dagger}_R\, \gamma^\mu\, V^D_R \hat{\bs{q}}^D_R \notag \\
			  &= 2 g^U_L\ \cj{\hat{\bs{q}}}^U_L \, \gamma^\mu\, \hat{\bs{q}}^U_L %
			   + 2 g^U_R\ \cj{\hat{\bs{q}}}^U_R \, \gamma^\mu\, \hat{\bs{q}}^U_R %
			   + 2 g^D_L\ \cj{\hat{\bs{q}}}^D_L \, \gamma^\mu\, \hat{\bs{q}}^D_L %
			   + 2 g^D_R\ \cj{\hat{\bs{q}}}^D_R \, \gamma^\mu\, \hat{\bs{q}}^D_R \ .
\end{align}
The neutral current with massive fields has the same form of the neutral current with un-rotated fields.
This is true also for the electromagnetic current of the EW Lagrangian.
This phenomenon is called Glashow-Iliopoulos-Maiani (GIM) mechanism, by which %
flavour-changing neutral currents (FCNCs) are forbidden at tree level thanks to the unitarity of the weak interaction, %
but allowed in suppressed loop diagrams.

Looking at the lepton sector, the transformation of \refeq{eq:lepton_eigenmass} are not analogously defined for neutrino fields.
Therefore, we can arbitrarily choose neutrino states such that $\hat{\bs{\nu}}_L = V^{\ell\dagger}_L\, \bs{\nu}_L \quad$, %
where $V^\ell_L$ is the same of \refeq{eq:lepton_eigenmass}.
The lepton CC current becomes
\begin{equation}
	j^\mu_\text{CC,L} = 2\ \cj{\bs{\nu}}_L\, \gamma^\mu\, \bs{\ell}_L %
			  = 2\ \cj{\hat{\bs{\nu}}}_L V^{\ell\dagger}_L\, \gamma^\mu\, V^\ell_L\hat{\bs{\ell}}_L % 
			  = 2\ \cj{\hat{\bs{\nu}}}_L\, \gamma^\mu\, \bs{\hat{\ell}}_L\ .
\end{equation}
The fields $\hat{\bs{\nu}} = \qty(\hat{\nu}_e, \hat{\nu}_\mu, \hat{\nu}_\tau)$ are called \emph{flavour neutrino fields}, %
because they couple only to the corresponding charged lepton fields in the equation above.
As in the case of the quark fields, thanks to the unitarity of the matrices $V^\ell_L$ and $V^\ell_R$ %
the GIM mechanism applies even for the leptonic neutral current.





\section{Neutrino oscillations}
\label{sec:neutrino_oscillations}

As seen in \refsec{sec:ew_sector}, the SM does not consider right-handed neutrino fields.
For this reason, a Yukawa term coupling the lepton $\text{SU(2)}_L$ doublet with the conjugate Higgs field %
does not appear in the EW Lagrangian.
It follows that after the spontaneous symmetry breaking caused by the Higgs non-zero vev the neutrinos %
do not gain Dirac mass terms, as the other fermions do.
This ``asymmetry'' between fermion fields can be easily resolved by extending the SM, %
introducing right-handed neutrino fields
\begin{equation}
	\bs{\nu}_R = \mqty(\nu_{eR} \\ \nu_{\mu R} \\ \nu_{\tau R})\ .
\end{equation}
With both chiralities, Dirac mass terms can be constructed also for neutrinos, leading however to neutrino mixing %
and so the neutrino oscillation phenomenon, observed in various neutrino experiments.

\subsection{Neutrino mixing}
\label{sec:neutrino_mixing}

Thanks to this extension, the following Yukawa term is now allowed
\begin{equation}
	\mathcal{L}_\text{EW} \supset - \qty( \cj{\bs{L}}_L\, Y^\ell H \,\bs{\ell}_R 
		      	     + \cj{\bs{L}}_L\, Y^\nu \widetilde{H}\, \bs{\nu}_R \ +\ \text{h.c.}) \ ,
\end{equation}
and using the expansion of \refeq{eq:vev_higgs} we have to diagonalise the Yukawa matrix with a biunitary transformation %
to define masses for the neutrino fields.
This lead to new eigenfields
\begin{equation}
	\label{eq:lepton_eigenmass}
	\hat{\bs{\nu}}_L = V^{\nu\dagger}_L\, \bs{\nu}_L = \mqty(\nu_{1L} \\ \nu_{2L} \\ \nu_{3L}) %
	\quad , \quad %
	\hat{\bs{\nu}}_R = V^{\nu\dagger}_R\, \bs{\nu}_R = \mqty(\nu_{1R} \\ \nu_{2R} \\ \nu_{3R})\ ,
\end{equation}
where the fields $\nu_i = \nu_{iL} + \nu_{iR}$ describe Dirac neutrinos with definite masses.
Looking at the lepton sector, the transformation of \refeq{eq:lepton_eigenmass} are not analogously defined for neutrino fields.
Therefore, we can arbitrarily choose neutrino states such that $\hat{\bs{\nu}}_L = V^{\ell\dagger}_L\, \bs{\nu}_L \quad$, %
where $V^\ell_L$ is the same of \refeq{eq:lepton_eigenmass}.
Having now neutrino mass eignestates, the lepton charged current of \refeq{eq:lepton_cc} becomes
\begin{equation}
	j^\mu_\text{CC,L} = 2\ \cj{\bs{\nu}}_L\, \gamma^\mu\, \bs{\ell}_L %
			  = 2\ \cj{\hat{\bs{\nu}}}_L V^{\nu\dagger}_L\, \gamma^\mu\, V^\ell_L\hat{\bs{\ell}}_L % 
			  = 2\ \cj{\hat{\bs{\nu}}}_L\, \gamma^\mu\, U \bs{\hat{\ell}}_L\ ,
\end{equation}
where the unitary matrix $U = V^\nu_L V^\ell_L$ is completely analogous to the CKM matrix of the quark weak charged current.
This matrix is called Pontecorvo-Maki-Nakagawa-Sakata (PMNS) matrix.
Since the flavour of charged lepton is uniquely defined by their masses, it is customary to re-define %
the left-handed flavour neutrino fields as
\begin{equation}
	\bs{\nu}_L = U \hat{\bs{\nu}}_L\ ,
\end{equation}
which allows to write the charged current Lagrangian in terms of flavour neutrinos
begin careful that if neutrino masses are taken into account mixing of the fields occur:
\begin{align}
	\label{eq:lepton_cc_lagrangian}
	\mathcal{L}_\text{CC,L} &= -\frac{g}{\sqrt{2}} \sum_{\alpha=e,\mu,\tau} \cj{\nu}_\alpha\, \sh{W} (1-\gamma^5)\, \ell_\alpha %
					\ +\  \text{h.c.} \notag \\
				&= -\frac{g}{\sqrt{2}} \sum_{\alpha=e,\mu,\tau} \sum_{i=1}^3 U_{\alpha i}^*\, \cj{\nu}_i %
					\,\sh{W} (1-\gamma^5)\, \ell_\alpha \ + \ \text{h.c.}
\end{align}
The GIM mechanism is still valid and no mixing occurs in neutral current interactions.

\subsection{Propagation of neutrinos in vacuum}
\label{sec:neutrino_vacuum}

The effect of neutrino mixing is mostly visible in the propagation of neutrinos in space-time.
Neutrinos with flavour $\alpha$ are produced and detected in charged current interactions in association with %
a charged lepton, according to \refeq{eq:lepton_cc_lagrangian}.
Approximating the neutrino field with plane-waves, the flavour states are described by
\begin{equation}
	\label{eq:neutrino_mix}
	\ket{\nu_\alpha} = \sum_i U^*_{\alpha i} \ket{\nu_i} \ ,
\end{equation}
where the mass eigenstates are orthonormal, $\braket{\nu_i}{\nu_j} = \delta_{ij}$, %
and due to the unitarity of the PMNS matrix, we have $\braket{\nu_\alpha}{\nu_\beta} = \delta_{\alpha \beta}$.
It follows that the probability of producing and detecting a neutrino in the same point of space-time is, unexpectedly, one.
However, in a typical neutrino experiment, production and detection of neutrinos happen in two different locations and times.
The massive neutrino states $\ket{\nu_i}$ are eigenstates of the Hamiltonian, with eigenvalues their energies:
\begin{equation}
	\label{eq:hamilton}
	\mathcal{L} \ket{\nu_i} = E_i \ket{\nu_i} = \sqrt{\vb{p}^2 + m_i^2} \ket{\nu_i}\ ,
\end{equation}
with $\vb{p}$ the momentum of the produced flavour neutrino.
The Hamiltonian dictates the time evolution of the states through the Schrodinger's equation, and %
so assuming neutrinos evolve as plane waves, we have
\begin{equation}
	\label{eq:neutrino_time}
	\ket{\nu_i(t)} = e^{-i E_i t} \ket{\nu_i}\quad \text{and} \quad %
	\ket{\nu_\alpha(t)} = \sum_i U^*_{\alpha i} e^{-i E_i t} \ket{\nu_i}\ .
\end{equation}
Using the relation of \refeq{eq:neutrino_mix}, the pure neutrino flavour state $\ket{\nu_\alpha(t)}$ at $t=0$ %
is expressed as a superposition of flavour states at time $t > 0$, as
\begin{equation}
	\label{eq:neutrino_time_flavour}
	\ket{\nu_\alpha(t)} = \sum_{\beta=e,\mu,\tau} \qty(\sum_i U^*_{\alpha i} e^{-i E_i t} U_{\beta i}) \ket{\nu_\beta}\ ,
\end{equation}
therefore the transition probability from a state $\nu_\alpha$ to a state $\nu_\beta$ over a certain amount of time $t$ is 
\begin{equation}
	\label{eq:oscillation_probability_bad}
	P(\nu_\alpha \to \nu_\beta) \equiv \qty|\braket{\nu_\alpha}{\nu_\beta(t)}|^2 = %
	\sum_{ij} U_{i\alpha}^* U_{\beta i} U_{\alpha j} U_{j\beta}^* %
	e^{-i(E_j - E_i)t}\ .
	%\exp \qty(-i \frac{\Delta m_{ij}^2 L}{2 E})\ ,
\end{equation}
For ultra relativistic neutrino, the energy values can be approximated by
\begin{equation}
	E_i \simeq E + \frac{m_i^2}{2E}\ ,
\end{equation}
whereas the propagation time is comparable to the propagation length, $t \simeq L$ %
as it is easier to determining experimentally.
Adopting these approximations, the probability of \refeq{eq:oscillation_probability_bad} is
\begin{equation}
	\label{eq:oscillation_probability}
	P(\nu_\alpha \to \nu_\beta) \equiv \qty|\braket{\nu_\alpha}{\nu_\beta(t)}|^2 = %
	\sum_{ij} U_{i\alpha}^* U_{\beta i} U_{\alpha j} U_{j\beta}^* %
	\exp \qty(-i \frac{\Delta m_{ij}^2 L}{2 E})\ ,
	%\exp \qty(-i \frac{\Delta m_{ij}^2 L}{2 E})\ ,
\end{equation}
where $\Delta m^2_{ij} = m_i^2 - m_j^2$ are the squared mass differences of the neutrinos.
The probability of \refeq{eq:oscillation_probability} is called \emph{oscillation probability} %
because it shows an oscillatory behaviour with respect to $\flatfrac{L}{E}$, which depend on the experiment, %
while the other parameters---the PMNS matrix elements and the neutrino masses---are constant of nature.
The transition probability for $\alpha = \beta$ is usually called \emph{disappearance probability}, %
and for $\alpha \neq \beta$ is called \emph{appearance probability}, because typically experiments %
measure the amount of neutrino of a certain flavour

The oscillating term is the result of the interference of different massive neutrinos, %
which propagate at different velocities, but coherency between states is preserved.
It could happen that neutrinos are produced or detected incoherently, for which interfere terms %
do not appear, or that the resolution on the propagation length or the energy is limited, %
for which the probability must be average.
In both cases, the oscillation probability simplifies to
\begin{equation}
	\label{eq:average_oscillation}
	\langle P(\nu_\alpha \to \nu_\beta)\rangle = \sum_{i} |U_{\alpha i}|^2 |U_{\beta i}|^2\ ,
\end{equation}
and for $\alpha \neq \beta$ it can be shown that the maximum value the averaged probability %
can take is 
\begin{equation}
	\langle P(\nu_\alpha \to \nu_\beta)\rangle_\text{max} = \frac{1}{N}\ ,
\end{equation}
with $N$ the number of massive neutrinos.
Under these circumstances, the elements of the mixing matrix have all the same absolute, called \emph{N-maximal mixing}.
This corresponds to minimal average disappearance probability and maximal %
average appearance probability, equal to 1/N in each possible channel.

Apart from this limit scenario, the PMNS matrix (and the CKM matrix) are generic unitary $N \times N$ matrices, %
where $N$ is the number of fermion generations.
A matrix with this characteristics depends on $N^2$ independent real parameters, divided among 
\begin{equation}
	\frac{N(N-1)}{2} \  \text{angles}  \quad \text{and} \quad
	\frac{N(N+1)}{2} \  \text{phases}\ .
\end{equation}
Due to the unitarity of the neutral currents, only the weak charged current can manifest these phases, %
even though not all of them are observable: $2N-1$ phases can be reabsorbed by a redefinition of the fermion fields.
The physical phases are therefore just
\begin{equation}
	\frac{(N-1)(N-2)}{2}\ ,
\end{equation}
meaning that with three generations the PMNS matrix (and the CKM matrix) can be described by three mixing angles %
and one physical phase.
The PMNS matrix is typically parameterised as follows
%\vspace{-0.2em}
\begin{equation}
	\label{eq:pmns}
	U = \mqty( 1 & 0 & 0 \\ 0 & c_{23} & s_{23} \\ 0 & -s_{23} & c_{23} ) %
	\mqty( c_{13} & 0 & s_{13}e^{-i\delta_\text{CP}} \\ 0 & 1 & 0 \\ -s_{13}e^{i\delta_\text{CP}} & 0 & c_{13} ) %
	\mqty( c_{12} & s_{12} & 0 \\ -s_{12} & c_{12} & 0 \\ 0 & 0 & 1 )\ , %
	%\mqty( 1 & 0 & 0 \\ 0 & e^{i\gamma_1} & 0 \\ 0 & 0 & e^{i\gamma_2} ) %
%\vspace{-0.2em}
\end{equation}
where $c_{ij} \equiv \cos\theta_{ij}$ and $s_{ij} \equiv \sin\theta_{ij}$.
The angle $\delta_\text{CP}$ is the physical phase which is responsible for CP violation (see \refsec{sec:cp_oscillation}).
%whereas the additional phases $\gamma_1$ and $\gamma_2$ bring about CP violation if the neutrinos are Majorana.

\subsection{Propagation of neutrinos in matter}
\label{sec:neutrino_matter}

Neutrinos propagating in a dense medium can interact with the particles in the medium.
The probability of an incoherent inelastic scattering is very small.
For example the characteristic cross section for $\nu$-proton scattering is of the order
\begin{equation}
	\sigma \simeq \frac{G_F^2 s}{\pi} \sim \np{e-43} \text{cm}^2 \qty(\frac{E}{\text{MeV}})^2
\end{equation}
where s is the square of the center of mass energy of the collision.
When neutrinos propagate in dense matter, they can also interact coherently with the particles in the medium.
By definition, in coherent interactions, the medium remains unchanged so it is possible to have interference %
of the forward scattered and the unscattered neutrino waves which enhances the effect of matter in the neutrino propagation.
In this case the effect of the medium is not on the intensity of the propagating neutrino beam, which remains unchanged, but on the phase
velocity of the neutrino wave, and for this reason the effect is proportional to $G_F$, %
instead of the $G_F^2$ dependence of the incoherent scattering.
Coherence also allows decoupling the evolution equation of the neutrinos from those of the medium.
In this limit the effect of the medium is introduced in the evolution equation for the neutrinos %
in the form of an effective potential which depends on the density and composition of the matter.

In general, the discussion of oscillations in matter for three neutrino mixing is quite complicated, %
because of the possible interplay of the effects of the different mass squared differences.
Fortunately, in the case of the hierarchy of squared-mass differences, for which
\begin{equation}
	\label{eq:hierarchy}
	\Delta m^2_\text{sol} = \Delta m^2_{21} \ll \Delta m^2_\text{atm} = |\Delta m^2_{31}|\ ,
\end{equation}
the effects of the large and small squared-mass differences can be separated out, %
leading to a considerable simplification of the discussion and a clear understanding of the %
physical mechanisms at work in experiments.


\section{Neutrino production}
\label{sec:prod}

Neutrinos are the most abundant fermion in the Universe.
With the expansion and cooling of the Universe, the rate of weak interaction primordial neutrinos %
decreased below the expansion rate, resulting in a decoupling of the neutrinos from the other particles.
These \emph{relic netruinos} form what is called the Cosmic Neutrino Background (C$\nu$B), %
a radiation analogous to the Cosmic Microwave Background (CMB) which decoupled at early stages of the Universe.
The C$\nu$B formed well before the CMB, due to the weak nature of neutrino interactions, %
and it has a temperature given by the famous relation
\begin{equation}
	\label{eq:nu_temperature}
	T_\nu = \qty(\frac{4}{11})^{\flatfrac{1}{3}} T_\gamma = (\np{1.945} \pm \np{0.001})\,\text{K}\ .
\end{equation}
This relation makes a precise prediction of the temperature of the neutrinos $T_\nu$, %
connecting it with the photon temperature $T_\gamma$, which is accurately measured by CMB surveys.
The temperature is equivalent to $T_\nu = (\np{1.676} \pm \np{0.001}) \times 10^{-4}$\,eV and %
so at least two mass states of relic neutrinos are non-relativistic, comparing this energy to the measured mass differences %
from oscillation experiments.
Neutrinos at the these energies are almost impossible to detect with the current technologies.
The density of relativistic neutrinos can be related to the density of photons, thanks to \refeq{eq:nu_temperature}, %
and 
\begin{equation}
	\label{eq:nu_gamma_density}
	\frac{\rho_\nu}{\rho_\gamma} = \frac{7}{8} \qty(\frac{4}{11})^{\flatfrac{1}{3}} N_\text{eff}\ ,
\end{equation}
by which the energy density with respect to the critical density is
\begin{equation}
	\label{eq:rel_density}
	\Omega_{\nu_\text{R}}\, h^2 = \qty(\frac{4}{11})^{\flatfrac{1}{3}} \Omega_\gamma\, h^2\ ,
\end{equation}
with $h$ the Hubble constant.
When the neutrinos are non-relativistic, their energy density is given by $\rho_\nu \simeq \sum_i m_i n_i$, %
where $n_i$ are the number density of each species which are equal to each other, %
up to negligible corrections from flavour effect.
Using the expression of \refeq{eq:nu_gamma_density} and knowing the number density of photons $n_\gamma$, %
the total energy density of non-relativistic neutrinos today is
\begin{equation}
	\label{eq:nonrel_density}
	\Omega_{\nu_\text{NR}}\, h^2 = \frac{\sum_i m_i}{\np{94.14}\,\text{eV}}\ ,
\end{equation}
and the contribution of relativistic neutrinos to the total mass is also negligible.
Due to the fact that thes density of \refeq{eq:nonrel_density} can never be greater than %
the energy density of all matter in the Universe, a naive bound on the neutrino masses can be derived:
\begin{equation}
	\sum_i m_i \lesssim 13\,\text{eV}\ .
\end{equation}
The limit can be improved with more precise theoretical considerations under $\Lambda$CDM assumption, %
which combined with the latest cosmological surveys goes down to $\sum_i m_i \lesssim 0.12$~\cite{1306.5544}.

Other than cosmological neutrinos, these elusive fermions are produced and involved in a large variety of processes.
They are emitted in astrophysical processes, such as supernova expolosions, blazars, and in the nuclear reactions %
of the cores of stars.
Cosmic ray interactions with the Earth's atmosphere also produces neutrino from decays of secondary mesons.
Human-made sources include accelerator facilities in which neutrino beams are produced in a controlled environment, %
as well as those emitted by nuclear reactors.
Neutrinos are also the products of $\beta$-decays, from the study of which neutrinos were first hypothesised. 
The study of the rare double-$\beta$ decay is of utter importance, % 
because the neutrinoless manifestition of this decay is a true lepton number violating process, %
the detection of which has deep and considerable connotation for neutrino mass theories.

In the following sections we discuss the most relevant sources relevant for the neutrino experiment %
mentioned in this thesis.

\subsection{Solar and supernova neutrinos}
\label{sec:nu_sun_sn}

Neutrinos produced in the core of the Sun were the first astrophysical source of neutrinos detected.
They are yielded in thermonuclear reaction in the solar core.
Being a G-type star in the main sequence, the Sun's energy is powered mostly from proton--proton ($pp$) chain %
and partly by the carbon--nitrogen--oxygen (CNO) cycle.
In both these processes, the net result is the conversion of four protons and two electrons into an $\alpha$-particle, %
of \tapi{4}He nucles, and two neutrinos
\begin{equation}
	\label{eq:sun_net}
	4\,p\ +\ 2\,e^- \longrightarrow {}^4\text{He}\ +\ 2\, \nu_e\ ,
\end{equation}
along with the release of $\np{26.731}$\,MeV in the form of photons or kinetic energy of the neutrinos.

When a massive star at the end of its life collapses to a neutron star, %
it radiates almost all of its binding energy in the form of neutrinos, %
most of which have energies in the range 10 to 30\,MeV, and are emitted over a timescale of several tens of seconds.
These neutrinos are released just after core collapse, whereas the photon signal may take hours %
or days to emerge from the stellar envelope.


\subsection{Atmospheric neutrinos}

\subsection{Reactor and accelerator neutrinos}

Neutrino beams are derived from the decays of charged $\pi$ and $K$ mesons, which in turn are created from %
proton beams striking thick nuclear targets.
The precise selection and manipulation of the $\pi/K$ beam control the energy spectrum and type of neutrino beam.

The $\pi^{\pm}$ mesons have a mass of \np{139.6}~MeV and a mean lifetime of \np{2.6e-8}~s.
The primary decay mode of a pion, with a branching fraction of \np{99,9877}\,\%, is a leptonic %
decay into a muon and a muon neutrino:

%%
\begin{minipage}[c][3cm][c]{0.5\textwidth}
	\centering
	\begin{align}
		\pi^+ &\rightarrow \mu^+ + \nu_\mu \\
		\pi^- &\rightarrow \mu^- + \bar{\nu}_\mu
	\end{align}
\end{minipage}
%
\begin{minipage}[c][3cm][c]{0.5\textwidth}
	\centering
	\begin{fmffile}{pion_muon}
		\begin{fmfgraph*}(80,50)
			\fmfleft{i2,i1}
			\fmfright{o2,o1}
			\fmf{fermion}{i1,v1,i2}
			\fmf{photon}{v1,v2}
			\fmf{fermion}{o1,v2,o2}
			\fmf{photon, label=$W^\pm$}{v1,v2}
			\fmflabel{$\pi$}{v1}
			\fmflabel{$u,d$}{i1}
			\fmflabel{$\bar{d},\bar{u}$}{i2}
			\fmflabel{$\mu^\pm$}{o1}
			\fmflabel{$\nu_\mu,\bar{\nu}_\mu$}{o2}
		\end{fmfgraph*}
	\end{fmffile}
\end{minipage}
%%

The second most common decay mode of a pion, is the leptonic decay into an electron and the %
corresponding neutrino, $\pi^\pm \rightarrow \nu_e + e$.
In spite of the considerable differences in the space momentum, this process is suppressed %
with respect to the muonic one.
This effect is called \emph{helicity suppression} and is due to the great mass of the muon %
($m_\mu = \np{105.658}$~MeV) compared to the electron's ($m_e = \np{0.510}$~MeV); this results in a %
stronger helicity-chirality correspondence for the electron rather than for the muon.
Given that the $\pi$ mesons are spinless, neutrinos are left-handed, and antineutrinos are %
right-handed, the muonic channel is preferred because of spin and linear momentum preservation.
The suppression of the electronic decay mode with respect to the muonic one is given %
approximately within radiative corrections by the ratio:
\begin{equation}
	R_\pi = \Bigl ( \frac{m_e}{m_\mu} \Bigr )^2 
	\Bigl (\frac{m_\pi^2 - m_e^2}{m_\pi^2 - m_\mu^2} \Bigr )
	= \np{1.283e-4}
\end{equation}
The measured branching ratio of the electronic decay is indeed $(\np{1.23}\pm{0.02})\times10^{-4}$.

As far as the charged $K$ meson is concerned, it mainly decays in a muon and its corresponding neutrino, %
with a branching ratio of \np{63.55}\,\%:

%%
\begin{minipage}[c][3cm][c]{0.5\textwidth}
	\centering
	\begin{align}
		K^+ &\rightarrow \mu^+ + \nu_\mu \\
		K^- &\rightarrow \mu^- + \bar{\nu}_\mu
	\end{align}
\end{minipage}
%
\begin{minipage}[c][3cm][c]{0.5\textwidth}
	\centering
	\begin{fmffile}{kaon_muon}
		\begin{fmfgraph*}(80,50)
			\fmfleft{i2,i1}
			\fmfright{o2,o1}
			\fmf{fermion}{i1,v1,i2}
			\fmf{photon, label=$W^\pm$}{v1,v2}
			\fmf{fermion}{o1,v2,o2}
			\fmflabel{$K$}{v1}
			\fmflabel{$u,s$}{i1}
			\fmflabel{$\bar{s},\bar{u}$}{i2}
			\fmflabel{$\mu^\pm$}{o1}
			\fmflabel{$\nu_\mu,\bar{\nu}_\mu$}{o2}
		\end{fmfgraph*}
	\end{fmffile}
\end{minipage}
%%

The second most frequent decay (\np{20.66}\,\%) is the decay into two pions, $K^{\pm} \rightarrow \pi^0 + \pi^\pm$.
Other decays have a branching ratio of 5\,\% or less and are listed in table Tab.~\ref{tab:kaons}.
On the contrary, the decays of the neutral kaon produce neutrino in few cases.
Because of the oscillation phenomenon given by the mixing between $K^0$ and $\bar K^0$, the neutral kaon has two %
manifestations, the short kaon $K_S$ and the long kaon $K_L$, named after their lifetimes.
While the $K$-short decays only in two pions ($2 \pi^0$ or $\pi^+ + \pi^-$), the $K$-long has a wider variety %
of final state combination, all of them involving three particles.
Among these, neutrinos are produced in the processes:

%%
\begin{minipage}[c][3cm][c]{0.5\textwidth}
	\centering
	\begin{align}
		K^0_L &\rightarrow \pi^\pm + \mu^\mp + \overset{(-)}{\nu}_\mu \\
		K^0_L &\rightarrow \pi^\pm + e^\mp + \overset{(-)}{\nu}_e 
	\end{align}
\end{minipage}
%
\begin{minipage}[c][3cm][c]{0.5\textwidth}
	\centering
	\begin{fmffile}{kaonlong}
		\begin{fmfgraph*}(70,70)
			\fmfleft{h,i}
			\fmfright{k,o3,o2,o1}
			\fmf{fermion}{i,v1,o3}
			\fmf{fermion}{h,w,k}
			\fmf{photon}{v1,v2}
			\fmf{fermion}{o1,v2,o2}
			\fmflabel{$d$}{h}
			\fmflabel{$d$}{k}
			\fmflabel{$s$}{i}
			\fmflabel{$\nu_\mu$}{o3}
			\fmflabel{$e^\pm$}{o2}
			\fmflabel{$\nu_e$}{o1}
		\end{fmfgraph*}
	\end{fmffile}
\end{minipage}
%%

\begin{table}
	\caption{Decay mode for a charged kaon, $K^\pm$, sorted by branching ration (in percent).}
	\label{tab:kaons}
	\[
		\begin{array}{lr}
			\toprule
			\mu^\pm + \overset{(-)}{\nu}_\mu	&	\np{65.55}\pm\np{0.11}	\\
			\midrule
			\pi^\pm + \pi^0			&	\np{20.66}\pm\np{0.08}	\\
			\midrule
			\pi^+ + \pi^\pm + \pi^-		&	\np{5.59}\pm\np{0.04}	\\
			\midrule
			\pi^0 + e^\pm + \overset{(-)}{\nu}_e	&	\np{5.07}\pm\np{0.04}	\\
			\midrule
			\pi^0 + \mu^\pm + \overset{(-)}{\nu}_\mu	&	\np{3.35}\pm\np{0.03}	\\
			\midrule
			\pi^\pm + \pi^0 + \pi^0		&	\np{1.76}\pm\np{0.02}	\\
			\bottomrule
		\end{array}
	\]
\end{table}

Neutrinos are also produced by the decay of muons.
Muons are unstable elementary particles and decay via the weak interaction. 
The dominant decay mode, called \emph{Michel decay}, is also the simplest possible:
because lepton numbers must be conserved, one of the product neutrinos of muon decay %
must be a muonic neutrino and the other an electronic antineutrino, along with an electron, %
because of the charge preservation.
Vice versa, an antimuon decay produces the corresponding antiparticles.
These two decays are:

%%
\begin{minipage}[c][3cm][c]{0.5\textwidth}
	\centering
	\begin{align}
		\label{eq:mupdecay}
		\mu^- &\rightarrow e^- + \bar\nu_e + \nu_\mu \\
		\label{eq:mundecay}
		\mu^+ &\rightarrow e^+ + \nu_e + \bar\nu_\mu
	\end{align}
\end{minipage}
%
\begin{minipage}[c][3.5cm][c]{0.5\textwidth}
	\centering
	\begin{fmffile}{mudecay}
		\begin{fmfgraph*}(70,70)
			\fmfleft{i}
			\fmfright{o3,o2,o1}
			\fmf{fermion}{i,v1,o3}
			\fmf{photon}{v1,v2}
			\fmf{fermion}{o1,v2,o2}
			\fmflabel{$\mu^\pm$}{i}
			\fmflabel{$\nu_\mu$}{o3}
			\fmflabel{$e^\pm$}{o2}
			\fmflabel{$\nu_e$}{o1}
		\end{fmfgraph*}
	\end{fmffile}
\end{minipage}
%%





\section{Neutrino interactions}
\label{sec:neutrino_interactions}

Neutrino interactions in their flavour states are described by \refeqs{eq:lepton_cc}{eq:lepton_nc}.
Replacing the values of $g^\nu_V$ and $g^\nu_A$ for a neutrino, the relevant lagrangian terms are
\begin{align}
	\mathcal{L}_{\text{CC},\nu} &= -\frac{g}{2\sqrt{2}}\ 	      %
	\sum_{\alpha=e,\mu,\tau} \cj{\nu}_\alpha \sh{W} (1-\gamma^5) \ell_\alpha + \text{h.c.} \\
	\mathcal{L}_{\text{NC},\nu} &= -\frac{g}{4\cos\vartheta_\text{W}}\ %
	\sum_{\alpha=e,\mu,\tau} \cj{\nu}_\alpha \sh{Z} (1-\gamma^5) \nu_\alpha \ .
\end{align}
For energies below the $W$ and $Z$ mass, the allowed tree-level interactions involving at least one neutrino %
are any allowed variation of the processes represented in \reffig{fig:neutrino_tree}.
The vector bosons cannot be produced on-shell and so their field operator must contract with some other external field.
The amplitudes of the processes shown in \reffig{fig:neutrino_tree} are
\begin{align}
	i \mathcal{M}_\text{CC} &= i\, \frac{g^2}{8}\ \cj{u}_{\ell_\alpha} \gamma^\mu (1-\gamma^5)\, u_{\nu_\alpha}
						    \,\frac{\eta_{\mu\nu}}{k^2 - m_W^2+i\varepsilon}
						    \,\cj{u}_{f_2} \gamma^\nu (1-\gamma^5)\,V_{12}\, u_{f_1} \\
	i \mathcal{M}_\text{NC} &= i\, \frac{g^2}{8\cos\vartheta}\ \cj{u}_{\nu_\alpha} \gamma^\mu (1-\gamma^5)\, u_{\nu_\alpha}
						    \,\frac{\eta_{\mu\nu}}{k^2 - m_Z^2+i\varepsilon}
						    \,\cj{u}_{f} \gamma^\nu (g^f_V-g^f_A\gamma^5) u_{f}\ ,
\end{align}
where $k$ is the momentum of the propagator and $V_{12}$ a possible mixing between generic fermions $f_1$ and $f_2$.
Because of the tyipical energies involved, the momentum propagating between the particles is small %
compared to the masses of the intermediate bosons and therefore this mass is typically factorised out, %
resulting in
\begin{align}
	i \mathcal{M}_\text{CC} &\simeq i\,  \frac{G_F^2}{\sqrt{2}}\ \cj{u}_{\ell_\alpha} \gamma^\mu (1-\gamma^5)\, u_{\nu_\alpha}
						    \,\cj{u}_{f_2} \gamma^\mu (1-\gamma^5)\,V_{12}\, u_{f_1} \\
	i \mathcal{M}_\text{NC} &\simeq i\,  \frac{G_F^2}{\sqrt{2}}\ \cj{u}_{\nu_\alpha} \gamma^\mu (1-\gamma^5)\, u_{\nu_\alpha}
						    \,\cj{u}_{f} \gamma^\mu (g^f_V-g^f_A\gamma^5) u_{f}\ ,
\end{align}
where the relation in \refeq{eq:magic_ratio} was used and a new constant was introduced 
\begin{equation}
	\label{eq:fermi_const}
	\frac{G_F}{\sqrt{2}} = \frac{g^2}{8\ m_W^2}\ .
\end{equation}
The constant in \refeq{eq:fermi_const} is called Fermi's constant and the interactions %
can be described by effective four-point Lagrangians.
%\begin{align}
%	\mathcal{L}_\text{eff, CC} &= - \frac{G_F}{\sqrt{2}} \qty[\cj{\ell}_\alpha \gamma^\mu(1-\gamma^5)\nu_\alpha] %
%							     \qty[\cj{f}_2\gamma_\mu(1 - \gamma^5) f_1] \\
%	\mathcal{L}_\text{eff, CC} &= - \frac{G_F}{\sqrt{2}} \qty[\cj{\nu}_\alpha \gamma^\mu(1-\gamma^5)\nu_\alpha] %
%							     \qty[\cj{f}\gamma_\mu(g^F_V - g^F_A \gamma^5) f]
%\end{align}
						    
\begin{figure}
	\centering
	\medskip
	\begin{fmffile}{neutrino_CC}
		\begin{fmfgraph*}(80,60)
			\fmfleft{f1,nu}
			\fmfright{f2,ell}
			\fmf{fermion}{nu,v1,ell}
			\fmf{fermion}{f1,v2,f2}
			\fmf{photon, label=$W$}{v1,v2}
			\fmflabel{$f_1$}{f1}
			\fmflabel{$f_2$}{f2}
			\fmflabel{$\nu_\alpha$}{nu}
			\fmflabel{$\ell_\alpha$}{ell}
		\end{fmfgraph*}
	\end{fmffile}
	\qquad
	\raisebox{2.5em}{,}
	\qquad
	\begin{fmffile}{neutrino_NC}
		\begin{fmfgraph*}(80,60)
			\fmfleft{f1,nu1}
			\fmfright{f2,nu2}
			\fmf{fermion}{nu1,v1,nu2}
			\fmf{fermion}{f1,v2,f2}
			\fmf{photon, label=$Z$}{v1,v2}
			\fmflabel{$f$}{f1}
			\fmflabel{$f$}{f2}
			\fmflabel{$\nu_\alpha$}{nu1}
			\fmflabel{$\nu_\alpha$}{nu2}
		\end{fmfgraph*}
	\end{fmffile}
	\bigskip
	\caption{Generic CC (right) and NC (left) tree-level interactions with neutrinos involved.
		Note that these diagrams have illustration purpose only, and that's why time flow convention is not respected. }
	\label{fig:neutrino_tree}
\end{figure}

In the following sections we will review the most important neutrino interactions with matter, %
from low to higher energies

\subsection{Coherent elastic neutrino--nucleus scattering}

\subsection{Neutrino--electron scattering}

The easiest interaction to study between neutrinos and matter components at low energies %
is the neutrino--electron elastic scattering
\begin{equation}
	\overset{(-)}{\nu}_\alpha + e^- \rightarrow \overset{(-)}{\nu}_\alpha + e^-\ .
\end{equation}
For the electron neutrino $\nu_e$, both CC and NC interactions are allowed and %
for the electron antineutrino $\cj{\nu}_e$ the $s$-channel and $t$-channel diagrams are swapped,
whereas for $\alpha = \mu, \tau$ only neutral-current interactions exist.
%The respective Feynman diagrams are shown in Fig.~\ref{fig:nescat} and~\ref{nutscat}.
%The effective lagrangian 
%\begin{align}
%	\mathcal{L}_\mathrm{eff}(\nu_e e^- \rightarrow \nu_e e^-) &= - \frac{G_F}{\sqrt{2}} %
%	[\overline{\nu_e}\gamma^\mu(1-\gamma^5)\nu_e][\bar{e}\gamma_\mu((1+g_V^l)-(1+g_A^l)\gamma^5)e] \\
%	\mathcal{L}_\mathrm{eff}(\nu_\alpha e^- \rightarrow \nu_\alpha e^-) &= - \frac{G_F}{\sqrt{2}} %
%	[\overline{\nu_\alpha}\gamma^\mu(1-\gamma^5)\nu_\alpha][\bar{e}\gamma_\mu(g_V^l-g_A^l)\gamma^5)e] %
%	\quad (\alpha = \mu,\tau)\,.
%\end{align}
Using the effective four-point Lagrangians, one can calculate the differential cross-sections in the laboratory frame %
with an initial electron at rest:
\begin{equation}
	\label{eq:nu_elastic_xsec}
	\dv{\sigma}{q^2} = \frac{G_F^2}{\pi} \qty[\kappa_1^2 + \kappa_2^2 
		\qty(1 - \frac{q^2}{2\,(p_\nu \cdot p_e)} )^2 - \kappa_1\, \kappa_2\, m_e^2\, \frac{q^2}{2\, (p_\nu \cdot p_e)^2} ] \ ,
\end{equation}
where $-q^2 = t$ and the quantities $\kappa_1$ and $\kappa_2$ %
depend on the neutrino flavour and embeds CC and NC contributions.
Using the vector and axial coupling of \refeq{eq:gv_ga} they read
\begin{align*}
	&\kappa_1^{\nu_e} = \kappa_2^{\cj{\nu}_e} = 1 + \frac{g_V^\ell + g_A^\ell}{2} = \frac{1}{2} + \sin^2\vartheta_W \\
	&\kappa_2^{\nu_e} = \kappa_1^{\cj{\nu}_e} = \frac{g_V^\ell - g_A^\ell}{2} = \sin^2\vartheta_W \\
	&\kappa_1^{\nu_{\mu,\tau}} = \kappa_2^{\cj{\nu}_{\mu,\tau}} = \frac{g_V^\ell + g_A^\ell}{2} = -\frac{1}{2} + \sin^2\vartheta_W \\
	&\kappa_2^{\nu_{\mu,\tau}} = \kappa_1^{\cj{\nu}_{\mu,\tau}} = \frac{g_V^\ell - g_A^\ell}{2} = \sin^2\vartheta_W \ .
\end{align*}

The variable $t$ is the squared difference between the four-momentum of the initial and final electrons, 
and denoting $T_e = E_e - m_e$ as the kinetic energy of the outgoing electron, we see that
\begin{equation}
	-t = q^2 = 2\,m_e\,T_e\ .
\end{equation}
With some simple kinematics the kinetic energy is found to be
\begin{equation}
	T_e = \frac{2\,m_e\,E_\nu^2 \cos^2 \theta}{(m_e + E_\nu)^2 - E_\nu^2 \cos^2\theta}\ ,
\end{equation}
and so the differential cross-section in \refeq{eq:nu_elastic_xsec} can be given as a function of the %
electron scattering angle $\theta$ with respect to the incoming neutrino with energy $E_\nu$:
\begin{align}
	\dv{\sigma}{\cos\theta} =&\ \frac{2 G_F^2 m_e}{\pi} %
			\frac{4 E_\nu^2 (m_e+E_\nu)^2 \cos \theta}{\qty[(m_e+E_\nu)^2-E_\nu^2 \cos^2 \theta ]^2} \quad \times \notag \\
			&\qty[g_1^2 + g_2^2\qty(1 - \frac{2 m_e E_\nu \cos^2 \theta}{(m_e+E_\nu)^2-E_\nu^2 \cos^2 \theta} )^2
		- g_1\, g_2\, \frac{2m_e^2 \cos^2 \theta}{(m_e+E_\nu)^2-E_\nu^2 \cos^2 \theta}]\ .
\end{align}

\subsection{Neutrino scattering with nucleons}

Neutrinos can also interact with hadrons in matter, namely proton and neutrons, %
as the quark components of these baryons couple to the $W$ and $Z$ vector bosons according to %
the currents of \refeqs{eq:real_quark_cc}{eq:real_quark_nc}.
The compoEven though
In general, these processes can be categorised according to the momentum transfer.
At small $q^2$, elastic interactions dominate and may be brought about by both charged and neutral currents.
When this occurs via neutral currents, all flavour of neutrinos and anti-neutrinos can scatter off %
both neutrons and protons in what is referred to as ``NC elastic'' scattering.
The process is:
\begin{align}
	\nu_\alpha + N \rightarrow \nu_\alpha + N \\
	\bar\nu_l + N \rightarrow \bar\nu_l + N\ ,
\end{align}

Once neutrinos acquire sufficient energy they can also undergo the analogous charged current interactions, %
called ``quasi-elastic'', due to the fact that the recoiling nucleon changes its charge and mass transfer occurs.
The processes are
\begin{align}
	\nu_l + n &\rightarrow p + l^- \\
	\bar\nu_l + p &\rightarrow n + l^+ \ ,
\end{align}
with $l=e, \mu, \tau$.
For the muonic neutrino with energy below one GeV, the CCQE is the dominant interaction, event though the %
cross-section plateaus at higher energies, as the available $Q^2$ increases: it becomes increasingly unlikely %
for the nucleon to remain intact.

The physics behind the CC quasi-elastic processes is more complicated.
The differential cross-section for the scattering in the laboratory frame is given by
\begin{equation}
	\label{eq:cc_xsec_q}
	\frac{\mathrm{d} \sigma_{CC}}{\mathrm{d}Q^2} = \frac{G_F^2 |V_{ud}|^2 m_N^4}{8\pi (p_\nu \cdot p_N)^2} %
	\bigg [A(Q^2) \pm B(Q^2) \frac{s-u}{m_N^2} + C(Q^2) \frac{(s-u)^2}{m_N^4} \bigg]\,,
\end{equation}
where the plus sign refers to the $N = n$ interactions, while the minus sign to $N = p$.

\begin{equation}
	\label{eq:cc_xsec_t}
	\frac{\mathrm{d} \sigma_{CC}}{\mathrm{d}\cos\theta} = -\frac{G_F^2 |V_{ud}|^2 m_N^2}{4\pi} \frac{p_l}{E_\nu} %
	\bigg [A(Q^2) \pm B(Q^2) \frac{s-u}{m_N^2} + C(Q^2) \frac{(s-u)^2}{m_N^4} \bigg]\,,
\end{equation}

The functions $A(Q^2)$, $B(Q^2)$, and $C(Q^2)$ depends on the nucleon form-factors in the following way:
\begin{align}
	\begin{split}
		\label{eq:A(Q)}
		A &= \frac{m_l^2+Q^2}{m_N^2} \bigg\{ \bigg(1+\frac{Q^2}{4m_N^2}\bigg) G_A^2 - \bigg(1-\frac{Q^2}{4m_N^2}\bigg) %
		\bigg(F_1^2 - \frac{Q^2}{4m_N^2}F_2^2 \bigg) +\frac{Q^2}{m_N^2} F_1 F_2 \\
		&\qquad- \frac{m_l^2}{4m_N^2} %
		\bigg[ (F_1+F_2)^2+(G_A+2G_P)^2-\frac{1}{4}\bigg(1+\frac{Q^2}{4m_N^2}\bigg) G_P^2 \bigg] \bigg\}\, 
	\end{split}\\
	\label{eq:B(Q)}
	B &= \frac{Q^2}{m_N^2} G_A (F_1+F_2)\,\\
	\label{eq:C(Q)}
	C &= \frac{1}{4} \big (G_A^2 +F_1^2+\frac{Q^2}{4m_N^2}F_2^2\big)\,.
\end{align}

The form factors $F_1(Q^2)$, $F_2(Q^2)$, $G_A(Q^2)$, and $G_P(Q^2)$ are called, respectively, \emph{Dirac}, %
\emph{Pauli}, \emph{axial}, and \emph{pseudoscalar} weak charged-current form factors of the nucleon.
These functions of $Q^2$ describe the spatial distributions of electric charge and current inside the nucleon %
and thus are intimately related to its internal structure.

CCQE interactions are particularly important to neutrino physics for mainly two reasons:
\begin{itemize}
	\item measurements of the differential cross-section in Eq.~\ref{eq:cc_xsec_q} give information on the %
		nucleon form-factors, which are difficult to measure; 
	\item their nature as two-body interactions enable the kinematics to be completely reconstructed, %
		and hence the initial neutrino energy determined which is critical for measuring the oscillation parameters.
\end{itemize}

In fact, if the target nucleon is at rest, at least compared to the neutrino energy, %
then this can be calculated as:
\begin{equation}
	E_\nu = \frac{m_n E_l + \frac{1}{2}\big ( m_p^2-m_n^2-m_l^2)}{m_n - E_l+p_l \cos \theta_l}\,,
\end{equation}
where the measurement of the momentum, $p_l$ and the angle with respect to the neutrino, $\theta_l$, of the %
outgoing charged lepton are only required.

Similar calculations can be made for the NCQE scatterings.
The cross-sections have the same form as the CC cross-sections in Eq.~\ref{eq:cc_xsec_q} and ~\ref{eq:cc_xsec_t}, %
without the mixing term $|V_{ud}|^2$ and with the proper nucleon form factors.
Since the values of the electromagnetic form factors, $F_1$ and $F_2$, are reasonably well known and the part %
in Eq.~\ref{eq:A(Q)} containing $G_P$ can be often neglected, thanks to the different mass magnitudes of %
leptons and nucleons, the axial form factor, $G_A$, can be determined through measurements of the charged-current %
quasielastic scattering processes.
On the contrary, measurements of the neutral-current elastic scattering cross-section give information %
on the \emph{strange} form factors of the nucleon, whose main contribute comes from the strange quark.

The low $Q^2$ region also presents an inelastic scattering contribution mostly affected by resonance production, %
where the nucleon is excited into a baryonic resonance before decaying.
At high $Q^2$, inelastic scattering is dominated by deep inelastic scattering (DIS), because the neutrino can scatter %
directly off a constituent quark, fragmenting the original nucleon.
In between these extreme scenarios, an additional contribution comes from interactions where the hadronic %
system is neither completely fragmented nor forms a recognisable resonance.
These interactions are referred to as ``shallow inelastic scattering'', and there is no clear model for dealing %
with them.

