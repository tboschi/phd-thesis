\clearpage
\chapter{Open charm production}
\label{cha:opencc}

\begin{figure}[t]
	\centering
	\begin{fmffile}{qq}
		\begin{fmfgraph*}(65,45)
			\fmfset{arrow_len}{2.5mm}
			\fmfleft{q1,q0}
			\fmfright{c1,c0}
			\fmflabel{$q$}{q0}
			\fmflabel{$\cj{q}$}{q1}
			\fmflabel{$c$}{c0}
			\fmflabel{$\cj{c}$}{c1}
			\fmf{fermion}{q0,vl,q1}
			\fmf{gluon}{vl,vr}
			\fmf{fermion}{c1,vr,c0}
		\end{fmfgraph*}
	\end{fmffile}
	\hspace{0.2em}
	\raisebox{1.5em}{,}
	\hspace{0.2em}
	\begin{fmffile}{gg_0}
		\begin{fmfgraph*}(65,45)
			\fmfset{arrow_len}{2.5mm}
			\fmfleft{g1,g0}
			\fmfright{c1,c0}
			\fmflabel{$g$}{g0}
			\fmflabel{$g$}{g1}
			\fmflabel{$c$}{c0}
			\fmflabel{$\cj{c}$}{c1}
			\fmf{gluon}{g0,vl}
			\fmf{gluon}{g1,vl}
			\fmf{gluon}{vr,vl}
			\fmf{fermion}{c1,vr,c0}
		\end{fmfgraph*}
	\end{fmffile}
	\hspace{0.1em}
	\raisebox{1.5em}{$+$}
	\hspace{0.1em}
	\begin{fmffile}{gg_1}
		\begin{fmfgraph*}(65,45)
			\fmfset{arrow_len}{2.5mm}
			\fmfleftn{g}{2}
			\fmfrightn{c}{2}
			\fmf{fermion}{c1,vt,vb,c2}
			\fmf{gluon}{g1,vt}
			\fmf{gluon}{g2,vb}
			\fmflabel{$g$}{g1}
			\fmflabel{$g$}{g2}
			\fmflabel{$\cj{c}$}{c1}
			\fmflabel{$c$}{c2}
		\end{fmfgraph*}
	\end{fmffile}
	\hspace{0.1em}
	\raisebox{1.5em}{$+$}
	\hspace{0.1em}
	\begin{fmffile}{gg_2}
		\begin{fmfgraph*}(65,45)
			\fmfset{arrow_len}{2.5mm}
			\fmfleftn{g}{2}
			\fmfrightn{c}{2}
			\fmf{gluon}{g1,vt}
			\fmf{gluon}{g2,vb}
			\fmf{phantom}{c2,vb,vt,c1}
			\fmf{fermion,tension=0}{c1,vb,vt,c2}
			\fmflabel{$g$}{g1}
			\fmflabel{$g$}{g2}
			\fmflabel{$\cj{c}$}{c1}
			\fmflabel{$c$}{c2}
		\end{fmfgraph*}
	\end{fmffile}
	\vspace{1em}
	\caption[Diagrams contributing to open charm production at the partonic level]%
		{These are the four diagrams contributing to the hard process in open charm production.
		The diagrams with gluons in the initial state interfere with each other giving rise to %
		cross terms in the colour structure.}
	\label{fig:parton}
\end{figure}

Following the same procedure as the one described in~\refref{Alekhin:2015byh}, %
the number of strange $D$ mesons can be estimated as
\begin{equation}
	\mathcal{N}_{D_s} = \frac{\sigma_{c \cj{c}}}{\sigma_{p A}} f_{D_s} = (2.8 \pm 0.2) \times \np{e-6}\ ,
\end{equation}
where $\sigma_{c \cj{c}} = \np{12}\pm\np{1}$\,\textmu b is the proton--target open charm cross section, %
$\sigma_{p A} = \np{331.4}\pm\np{3.4}$\,mb is the total inelastic proton--target on carbon ($A =$ \tapi{12}C)~\cite{RamanaMurthy:1975vfu} %
cross section, and $f_{D_s} = \np{7.7}\,\%$ is the $D_s$ fragmentation fraction~\cite{Abramowicz:2013eja}.
%where $\sigma_{c \cj{c}}$ is the proton-target open charm cross section, $\sigma_{p A}$ is the total inelastic %
%proton-target inelastic cross section, and $f_{D_s}$ is the $D_s$ fragmentation fraction.[h]
The open charm production cross section is computed at the leading order in perturbation theory, %
with a graphite fixed target and a 80\,GeV proton $p$.
The correct process to consider is the proton--nucleon interaction, therefore %
\begin{equation}
	\sigma_{c \cj{c}} \equiv \sigma(pA \to c \cj{c} + X) \approx A\,\sigma(pN \to c \cj{c} +X)\ , %
\end{equation}
using the correct Parton Distribution Function (PDF) for a bound nucleon $N$ in the nucleus $A$.
There are four diagrams, shown in figure~\reffig{fig:parton}, that contributes to the cross section, %
but three of them interfere with each other.
These cross sections are well-known SM calculations and can be found in~\refref{Tanabashi:2018oca}.
The integrated cross section is:
\begin{multline}
	\sigma(pN \to c \cj{c} + X) = \int_{\tau_0}^1 \dd{x_1} \int_{\frac{\tau_0}{x_1}}^1 \dd{x_2} \int \dd{\Omega} %
	\bigg[\qty(f_{g/p}^1\, f_{g/A}^2 + f_{g/p}^2\, f_{g/A}^1) \dv{\sigma_{gg \to c \cj{c}}}{\Omega} \\
	+\sum_{q=u,d,s} \qty(f_{q/p}^1\, f_{\cj{q}/A}^2 + f_{q/p}^2\, f_{\cj{q}/A}^1 + %
	f_{\cj{q}/p}^1\, f_{q/A}^2 + f_{\cj{q}/p}^2\, f_{q/A}^1) \dv{\sigma_{q\cj{q} \to c \cj{c}}}{\Omega}\bigg]\ ,
\end{multline}
with $\tau_0 = \hat{s}_0 / s$ and $\hat{s}_0$ being the threshold energy at the partonic level and %
$s = 2 m_p ( m_p + E_p)$ is the centre of mass energy, given that $m_p \simeq m_n$.
The partonic structure of the nucleus is described by the functions $f_{\rho/\eta}^i = f_{\rho/\eta}(x_i, M_F)$, %
which are interpreted as the probability of finding a parton $\rho$ in the particle $\eta$ %
carrying a $x_i$ fraction of the momentum of $\eta$, at the energy scale $M_F$.	
The two momentum fractions are related by $x_1\,x_2\,s = \hat{s}$, where the hat symbol denotes the energy %
of the parton-level process.



A factorisation scale of $M_F = 2.1\, m_c$ for the computation of $\sigma_{c \cj{c}}$ is adopted, 
while the renormalisation scale of $\alpha_s$ is set to $\mu_R = 1.6\, m_c$, and the charm mass has the value %
\mbox{$m_c = (\np{1.28}\pm\np{0.03})$\,GeV}.
The integration is regulated for $|\cos \theta| < 0.8$, with $\theta$ the angle in the centre of mass frame.
The theoretical curve in Fig.~7.4(a) of~\refref{Alekhin:2015byh} was used to check this calculation, %
and it was successfully reproduced up to NLO corrections.
For the generation of PDF the LHAPDF libraries~\cite{Buckley:2014ana} %
and the nCTEQ15 PDF set~\cite{Kovarik:2015cma} are used, %
resulting in $\sigma_{pA \to c \cj{c}} = (\np{12}\pm1)\,\mu$b, for an 80\,GeV protons on a graphite target.
