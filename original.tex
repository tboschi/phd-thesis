\clearpage

I, Tommaso Boschi, confirm that the research included within this thesis %
is my own work or that where it has been carried out in collaboration with, %
or supported by others, that this is duly acknowledged below and my contribution indicated.
Previously published material is also acknowledged below.

\medskip
\noindent
I attest that I have exercised reasonable care to ensure that the work is original, %
and does not to the best of my knowledge break any UK law, infringe any third party’s copyright %
or other Intellectual Property Right, or contain any confidential material.

\medskip
\noindent
I accept that the College has the right to use plagiarism detection %
software to check the electronic version of the thesis.

\medskip
\noindent
I confirm that this thesis has not been previously submitted %
for the award of a degree by this or any other university.

\medskip
\noindent
The copyright of this thesis rests with the author and no quotation from it %
or information derived from it may be published without the prior written consent of the author.

\medskip
Signature: 

Date:

\bigskip
Details of collaboration and publications:

\refcha{cha:intro} and partially \refchas{cha:skgd}{cha:cp_hk} provide an overview of background work, %
which does not constitute original research of this thesis.
The remainders of \refcha{cha:skgd} and \refcha{cha:cp_hk} are completed %
as part of respectively the Super-Kamiokande and Hyper-Kamiokande collaborations.
In \refcha{cha:skgd}, the simulation of the californium-252 source and the development %
of a gadolinium concentration monitoring device were carried out by the author in collaboration with %
B. Richards of the University of Warwick.
The neutron calibration procedure was developed by members of the Super-Kamiokande collaboration. 
The extensive thesting of the gadolinium-loaded water Cherenkov techinque was carried by the Super-Kamiokande and EGADS collaborations.
In \refcha{cha:cp_hk}, the oscillation analysis was developed by members of the Super-Kamiokande and Hyper-Kamiokande collaborations, %
in particular R. Wendell and M. Jiang.
The improvements to the fitting framework was developed by the author, thanks to the support of members of the Hyper-Kamiokande collaborations, %
especially M. Friend, M. Scott, and T. Dealtry.
Inputs to the analysis are provided by T. Dealtry.
The work of \refchas{cha:mass_models}{cha:hnl_dune} was performed in collaboration with %
S. Pascoli and P. Ballett of the University of Durham, as published in \refref{Ballett:2019bgd}.
The discussion on neutrino mass models of \refcha{cha:mass_models} was led by S. Pascoli with %
fundamental inputs by P. Ballett.
The derivation of the formulae of \refsecs{sec:decay}{sec:production} was done by the author and P. Ballett.
The prediction of the $\nu_\tau$ flux contribution in \refcha{cha:hnl_dune} was done by the author, %
thanks to the useful discussions with R. Ruiz, %
whereas the other contributions to the flux were provided by the DUNE collaboration.
The simulation of the Near Detector of DUNE in \refcha{cha:hnl_dune} was developed by the author, %
as well as the framework to carry out sensitivity studies.
The mass matrix scan in \refsec{sec:combined} was carried out by the author, thanks to the %
useful discussions with C. Weiland and M. Lucente.
