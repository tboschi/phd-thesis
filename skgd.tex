\chapter{The gadolinium phase of Super-Kamiokande}
\label{cha:skgd}

The typical total cross-section for an accelerator neutrino \mbox{($\sim1$\,GeV)} %
is of the order of \np{e-38}\,cm\tapi{2} and for a solar neutrino ($\lesssim10$\,MeV) %
it is roughly one order of magnitude lower, as it can be seen from \reffigs{fig:cevns}{fig:xsec}.
These small values make the detection of neutrinos a challenging task.
Neutrino experiments usually compensate the small cross-section by instrumenting large active volumes, %
so that a statistically significant amount of neutrino interactions with matter can be recorded.
By studying the final state particles some knowledge on the incoming neutrino may be retrieved.
Large volumes however are prone to intense background too, either from cosmic rays or natural radioactivity.
Passive or active techniques to reduce such undesired events must be therefore put into place.
Neutrino physics has greatly progressed in recent times, thanks to both massive detectors and %
improvements in detection technologies.
The most emblematic example of neutrino experiments are Sudbury Neutrino Observatory (SNO) and Super-Kamiokande (SK) %
which respectively demonstrated solar~\cite{Aharmim:2005gt} and atmospheric~\cite{Fukuda:1998mi} neutrino oscillations.
The picture of three-flavour neutrino oscillations has since been studied in depth %
by numerous experiments with not only atmospheric and solar neutrinos, but also with neutrinos from reactors and accelerators.
Current and future neutrino experiments rely on the experimental techniques that were pioneered and perfected %
by these two successful experiments.

SK and SNO are both Cherenkov detectors, in which a large body of water---heavy water in the case of SNO---is surrounded by %
photodetectors to capture Cherenkov radiation emitted by the charged leptons produced in neutrino CC and NC interactions.
The volumes of SNO and SK are located in mines so as to take advantage of the thick rock overburden, %
which acts as a passive shield to cosmic muons.
An outer detector system provides a passive and active veto to the fiducial region of both experiments.
These simple measures are responsible for the suppression of the majority of backgrounds.
However, backgrounds from atmospheric neutrino interactions with or without neutron emission together %
with $\beta$-decay from spallation events induced by cosmic rays are more difficult to control.
These events can generate electrons or positrons that pass the selection criteria and so contaminate %
candidate samples of inverse beta decay (IBD, see \refsec{sec:ccqe})~\cite{Zhang:2013tua, Super-Kamiokande:2015xra}.
%The most difficult background to control, however, is given by neutrons which are typically produced by spallation events %
%induced by cosmic rays or by the decay chains of radioactive elements---such as uranium and thorium---naturally %
%present in the detector's components.
As we are entering a precision era for neutrino experiments, the capability of detecting neutrons %
in signal and background events is becoming a more and more crucial requisite, % ??
especially for detectors studying solar, supernova, or reactor neutrinos.
One of the most promising approaches is the addition of gadolinium to the active medium of the detector.
Certain isotopes of Gd have a very high cross-section for neutron capture %thermal 
accompanied with the emission of high energy photons, which makes neutron detection more efficient.
Many existing neutrino experiments are already using Gd-loaded water or Gd-loaded scintillator for neutron tagging, %
such as EGADS~\cite{Ikeda:2019pcm}, ANNIE~\cite{Back:2019aqi}, and RENO~\cite{Ahn:2010vy}, %
and future experiments are planning to use the same principle.

In the next phase of SK, hundred tons of gadolinium sulphate will be dissolved in water, making %
Super-Kamiokande the largest Gd-doped water Cherenkov detector.	% expand this point
The benefits and challenges of this technique are extensively covered in this chapter, %
after a review of the working principle of a generic Cherenkov detector in \refsec{sec:wch}, %
and the Super-Kamiokande experiment in \refsec{sec:sk}.
The theory of neutron thermalisation and neutron capture is explained in \refsec{sec:sk_neutron}.
In the same chapter, neutron calibration in SK is discussed together with %
possible improvements of the neutron calibration device studied for this thesis.
The SK Gd-phase is outlined in \refsec{sec:gadolinium} and in \refsec{sec:gad} %
a novel technique to monitor gadolinium concentration in water using UV spectroscopy is presented.

\section{Cherenkov detectors}
\label{sec:wch}

%One of the most promising techniques is to combine liquid argon with time projection chambers.
%As with most other liquefied noble gases, argon has a high scintillation light yield %
%(ca 51~photons/keV[arXiv:1004.0373]), is transparent to its own scintillation light, and is relatively easy to purify.
%Compared to xenon, argon is also cheaper and has a distinct scintillation time profile which allows the separation %
%of electronic recoils from nuclear recoils.

Light travelling through a transparent material undergoes a reduction of its phase velocity %
due to a superposition with the electromagnetic fields of the electrons in the medium. %, which can be %
The change in velocity will therefore depend on the frequency of the incoming photons.
%polarized both electrically and mangetically.
The ratio of the new phase velocity and the speed of light in vacuum defines the \emph{refractive index} %
of the material:
\begin{equation}
	\label{eq:ref_index}
	n(\lambda) = \frac{c}{v_P(\lambda)}\ ,
\end{equation}
which is greater than one by definition.

A charged particle moving at a velocity faster than the speed of light in a medium %
emits a coherent electromagnetic radiation, called \emph{Cherenkov radiation}. %
Provided that the distance covered by the particle in the medium is large compared to the emitted wavelength, %
%and that the speed of the particle is constant relative to the period of emission $\tau = \frac{\lambda}{c}$,
the radiation is produced when~\cite{Cerenkov:1937vh}
\begin{equation}
	\label{eq:cherenkov}
	\beta = \frac{v}{c} > \frac{1}{n}\ ,
\end{equation}
where $v$ is the velocity of the particle.
The minimum energy of the particle to reach this condition is therefore
\begin{equation}
	\label{eq:cherenkov_threshold}
	E_\text{thr} = m\ \sqrt{1 + \frac{1}{n^2-1}}\ ,
\end{equation}
with $m$ the mass of the charged particle.
As the particle is moving faster than $c / n$, the wave front of the EM radiation forms a cone %
which follows the charged particle.
%Particles of mass $m$ with energy above the threshold will radiate Cherenkov light in the shape of a cone, %
%which has an angular aperture $\theta$ given by
The angular aperture $\theta$ of such cone is given by
\begin{equation}
	\label{eq:cherenkov_angle}
	\cos \theta = \frac{1}{n\,\beta} \ .
\end{equation}
The maximum angle is reached by ultrarelativistic particles with $\beta \simeq 1$.
The charged particle, however, will typically lose energy in the medium due to ionisation, %
slowing down until its energy falls below the threshold $E_\text{thr}$.
As soon as the condition of \refeq{eq:cherenkov} does not hold anymore, %
the particle stops emitting radiation and the cone of light reduces to a truncated cone, %
which forms a ring when projected onto a surface.

Many particle physics experiments take advantage of this effect, in order to convert the passage of %
charged particles into detectable light.
A volume of transparent material, such as water, ice, aerogel, %
can be instrumented with photodetectors to capture Cherenkov radiation.
Liquid scintillators are also employed in such detectors despite not being transparent to the radiation: %
most scintillators absorb the Cherenkov light and re-emit it isotropically at a different wavelength, %
therefore loosing the information on directionality.
The number of photons emitted by a charged particle of charge $z$ per unit path length and per unit %
energy interval, or equivalently to $\lambda$, has a distribution that follows the Frank-Tamm formula~\cite{Frank:1937fk}:
\begin{equation}
	\label{eq:franktamm}
	\pdv{N}{x}{\lambda} = \frac{2\,\pi\,\alpha\,z^2}{\lambda^2} %
	\qty(1-\frac{1}{\beta^2 n^2(\lambda)}) = %
	\frac{2\,\pi\,\alpha\,z^2}{\lambda^2} \sin^2\theta\ ,
\end{equation}
where $\alpha$ is the fine structure constant and $z$ is the charge of the particle.
Due to the inverse dependence on $\lambda^2$, most of the Cherenkov photons are emitted at shorter wavelengths.
However, a real medium is always dispersive and allowed frequencies are restricted to the region for which $n(\lambda) > \flatfrac{1}{\beta}$.
The radiation is hence typically emitted in the near visible and ultraviolet regions of the EM spectrum.
At higher frequencies, for example in the x-ray region, the refractive index drops below one and %
Cherenkov photons cannot be produced at these shorter wavelengths.
Experiments exploiting the Cherenkov effect try to match the sensitivity band of photodetectors to the radiation region.
Photomultipliers (PMT) are the detectors of choice, thanks to their low noise and capability of single photodetection, %
but microchannel plates (MCP) and silicon photomultipliers (SiPM) are used too (see for e.g.\ \cite{Ambrosio:2016ijk}).

This technique is largely used in neutrino detection, since it easily allows %
to transform large volumes of some transparent or scintillating medium into a detector sensitive to charged particles.
Amongst the most notable examples, the IceCube experiment is the largest Cherenkov detector~\cite{Abbasi:2008aa}, %
in which more than five thousands PMTs are deployed into the Antarctic ice, covering a volume of one cubic kilometre.
Charged-current interactions of neutrinos on a nucleon produce charged leptons %
that are likely to acquire most of the incoming neutrino momentum thanks to the heavy mass of the nucleon.
If the energy of the outgoing lepton is above the threshold of \refeq{eq:cherenkov_threshold}, %
the emitted radiation can be collected and used to reconstruct the interacting neutrino's properties, with some limited resolution.
The photons collected on the light sensors are correlated to the particle's energy, %
whereas the location and timing of the hit photodetectors is used to reconstruct the vertex %
and the direction of the interaction.
The directionality information is lost in scintillator Cherenkov detectors, because %
scintillating materials typically emit light isotropically.
The topology and pattern of the radiation collected is often used for event classification, %
but a preciser particle identification is possible in certain cases, such as in SK which is discussed in the following section.
%Most detectors are capable of event classification particle identification or some sort of classification, and the particle identification capabilities %
%of SK are covered in the following section. 


%%% icecube pid excessive, talk about T2K
%In IceCube, for example, events are classified in the following categories:
%``cascades'', typically generated by CC interactions of $\nu_e$ or NC interactions;
%``tracks'', produced by very energetic muons; ``double bangs'', %
%the results of a $\nu_\tau$ producing a tau lepton, which decays inside the detector.

\section{The Super-Kamiokande experiment}
\label{sec:sk}

\begin{figure}
	\centering
	\includegraphics[width=0.50\textwidth]{pics/superk-schematic-tags.png}
	\hfill
	\includegraphics[width=0.40\textwidth]{pics/superk-internal.jpeg}
	\caption[View of the Super-Kamiokande detector]%
	{Cut-open view (left) of the Super-Kamiokande detector in Kamioka mine, %
	located at the centre of Mt.\ Ikeno.
	On the right, a wide-angle picture of the inside of the tank; each golden bulb is a 20'' PMT.}
	\label{fig:sk_scheme}
\end{figure}

The Super-Kamiokande experiment (SK)~\cite{Fukuda:2002uc} is a water Cherenkov detector, %
located in the Kamioka mine, under Mt.\ Ikeno in the Gifu prefecture, Japan.
A schematic of the cavern is shown in~\reffig{fig:sk_scheme}.
The rock overburden of $\sim\np{1000}$\,m (\np{2700}\,m.w.e.), shields very efficiently the experiment %
and only cosmic rays with energies above 1.3\,TeV can reach the detector: %
the muon flux at SK is about \np{6e-8}\,cm\tapi{-2}s\tapi{-1}sr\tapi{-1}, which translates to a rate of 2\,Hz %
in the fiducial volume.
The detector consists of a cylindrical stainless steel tank, with a height of 41.4\,m and a diameter of 39.3\,m, %
and it is filled with 50\,kt of ultra pure water.
The water region is separated in two concentric cylindrical regions, %
called the inner detector (ID, 33.8\,m in diameter) and the outer detector (OD), the latter working as a passive and active veto.
The two regions are separated by a 55\,cm insensitive region containing the support structure for the PMTs and their cables.
The ID is instrumented with \np{11129} 20'' Hamamatsu R3600 PMTs, facing towards the inside of the tank.
Since the 2001 implosion accident, the photocathode of the PMTs are protected by UV--transparent acrylic domes %
and their sides with fibre-reinforced plastic.
The photo coverage of the inner surface is around 40\,\%; the remaining surface not occupied by a PMT is %
lined with black sheet to reduce light reflection.
The main goal of the OD is to tag events originating or finishing outside the ID.
The only signal events that can originate inside the ID without triggering the OD are neutrino or proton decay events.
The OD is instrumented with \np{1885} 8'' Hamamatsu R1408 PMTs which are optically coupled to wavelength shifting plates %
to increase light collection.
Sheets of white tyvek maximise the propagation of photons inside the OD and help the reconstruction of %
events occurring at the edges of the cylinder tank, also known as \emph{corner-clipping events}.

\begin{figure}
	\centering
	\includegraphics[width=0.48\textwidth]{pics/Electron.pdf}
	\hfill
	\includegraphics[width=0.48\textwidth]{pics/Muon.pdf}
	\caption[Reconstructed fully-contained events in Super-Kamiokande]%
	{Reconstructed fully-contained events in the fiducial volume of Super-Kamiokande.
	Electron rings (left) are less defined than muon ones (right) because they are scattered more %
	along their path.
	The developed cylinders on the top right of each figure show the outer detector hits.
	The histograms on the bottom right corners are the time distributions of the events.}
	\label{fig:sk_events}
\end{figure}

The refractive index of water in the visible and ultraviolet region is $n \geq 1.33$, %
and so a charged particle can emit Cherenkov radiation if it is travelling %
with speed $\beta \gtrsim 0.75$.
This translates to an energy threshold of \np{0.78}\,MeV for electrons, \np{160.26}\,MeV for muons, %
\np{211.69}\,MeV for pions, and \np{1423.13}\,MeV for protons, which are the particles usually detected in SK.
The Cherenkov photons, thanks to the high-purity water, reach the walls of the tank where they %
are collected by the PMTs, forming ring patterns.
From the time and charge deposited on the PMTs, the event reconstruction algorithm extracts information of the event %
such as vertex, direction, and energy.
%High energy events, for which $\beta \simeq 1$, generates Cherenkov rings with the same conic angle, %
%which is $\theta \simeq 41^\circ$ in water. 
The same algorithm performs particle identification by looking at the topology and the multiplicity of the ring patterns: %
for example, an electron travelling in the water is more likely to be scattered compared to a muon %
which on the other hand will follow a more straight path.
The electron-induced ring is therefore less defined and fuzzier than the muon-induced one.
This can be appreciated from the reconstructed events of \reffig{fig:sk_events} showing an electron-like and %
a muon-like ring originated from neutrino interactions.
Pions usually interact hadronically with oxygen nuclei, and the high cross-section does not allow long tracks; %
their Cherenkov rings are also less defined than the muon ones.
Muon and pion events can be accompanied by secondary rings from the electrons produced in their decays.
Energetic gamma rays can also be detected, as for instance the ones from $\pi^0$ decays, %
thanks to the rings produced by Compton-scattered or pair-produced electrons.
%Cosmic rays are easily identified, not only because of the light deposited in the OD, %
%but also thanks to the strong trail


As part of the strategies to tackle backgrounds, only events originating at the centre %
of the ID are considered to be valid neutrino interactions.
%Events producing substanial hits in the OD are likely to be bac
The materials of the walls of the tank, and particularly the glass of the PMTs, contain radioactive material, %
such as isotopes of thorium, uranium and potassium.
These impurities can mimic a signal event in the MeV energy region, crucial for solar neutrino studies.
The selection algorithm in most of the SK analyses simply discards any reconstructed event %
with the vertex less than 200\,cm away from the ID walls.
This criterion defines a fiducial volume (FV), which is around 22.5\,kt in water mass.
The water used to fill the tank may also contain impurities, among which radon, that cause background events.
Thus, the water in the tank is continuously filtered at a flow rate of 60\,t/hour by a water purification system (see \refsec{sec:gadolinium}).
An air purification system pumps radon-free air into the SK area inside the mine, %
and therefore less radon can dissolve from air into the water system.
%This prevents radon in the air from dissolving in the clean water, with a %
The level of contamination reached is usually less than 3\,mBq/m\tapi{3}.
As a comparison, unpurified air can peak at about 1200\,Bq/m\tapi{3}.
Thanks to these precautions and state-of-the-art electronics~\cite{Nishino:2009zu}, the current energy threshold for SK is \np{3.5}\,MeV.


\section{Neutrons in Super-Kamiokande}
\label{sec:sk_neutron}

Super-Kamiokande and other large scale water Cherenkov detectors have provided %
much evidence in the experimental understanding of neutrinos, %
be it originated in solar, atmospheric, or accelerator facilities.
Despite the large exposure of the experiment, some studies are still curbed by statistical uncertainty %
and would benefit from simply collecting more data.
Other searches, instead, suffer from irreducible background events, as for example the detection of Supernova Relic Neutrinos (SRN), %
%One of the main motivation to detect neutrons is to be able to detect relic supernova neutrinos, %
an incoherent background spectrum produced from all previous core-collapse supernova explosions in the Universe (see \refsec{sec:nu_sun_sn}).
The observation of SRN would be of great importance for improving our knowledge on the population of core-collapse SN %
and the rate of star formation.
The average energies of these neutrinos are around a few tens of MeV, at which the largest cross-section %
corresponds to inverse beta decay (IBD, see \reffig{sec:xsec}).
%which is two orders of magnitude larger than $\nu_e$ elastic scattering.
This measurement is afflicted for energies between $14\,\text{MeV} < E_\nu < 30\,\text{MeV}$ %
by nonrelativistic muons produced from atmospheric neutrinos which are below Cherenkov threshold and decay into electrons, %
known as \emph{invisible muon decays}~\cite{Kaplinghat:1999xi}.
At lower energy, the dominant backgrounds are decays of muon-induced spallation products, %
between 10\,MeV and 18\,MeV, and reactor neutrinos below 10\,MeV.
No SRN has been detected yet~\cite{Bays:2011si, Zhang:2013tua}, but %
theoretical predictions place the diffuse supernova neutrino background flux (DSNB, see \refsec{sec:nu_sun_sn}) %
within a factor of four below the upper limit obtained by SK in 2003~\cite{Malek:2002ns, Horiuchi:2008jz}.
Even though a more recent and detailed analysis with increased efficiency, lower energy threshold, and expanded statistics %
revealed less stringet limits~\cite{Bays:2011si}, neutron tagging capabilities could reduce %
the invisible muon background by a factor of five and remove the spallation background.
%upper limits on supernova relic neutrino flux are between 2.8 and 3:1 e cm2 s1 > 16 MeV total
%positron energy (17.3 MeV E).


Searches of galactic supernova bursts could also avail of an improved background suppression.
SK will be exposed to a huge number of neutrino events if a core-collapse supernova occurs inside our galaxy.
Such an event would provide not only an early warning system for other observatories, %
but also information about the neutrino spectrum, time profile, and information about early stages of the core-collapse process, %
like pre-supernova neutrinos from the silicon burning phase~\cite{Simpson:2019xwo}.
A sizeable amount of the neutrinos emitted are actually antineutrinos in the energies where the IBD cross-section dominates.
%(see \reffig{fig:xsec}).
Detecting the neutron in the final state is fundamental for distinguishing whether the incoming particle is %
a neutrino or an antineutrino.
Improved neutron tagging is also helpful to understand atmospheric and accelerator neutrino interactions and final states.
At energies $E_\nu > 1$\,GeV, the number of CCQE interactions decreases and often additional neutrons %
from $2p2h$ interactions or pions from baryon resonance are also released in the scattering process.
The possibility of separating neutrinos and antineutrinos can nevertheless improve the purity of antineutrino samples.
Finally, neutron-induced background in proton decay searches can be reduced, %
since most channels do not require neutrons to appear in the final state.
%since it requires no neutrons to appear in the final state.

%Among these background sources, flasher events\footnote{A flasher event refers to the phenomenon %
%	in which a PMT mis-fires and many repetitive internal discharges are triggered also on nearby PMTs.
%	Flasher events tend to have a broader hit timing distribution than neutrino events.} %

\subsection{Neutron tagging}
\label{sec:neutron_tagging}

Neutrons have a very long lifetime and since they cannot directly emit Cherenkov radiation %
they can travel quite some distance from their production point without being detected.
Neutrons interact with nuclei in various ways, depending on their energy and the target nucleus.
The interactions are divided in three typologies: scatterings, which can be elastic or inelastic, %
absorptions, such as radiative capture or fission, and nucleon transfer reactions.
The kinetic energy of the neutron $T$, sometimes referred to as the \emph{detection temperature}, %
is the main factor in determining the dominant interaction mode. %cross-section of these interactions.
For fast neutrons, with $T > 1$\,MeV, elastic scattering is the prevailing process, and %interaction mode, and %
for some target nuclei the only mode possible.
The cross-section for these energetic neutrons present some resonance peaks the structure of which depend %
on the target nuclei.
The elastic scattering cross-section at energies below 1\,MeV is almost independent of the neutron energy.
This is true for most light isotopes, but some intermediate and heavy elements present some specific %
behaviour at higher energy.
Neutrons with a kinetic energy $T \simeq \np{0.025}$\,eV are called \emph{thermal neutrons} because %
a Maxwell-Boltzmann distribution at the temperature of 290\,K peaks at that energy; %
such neutrons are in thermal equilibrium with the surrounding medium.
At thermal energies and below, the most important interaction is neutron absorption, %
the cross-section of which follows the ``$1/v$ law'', with $v$ being the velocity of the neutron.
In this region, the absorption cross-section increases as the velocity of the neutron, i.e.\ its temperature, decreases:
\begin{equation}
	\label{eq:abs_xsec}
	\sigma \sim \frac{1}{v} \sim \frac{1}{T}\ .
\end{equation}
Being in thermal equilibrium and having a lower kinetic energy allows the neutron to form %
a compound nucleus with the target, which might undergo fission or radiation emission if it is a nonfissile nuclide.
For energies between a few eV and hundreds of keV there are resonances in the capture cross-section which %
are strongly dependent on the target isotope, but also on the temperature of the material.
In fact, the thermal motions of the target relative to the incident neutron broadens the resonance peaks, %
even though the integrated cross-section over the energy range remains constant.
This effect is called \emph{Doppler broadening}, and results in a decreased likelihood of capture or fission %
when the target material has a wide energy distribution.

%%https://doi.org/10.1016/j.nds.2018.02.001
%BBrown:2018jhjrown:2018jhj,

Fast neutrons slow down to thermal energies via subsequent elastic collisions, until %
the free neutron is captured by some nucleus.
This process is called \emph{thermalisation}.
Considering a scattering between a neutron and a nucleus,
\begin{equation}
       n + A \to n + A\ ,
\end{equation}
the final state energy of the neutron depends on the outgoing angle $\hat\theta$ %
in the centre of mass frame
\begin{equation}
	\label{eq:outgoing_E}
	E_\text{f} = E_\text{i}\ \frac{(1+\zeta) + (1 - \zeta) \cos\hat\theta}{2} \ ,
\end{equation}
where $E_\text{i,f}$ are the initial and final energies of the free neutron.
The kinematic factor $\zeta$ is defined as
\begin{equation}
	\label{eq:n_alpha}
	\zeta = \frac{(m_n - M_A)^2}{(m_n + M_A)^2} \simeq \frac{(1 - A)^2}{(1 + A)^2}\ ,
\end{equation}
where $m_n$ is the neutron mass, $M_A \approx A\, m_n$ is the mass of the nucleon and $A$ the mass number.
Using classic mechanics, the scattering angle $\theta$ in the laboratory frame is given by
\begin{equation}
	\label{eq:lab_angle}
	\cos\theta = \frac{A\,\cos\hat\theta + 1}{\sqrt{1 + 2\,A\,\cos\hat\theta + A^2}}\ ,
\end{equation}
and the average scattering angle can be therefore estimated as
\begin{equation}
	\label{eq:average_angle}
	\expval{\cos\theta} = \frac{\displaystyle \int\dd{\Omega} \cos\theta}{\displaystyle \int \dd{\Omega}} = \frac{2}{3 A}\ .
\end{equation}
%Heavy targets will deviate the neutron from its path more than light targets on average.
On average, heavy targets will modify the neutron's path more significantly than light ones.
For hydrogen ($A = 1$), the average angle is $\expval{\theta} \simeq 48^\circ$, but %
for heavier elements with large atomic mass the average angle $\expval{\theta} \to 90^\circ$.
Assuming uniform isotropic scattering, the average outgoing energy for the neutron is
\begin{equation}
	\label{eq:average_outgoing_E}
	\expval{E_\text{f}} = \frac{\displaystyle \int \dd{\Omega} E_\text{f}}{\displaystyle \int\dd{\Omega}} = E_\text{i}\ \frac{1+\zeta}{2}\ .
\end{equation}
This means that if each collision decreases the neutron energy by the same factor as in \refeq{eq:average_outgoing_E}, 
%if each neutron collision decreases to the same average final state energy, %
after $n$ interactions the neutron energy is
\begin{equation}
	\label{eq:energy_loss}
	E_n = E_0 \qty(\frac{1+\zeta}{2})^n = E_0\,e^{-n \xi}\ .
\end{equation}
The thermalisation process follows an exponential law and $\xi$ represents the typical %
energy decrease in log-scale, which on average equals to
\begin{equation}
	\label{eq:average_decrease}
	\expval{\xi} = \frac{\displaystyle \int \dd{\Omega} \log\qty(\frac{E_\text{i}}{E_\text{f}})}{\displaystyle \int\dd{\Omega}} %
		     = \frac{\zeta - \zeta \log \zeta -1 }{\zeta - 1} = 1 + \frac{(A-1)^2}{2A} \log \frac{A-1}{A+1}\ .
\end{equation}
The average number of collisions a neutron undergoes when %
propagating in a medium can be then computed with the above expression.
Lighter nuclides appear to be more effective at slowing down neutrons than heavier ones because of the $A$-dependency of~$\xi$.
For example, the number of interactions necessary for a fast neutron to decelerate to thermal energies in lead ($A = 208$) %
is about hundred times the number of interactions that would occur in hydrogen.
In most cases, the media in which neutrons propagate are compounds with more than one type of target.
The energy decrease for these materials is an average of the individual components weighted by %
their respective elastic cross-sections
\begin{equation}
	\label{eq:decrease_weighted}
	\xi_\text{tot}(E) = \frac{\sum_k \sigma_k(E) \xi_k(E)}{\sum_k \sigma_k(E)}\ .
\end{equation}
In this simplified scenario, the impact of resonance capture has been overlooked, %
which is only manifest at around the keV energy range.
While fast neutrons with energies above 1\,MeV are being slowed down to thermal energies, %
the mean free path decreases with the neutron's velocity, and the possibility of resonance capture increases.
These captures affect the relative neutron flux around the energy of the resonance peaks, which %
might be widened by the Doppler broadening effect.

\begin{figure}
	\centering
	\resizebox{\linewidth}{!}{% GNUPLOT: LaTeX picture with Postscript
\begingroup
  \makeatletter
  \providecommand\color[2][]{%
    \GenericError{(gnuplot) \space\space\space\@spaces}{%
      Package color not loaded in conjunction with
      terminal option `colourtext'%
    }{See the gnuplot documentation for explanation.%
    }{Either use 'blacktext' in gnuplot or load the package
      color.sty in LaTeX.}%
    \renewcommand\color[2][]{}%
  }%
  \providecommand\includegraphics[2][]{%
    \GenericError{(gnuplot) \space\space\space\@spaces}{%
      Package graphicx or graphics not loaded%
    }{See the gnuplot documentation for explanation.%
    }{The gnuplot epslatex terminal needs graphicx.sty or graphics.sty.}%
    \renewcommand\includegraphics[2][]{}%
  }%
  \providecommand\rotatebox[2]{#2}%
  \@ifundefined{ifGPcolor}{%
    \newif\ifGPcolor
    \GPcolortrue
  }{}%
  \@ifundefined{ifGPblacktext}{%
    \newif\ifGPblacktext
    \GPblacktexttrue
  }{}%
  % define a \g@addto@macro without @ in the name:
  \let\gplgaddtomacro\g@addto@macro
  % define empty templates for all commands taking text:
  \gdef\gplbacktext{}%
  \gdef\gplfronttext{}%
  \makeatother
  \ifGPblacktext
    % no textcolor at all
    \def\colorrgb#1{}%
    \def\colorgray#1{}%
  \else
    % gray or color?
    \ifGPcolor
      \def\colorrgb#1{\color[rgb]{#1}}%
      \def\colorgray#1{\color[gray]{#1}}%
      \expandafter\def\csname LTw\endcsname{\color{white}}%
      \expandafter\def\csname LTb\endcsname{\color{black}}%
      \expandafter\def\csname LTa\endcsname{\color{black}}%
      \expandafter\def\csname LT0\endcsname{\color[rgb]{1,0,0}}%
      \expandafter\def\csname LT1\endcsname{\color[rgb]{0,1,0}}%
      \expandafter\def\csname LT2\endcsname{\color[rgb]{0,0,1}}%
      \expandafter\def\csname LT3\endcsname{\color[rgb]{1,0,1}}%
      \expandafter\def\csname LT4\endcsname{\color[rgb]{0,1,1}}%
      \expandafter\def\csname LT5\endcsname{\color[rgb]{1,1,0}}%
      \expandafter\def\csname LT6\endcsname{\color[rgb]{0,0,0}}%
      \expandafter\def\csname LT7\endcsname{\color[rgb]{1,0.3,0}}%
      \expandafter\def\csname LT8\endcsname{\color[rgb]{0.5,0.5,0.5}}%
    \else
      % gray
      \def\colorrgb#1{\color{black}}%
      \def\colorgray#1{\color[gray]{#1}}%
      \expandafter\def\csname LTw\endcsname{\color{white}}%
      \expandafter\def\csname LTb\endcsname{\color{black}}%
      \expandafter\def\csname LTa\endcsname{\color{black}}%
      \expandafter\def\csname LT0\endcsname{\color{black}}%
      \expandafter\def\csname LT1\endcsname{\color{black}}%
      \expandafter\def\csname LT2\endcsname{\color{black}}%
      \expandafter\def\csname LT3\endcsname{\color{black}}%
      \expandafter\def\csname LT4\endcsname{\color{black}}%
      \expandafter\def\csname LT5\endcsname{\color{black}}%
      \expandafter\def\csname LT6\endcsname{\color{black}}%
      \expandafter\def\csname LT7\endcsname{\color{black}}%
      \expandafter\def\csname LT8\endcsname{\color{black}}%
    \fi
  \fi
    \setlength{\unitlength}{0.0500bp}%
    \ifx\gptboxheight\undefined%
      \newlength{\gptboxheight}%
      \newlength{\gptboxwidth}%
      \newsavebox{\gptboxtext}%
    \fi%
    \setlength{\fboxrule}{0.5pt}%
    \setlength{\fboxsep}{1pt}%
\begin{picture}(14400.00,5760.00)%
    \gplgaddtomacro\gplbacktext{%
      \csname LTb\endcsname%%
      \put(618,977){\makebox(0,0)[r]{\strut{}\np{e-4}}}%
      \csname LTb\endcsname%%
      \put(618,1742){\makebox(0,0)[r]{\strut{}\np{e-2}}}%
      \csname LTb\endcsname%%
      \put(618,2507){\makebox(0,0)[r]{\strut{}\np{1}}}%
      \csname LTb\endcsname%%
      \put(618,3271){\makebox(0,0)[r]{\strut{}\np{e2}}}%
      \csname LTb\endcsname%%
      \put(618,4036){\makebox(0,0)[r]{\strut{}\np{e4}}}%
      \csname LTb\endcsname%%
      \put(618,4801){\makebox(0,0)[r]{\strut{}\np{e6}}}%
      \csname LTb\endcsname%%
      \put(1147,409){\makebox(0,0){\strut{}\np{e-4}}}%
      \csname LTb\endcsname%%
      \put(2002,409){\makebox(0,0){\strut{}\np{e-2}}}%
      \csname LTb\endcsname%%
      \put(2857,409){\makebox(0,0){\strut{}\np{1}}}%
      \csname LTb\endcsname%%
      \put(3712,409){\makebox(0,0){\strut{}\np{e2}}}%
      \csname LTb\endcsname%%
      \put(4567,409){\makebox(0,0){\strut{}\np{e4}}}%
      \csname LTb\endcsname%%
      \put(5422,409){\makebox(0,0){\strut{}\np{e6}}}%
    }%
    \gplgaddtomacro\gplfronttext{%
      \csname LTb\endcsname%%
      \put(126,2889){\rotatebox{-270}{\makebox(0,0){\strut{}Cross-section (b)}}}%
      \csname LTb\endcsname%%
      \put(3284,130){\makebox(0,0){\strut{}Energy (eV)}}%
      \csname LTb\endcsname%%
      \put(3284,5462){\makebox(0,0){\strut{}Hydrogen \tapi{1}H}}%
      \csname LTb\endcsname%%
      \put(5316,5016){\makebox(0,0)[r]{\strut{}Total}}%
      \csname LTb\endcsname%%
      \put(5316,4830){\makebox(0,0)[r]{\strut{}Elastic}}%
      \csname LTb\endcsname%%
      \put(5316,4644){\makebox(0,0)[r]{\strut{}Capture}}%
    }%
    \gplgaddtomacro\gplbacktext{%
      \csname LTb\endcsname%%
      \put(5748,977){\makebox(0,0)[r]{\strut{}}}%
      \csname LTb\endcsname%%
      \put(5748,1742){\makebox(0,0)[r]{\strut{}}}%
      \csname LTb\endcsname%%
      \put(5748,2507){\makebox(0,0)[r]{\strut{}}}%
      \csname LTb\endcsname%%
      \put(5748,3271){\makebox(0,0)[r]{\strut{}}}%
      \csname LTb\endcsname%%
      \put(5748,4036){\makebox(0,0)[r]{\strut{}}}%
      \csname LTb\endcsname%%
      \put(5748,4801){\makebox(0,0)[r]{\strut{}}}%
      \csname LTb\endcsname%%
      \put(5850,409){\makebox(0,0){\strut{}\np{e-5}}}%
      \csname LTb\endcsname%%
      \put(6502,409){\makebox(0,0){\strut{}\np{e-4}}}%
      \csname LTb\endcsname%%
      \put(7155,409){\makebox(0,0){\strut{}\np{e-3}}}%
      \csname LTb\endcsname%%
      \put(7807,409){\makebox(0,0){\strut{}\np{e-2}}}%
      \csname LTb\endcsname%%
      \put(8460,409){\makebox(0,0){\strut{}\np{0.1}}}%
      \csname LTb\endcsname%%
      \put(9112,409){\makebox(0,0){\strut{}\np{1}}}%
      \csname LTb\endcsname%%
      \put(9765,409){\makebox(0,0){\strut{}\np{10}}}%
      \csname LTb\endcsname%%
      \put(10417,409){\makebox(0,0){\strut{}\np{e2}}}%
      \csname LTb\endcsname%%
      \put(11069,409){\makebox(0,0){\strut{}\np{e3}}}%
      \csname LTb\endcsname%%
      \put(11722,409){\makebox(0,0){\strut{}\np{e4}}}%
      \csname LTb\endcsname%%
      \put(12374,409){\makebox(0,0){\strut{}\np{e5}}}%
      \csname LTb\endcsname%%
      \put(13027,409){\makebox(0,0){\strut{}\np{e6}}}%
      \csname LTb\endcsname%%
      \put(13679,409){\makebox(0,0){\strut{}\np{e7}}}%
    }%
    \gplgaddtomacro\gplfronttext{%
      \csname LTb\endcsname%%
      \put(5748,2889){\rotatebox{-270}{\makebox(0,0){\strut{}}}}%
      \csname LTb\endcsname%%
      \put(9764,130){\makebox(0,0){\strut{}Energy (eV)}}%
      \csname LTb\endcsname%%
      \put(9764,5462){\makebox(0,0){\strut{}Gadolinium \tapi{155}Gd}}%
      \csname LTb\endcsname%%
      \put(13146,5016){\makebox(0,0)[r]{\strut{}Total}}%
      \csname LTb\endcsname%%
      \put(13146,4830){\makebox(0,0)[r]{\strut{}Elastic}}%
      \csname LTb\endcsname%%
      \put(13146,4644){\makebox(0,0)[r]{\strut{}Capture}}%
    }%
    \gplgaddtomacro\gplbacktext{%
      \csname LTb\endcsname%%
      \put(8660,720){\makebox(0,0)[l]{\strut{}\np{1}}}%
      \csname LTb\endcsname%%
      \put(8660,1872){\makebox(0,0)[l]{\strut{}\np{e2}}}%
      \csname LTb\endcsname%%
      \put(8660,3023){\makebox(0,0)[l]{\strut{}\np{e4}}}%
      \csname LTb\endcsname%%
      \put(7786,3209){\makebox(0,0){\strut{}\np{50}}}%
      \csname LTb\endcsname%%
      \put(5994,3209){\makebox(0,0){\strut{}\np{10}}}%
      \csname LTb\endcsname%%
      \put(8558,3209){\makebox(0,0){\strut{}\np{e2}}}%
    }%
    \gplgaddtomacro\gplfronttext{%
      \csname LTb\endcsname%%
      \put(6003,1871){\rotatebox{-270}{\makebox(0,0){\strut{}}}}%
      \csname LTb\endcsname%%
      \put(7276,664){\makebox(0,0){\strut{}}}%
      \csname LTb\endcsname%%
      \put(7276,3116){\makebox(0,0){\strut{}}}%
    }%
    \gplbacktext
    \put(0,0){\includegraphics{pics/neutron_xsec}}%
    \gplfronttext
  \end{picture}%
\endgroup
}
	\caption[Neutron cross-sections on hydrogen and gadolinium-157]%
	{Neutron cross-sections on hydrogen (left) and gadolinium (right) as a function of energy from \refref{Brown:2018jhj}. %
	The dominant component for \tapi{1}H is elastic scattering, despite neutron capture %
	reaches relative high cross-section values for subthermal neutrons ($T < 2.5$\,meV).
	Fast neutrons interact on \tapi{157}Gd via elastic scattering, but for lower energies %
	the main contribution comes from neutron capture cross-section.
	The $1/v$ dependency can be appreciated in both cases.
	The inset on the right plot shows a detail of the resonance region of the cross-section, between 10\,eV and 500\,eV.}
	\label{fig:n_xsec}
\end{figure}


From the considerations above, water is a good moderator for thermalising fast neutrons.
The energy decrease for water is $\xi_{\text{H}_2\text{O}} \simeq 0.93$, %
when the kinetic energy of the neutron is in the range $2.5\,\text{meV} \lesssim T \lesssim 0.1\,\text{MeV}$.
Neutrons that reach subthermal energies enter the $1/v$ regions in which the capture probabilities increases as $T^{-1}$.
Hydrogen presents a relative high cross-section for neutron capture, as it can be seen from~\reffig{fig:n_xsec}: %
the cross-section for a thermal neutron on hydrogen is measured to be $\np{332.6}$~\,mb~\cite{ZERKIN201831}. %
The cross-section of capture on the oxygen nucleus \tapi{16}O is around $0.19$~\,mb, four orders of magnitude smaller %
than the capture on hydrogen, and therefore typically neglected.
The hydrogen capture is followed by the emission of a single $\np{2.2}$\,MeV gamma from the %
de-excitation of the newly-formed deuterium.
The characteristic time for Maxwellian neutrons with energy below 10\,MeV to thermalise and being captured %
by hydrogen in water has been measured to be $(\np{204.8}\pm\np{0.4})$\,\textmu s~\cite{Cokinos:1977zz}.
The typical time and energy of this event require special triggers and dedicated analysis %
to correctly identify the neutron capture on hydrogen in SK.
Since the SK-IV phase, after any primary event above the standard higher energy threshold is detected, 
a time window of 535\,\textmu s is saved with no threshold requirement, so that approximately the 92\,\% %
of neutron capture signals are collected.
The 2.2\,MeV photon produces on average 7$\sim$8 PMT hits, which are difficult to reconstruct accurately.
The PMT hits are expected to happen in a narrow timing distribution and to be anisotropic.
A 10\,ns sliding window is used to look for $\gamma$ candidates with more than 7 hits and less than 50 %
to avoid high energy backgrounds.
With these simple selection criteria, an efficiency of 33.2\,\% is obtained from a simulation of neutron capture events, %
with an expected 4.5 background events per neutrino event.
The background rejection is improved by feeding the simulated data of the selected candidates %
to a neural network, which retains an overall efficiency of 20.5\,\% with \np{0.018} backgrounds per signal event~\cite{Irvine:2014hja}.
The efficiency is found to be highly dependent on the distance travelled by the released neutron, %
due to the fact that knowing the location of the capture significantly reduces the background.
A completely analgous search can be used for the gadolinium phase of SK, %
even though the energy released in the neutron capture process is above the detection threshold, %
the thermal lifetime is shorter, and the expected efficiency is much higher as explained in \refsec{sec:gadolinium}.

%follows an exponential and the time constant is largely studied amongst all the thermalisation in water.
%It was found that neutron thermalisation in water has a time constant of 5$\mu$s [fujino, sumita, shiba].
%Neutrons can be captured by either the hydrogen or the oxygen.
%Free neutron will capture on a hydrogen nucleus, releasing a 2.2 MeV gamma.
%In SK, for instance, this gamma results in about seven photo-electrons, and thus only detectable with $\simeq$20\,\% %
%efficiency.

\subsection{Neutron calibration}

%https://arxiv.org/pdf/0811.0735.pdf
%https://www.sno.phy.queensu.ca/sno/papers/labranche_phd.pdf
%https://www.sno.phy.queensu.ca/sno/papers/lyon_phd.pdf

Neutron tagging efficiency in SK is measured with an americium-beryllium (Am-Be) source~\cite{Watanabe:2008ru}.
In an Am-Be source, \tapi{214}Am decays 100\,\% of the time into \tapi{237}Np via $\alpha$-emission, %
with a half-life of 432.6\,y.
The $\alpha$ particle is captured by a \tapi{9}Be nucleus to become \tapi{12}C\tapi{*} with the emission of a neutron.
The carbon de-excites to the ground state with sometimes the emission of a 4.43\,MeV photon.
This gamma is used to trigger the neutron emission.
To maximise this trigger signal, the Am-Be source is placed at the centre of a 5\,cm cube of %
bismuth germanite (BGO) crystal scintillator which is lowered inside the water Cherenkov tank during calibration runs.
The gamma ray from the beryllium neutron capture is promptly absorbed by the BGO crystal, %
with the release of intense scintillation light that triggers the SK detector. 
Around a thousands of photoelectrons are typically observed by the PMTs from the BGO emission; %
this signal triggers the search for a neutron capture on hydrogen.
%Forced triggers were imposed during the SK-III phase to extend the acquisition time window.
The candidates are selected by an analysis similar to the simulation of \refref{Irvine:2014hja} and outlined in \refsec{sec:neutron_tagging}.
Neutrons from beryllium have energies below 10\,MeV, as seen in \reffig{fig:spectra}, %
less than the typical neutron energy resulting from atmospheric neutrino interactions.
For this reason, in the calibration analysis the neutron capture vertex is assumed to be roughly at the same location %
of the Am-Be source.
The efficiency measured in \refref{Watanabe:2008ru} ranges from 13.1\,\% to 24.5\,\%, %
the exact value of which is position dependent.
These values agree overall with the simulation analysis.

\begin{figure}
	\centering
	\resizebox{0.6\textwidth}{!}{\input{pics/source_spectrum.tex}}
	\captionof{figure}[Normalised energy distribution of Am-Be and \tapi{252}Cf sources]%
	{Normalised energy distribution of neutrons emitted by an Am-Be source~\cite{PMID:4744412} %
	and neutrons and photons from the spontaneous fission of a \tapi{252}Cf source~\cite{PhysRev.104.699, PhysRev.108.411}.
	The Am-Be emission is more uniform over the energy range, compared to \tapi{252}Cf which %
	peaks at around 1\,MeV.}
	\label{fig:spectra}
\end{figure}


\begin{figure}
	\centering
	\includegraphics[width=0.3\linewidth]{pics/device.png}
	\hspace{1em}
	\raisebox{2.8em}{\includegraphics[width=0.3\linewidth]{pics/device_in.png}}
	\captionof{figure}[Setup used to test a californium source]%
	{Setup used to test the californium source.
		The plastic scintillator cylinder is encased in an aluminium structure %
		connected to the PMT support (black).
		The scintillator is optically coupled to the PMT.}
	\label{fig:setup}
\end{figure}

Another possibility as a source for neutron calibration is using californium-252, which %
undergoes $\alpha$-emissions (96.91\%) or spontaneous fission (SF, 3.09\%).
Thanks to a shorter half-life of 2.645\,y, a \tapi{252}Cf source presents a higher activity compared to an Am-Be source %
with the same number of nucleons.
Furthermore, the SF process emits an average of 3.75 neutrons per fission and an average of 10.3 photons %
summing up to a total energy of 8.2\,Mev~\cite{PhysRev.104.699}.
As for the case of the Am-Be calibrating device, the photons can be collected by a scintillating material %
and tag the emission of neutrons.
After the trigger signal, multiple neutron captures on hydrogen are expected, separated by short intervals in time %
of the order of milliseconds.
The yield of multiple neutrons is an advantage which the SNO collaboration exploited with a method %
called \emph{Time Series Analysis}.
Multiplicity and time intervals between the detected events can be used to determine the neutron detection efficiency, %
the neutron mean life inside the detector, and activity from nonfission events~\cite{Labranche:2004sya}.
Differently from the Am-Be source, the activity of which must be known precisely, the neutron tagging calibration %
with a californium source can be done in principle regardless of that information.
The Time Series Analysis could be implemented also in SK if californium was used as a calibrating source.
The fast-neutron and photon energy spectra from \tapi{252}Cf are shown in \reffig{fig:spectra}. 
The neutrons are emitted with a most-probable energy of 1\,MeV and an average energy of 2.1\,MeV.
It is reasonable to assume that the capture on hydrogen occurs in the proximity of the source. %
Using a californium source would bring about an even more accurate calibration than with the Am-Be source, %
being the location of the capture an important semplification in the calibration measurement.	% ???


\begin{figure}
	\begin{minipage}[t]{0.48\textwidth}
		\centering
		\resizebox{\textwidth}{!}{\input{pics/peaksources.tex}}
		\captionof{figure}[Distribution of PMT peaks measured with a californium source]%
		{Distribution of PMT peaks measured with source inside (black) and source removed from the scintillator~(red).}
		\label{fig:source}
	\end{minipage}
	\hfill
	\begin{minipage}[t]{0.48\textwidth}
		\centering
		\resizebox{\textwidth}{!}{\input{pics/QE.tex}}
		\captionof{figure}[Distribution of optical photons from the simulation of a neutron calibration device]{%
			Distributions showing PMT quantum efficiency (red), the scintillator yield (blue), %
			their correlation (magenta), and the photons collected by the PMT in the GEANT4 %
			simulation of optical photons (black).}
		\label{fig:QE}
	\end{minipage}
\end{figure}

Some preliminary studies were performed to develop a device for neutron calibration %
with californium of a generic water Cherenkov detector.
For laboratory measurements, a \tapi{252}Cf source with an activity of 4.3\,kBq at the time of its production was used.
%measured on the 28\tapi{th} of July, 2017.
The californium is encapsulated in a double-hull stainless steel cylinder, 9.5\,mm in diameter and 37.5\,mm high.
The source is placed in a simple prototype of the device, shown in \reffig{fig:setup}, %
consisting of a cylindrical plastic scintillator (EJ-200), coated with mylar to contain optical photons, %
and optically coupled to an ET Enterprise 9902B series~PMT.
A hole, coaxial to the plastic cylinder, allows to insert the source in the middle of the scintillator.
The cylinder is 3'' in length and 1.5'' in diameter.
The rate of photons emitted by the \tapi{252}Cf source is measured with this setup; %
the signal from the PMT is amplified and cleaned by NIM modules and finally recorded with a 14-bit VME digitiser.
%to count the number of photon triggers from the PMT and %
%a daisy-chain of NIM amplifier, threshold, and discriminator is used before the digitiser to optimise the signal.
Without the source inside the scintillator, a dark rate of 1.633\,Hz is measured, %
whereas with the source the rate increases to 41.395\,Hz.
The distributions of the PMT peaks collected by the digitiser are shown in \reffig{fig:source}.
A predicted activity of 3.1\,kBq is expected on the day of the measurement---446 days after the production of the source--- %
which translates to a SF rate of 96.46\,Hz.
The SF tagging efficiency with this setup is therefore estimated to be around~41\,\%.

\begin{figure}
	\centering
	\resizebox{\textwidth}{!}{% GNUPLOT: LaTeX picture with Postscript
\begingroup
  \makeatletter
  \providecommand\color[2][]{%
    \GenericError{(gnuplot) \space\space\space\@spaces}{%
      Package color not loaded in conjunction with
      terminal option `colourtext'%
    }{See the gnuplot documentation for explanation.%
    }{Either use 'blacktext' in gnuplot or load the package
      color.sty in LaTeX.}%
    \renewcommand\color[2][]{}%
  }%
  \providecommand\includegraphics[2][]{%
    \GenericError{(gnuplot) \space\space\space\@spaces}{%
      Package graphicx or graphics not loaded%
    }{See the gnuplot documentation for explanation.%
    }{The gnuplot epslatex terminal needs graphicx.sty or graphics.sty.}%
    \renewcommand\includegraphics[2][]{}%
  }%
  \providecommand\rotatebox[2]{#2}%
  \@ifundefined{ifGPcolor}{%
    \newif\ifGPcolor
    \GPcolortrue
  }{}%
  \@ifundefined{ifGPblacktext}{%
    \newif\ifGPblacktext
    \GPblacktexttrue
  }{}%
  % define a \g@addto@macro without @ in the name:
  \let\gplgaddtomacro\g@addto@macro
  % define empty templates for all commands taking text:
  \gdef\gplbacktext{}%
  \gdef\gplfronttext{}%
  \makeatother
  \ifGPblacktext
    % no textcolor at all
    \def\colorrgb#1{}%
    \def\colorgray#1{}%
  \else
    % gray or color?
    \ifGPcolor
      \def\colorrgb#1{\color[rgb]{#1}}%
      \def\colorgray#1{\color[gray]{#1}}%
      \expandafter\def\csname LTw\endcsname{\color{white}}%
      \expandafter\def\csname LTb\endcsname{\color{black}}%
      \expandafter\def\csname LTa\endcsname{\color{black}}%
      \expandafter\def\csname LT0\endcsname{\color[rgb]{1,0,0}}%
      \expandafter\def\csname LT1\endcsname{\color[rgb]{0,1,0}}%
      \expandafter\def\csname LT2\endcsname{\color[rgb]{0,0,1}}%
      \expandafter\def\csname LT3\endcsname{\color[rgb]{1,0,1}}%
      \expandafter\def\csname LT4\endcsname{\color[rgb]{0,1,1}}%
      \expandafter\def\csname LT5\endcsname{\color[rgb]{1,1,0}}%
      \expandafter\def\csname LT6\endcsname{\color[rgb]{0,0,0}}%
      \expandafter\def\csname LT7\endcsname{\color[rgb]{1,0.3,0}}%
      \expandafter\def\csname LT8\endcsname{\color[rgb]{0.5,0.5,0.5}}%
    \else
      % gray
      \def\colorrgb#1{\color{black}}%
      \def\colorgray#1{\color[gray]{#1}}%
      \expandafter\def\csname LTw\endcsname{\color{white}}%
      \expandafter\def\csname LTb\endcsname{\color{black}}%
      \expandafter\def\csname LTa\endcsname{\color{black}}%
      \expandafter\def\csname LT0\endcsname{\color{black}}%
      \expandafter\def\csname LT1\endcsname{\color{black}}%
      \expandafter\def\csname LT2\endcsname{\color{black}}%
      \expandafter\def\csname LT3\endcsname{\color{black}}%
      \expandafter\def\csname LT4\endcsname{\color{black}}%
      \expandafter\def\csname LT5\endcsname{\color{black}}%
      \expandafter\def\csname LT6\endcsname{\color{black}}%
      \expandafter\def\csname LT7\endcsname{\color{black}}%
      \expandafter\def\csname LT8\endcsname{\color{black}}%
    \fi
  \fi
    \setlength{\unitlength}{0.0500bp}%
    \ifx\gptboxheight\undefined%
      \newlength{\gptboxheight}%
      \newlength{\gptboxwidth}%
      \newsavebox{\gptboxtext}%
    \fi%
    \setlength{\fboxrule}{0.5pt}%
    \setlength{\fboxsep}{1pt}%
\begin{picture}(14400.00,7200.00)%
    \gplgaddtomacro\gplbacktext{%
      \csname LTb\endcsname%%
      \put(618,540){\makebox(0,0)[r]{\strut{}0}}%
      \csname LTb\endcsname%%
      \put(618,1726){\makebox(0,0)[r]{\strut{}0.2}}%
      \csname LTb\endcsname%%
      \put(618,2912){\makebox(0,0)[r]{\strut{}0.4}}%
      \csname LTb\endcsname%%
      \put(618,4098){\makebox(0,0)[r]{\strut{}0.6}}%
      \csname LTb\endcsname%%
      \put(618,5284){\makebox(0,0)[r]{\strut{}0.8}}%
      \csname LTb\endcsname%%
      \put(618,6470){\makebox(0,0)[r]{\strut{}1}}%
      \csname LTb\endcsname%%
      \put(720,354){\makebox(0,0){\strut{}$0$}}%
      \csname LTb\endcsname%%
      \put(1570,354){\makebox(0,0){\strut{}$5$}}%
      \csname LTb\endcsname%%
      \put(2421,354){\makebox(0,0){\strut{}$10$}}%
      \csname LTb\endcsname%%
      \put(3271,354){\makebox(0,0){\strut{}$15$}}%
      \csname LTb\endcsname%%
      \put(4122,354){\makebox(0,0){\strut{}$20$}}%
      \csname LTb\endcsname%%
      \put(4972,354){\makebox(0,0){\strut{}$25$}}%
      \csname LTb\endcsname%%
      \put(5822,354){\makebox(0,0){\strut{}$30$}}%
      \csname LTb\endcsname%%
      \put(6673,354){\makebox(0,0){\strut{}$35$}}%
    }%
    \gplgaddtomacro\gplfronttext{%
      \csname LTb\endcsname%%
      \put(126,3653){\rotatebox{-270}{\makebox(0,0){\strut{}Average fraction of absorbed energy (\%)}}}%
      \csname LTb\endcsname%%
      \put(4121,75){\makebox(0,0){\strut{}Size (cm)}}%
      \csname LTb\endcsname%%
      \put(4121,7046){\makebox(0,0){\strut{}BGO scintillator}}%
      \csname LTb\endcsname%%
      \put(6990,1451){\makebox(0,0)[r]{\strut{}Cube}}%
      \csname LTb\endcsname%%
      \put(6990,1265){\makebox(0,0)[r]{\strut{}Cylinder}}%
      \csname LTb\endcsname%%
      \put(6990,1079){\makebox(0,0)[r]{\strut{}Rod}}%
      \csname LTb\endcsname%%
      \put(6990,893){\makebox(0,0)[r]{\strut{}$\gamma$}}%
      \csname LTb\endcsname%%
      \put(6990,707){\makebox(0,0)[r]{\strut{}$n$}}%
    }%
    \gplgaddtomacro\gplbacktext{%
      \csname LTb\endcsname%%
      \put(7421,540){\makebox(0,0)[r]{\strut{}}}%
      \csname LTb\endcsname%%
      \put(7421,1726){\makebox(0,0)[r]{\strut{}}}%
      \csname LTb\endcsname%%
      \put(7421,2912){\makebox(0,0)[r]{\strut{}}}%
      \csname LTb\endcsname%%
      \put(7421,4098){\makebox(0,0)[r]{\strut{}}}%
      \csname LTb\endcsname%%
      \put(7421,5284){\makebox(0,0)[r]{\strut{}}}%
      \csname LTb\endcsname%%
      \put(7421,6470){\makebox(0,0)[r]{\strut{}}}%
      \csname LTb\endcsname%%
      \put(7523,354){\makebox(0,0){\strut{}$0$}}%
      \csname LTb\endcsname%%
      \put(8374,354){\makebox(0,0){\strut{}$5$}}%
      \csname LTb\endcsname%%
      \put(9224,354){\makebox(0,0){\strut{}$10$}}%
      \csname LTb\endcsname%%
      \put(10075,354){\makebox(0,0){\strut{}$15$}}%
      \csname LTb\endcsname%%
      \put(10925,354){\makebox(0,0){\strut{}$20$}}%
      \csname LTb\endcsname%%
      \put(11776,354){\makebox(0,0){\strut{}$25$}}%
      \csname LTb\endcsname%%
      \put(12626,354){\makebox(0,0){\strut{}$30$}}%
      \csname LTb\endcsname%%
      \put(13477,354){\makebox(0,0){\strut{}$35$}}%
      \csname LTb\endcsname%%
      \put(14327,354){\makebox(0,0){\strut{}$40$}}%
    }%
    \gplgaddtomacro\gplfronttext{%
      \csname LTb\endcsname%%
      \put(10925,75){\makebox(0,0){\strut{}Size (cm)}}%
      \csname LTb\endcsname%%
      \put(10925,7046){\makebox(0,0){\strut{}Plastic scintillator}}%
      \csname LTb\endcsname%%
      \put(13794,1451){\makebox(0,0)[r]{\strut{}Cube}}%
      \csname LTb\endcsname%%
      \put(13794,1265){\makebox(0,0)[r]{\strut{}Cylinder}}%
      \csname LTb\endcsname%%
      \put(13794,1079){\makebox(0,0)[r]{\strut{}Rod}}%
      \csname LTb\endcsname%%
      \put(13794,893){\makebox(0,0)[r]{\strut{}$\gamma$}}%
      \csname LTb\endcsname%%
      \put(13794,707){\makebox(0,0)[r]{\strut{}$n$}}%
    }%
    \gplbacktext
    \put(0,0){\includegraphics{n_vs_gamma_2}}%
    \gplfronttext
  \end{picture}%
\endgroup
}
	\caption[Performance of different scintillators from the simulation of a neutron calibration device]%
	{Result of the GEANT4 simulation (points), showing the expected performance of different scintillators %
		as a function of the characteristic size.
		The lines are a smooth interpolation between simulation points.
		The fraction of absorbed energy per initial energy of neutrons (solid) and photons (dashed) %
		is averaged and plotted against the scintillator size.
		Three shapes are tested: a cube (green), a cylinder with diameter-height-ratio of 1:1 (blue), %
		and a cylinder with ratio 1:2 (red).
		On the left, the scintillating material is BGO crystal; on the right, %
		a generic vinyltouluene plastic is used.}
	\label{fig:geant4}
\end{figure}



A GEANT4~\cite{Agostinelli:2002hh} simulation of the setup was performed with the aim of optimising the calibration instrument.
An ideal device would absorb all the photons converting them into visible light without affecting the neutrons.
The plot in \reffig{fig:QE} shows the correct implementation of the scintillator and PMT efficiencies in the simulation: %
there is good agreement with the MC distribution and the expected optical photon spectrum.
Different volumes and materials are tested for the scintillator in the simulation.
As far as materials are concerned, BGO crystal and a generic vinyltouluene are chosen.
The selected shapes for the volume are a cube and cylinders with a diameter-height-ratio 1:1 and 1:2.
The latter shape models the prototype tested in the laboratory.
The characteristic size, i.e.\ the side for the cube and the height for the cylinders, is varied from 1\,cm to 40\,cm.
The simulation tracks the energy deposited in the scintillator by the photons and the number of neutrons escaping the device, %
from a simulation of \np{10000} SF events.
The result is shown in Fig.~\ref{fig:geant4}, where for both photons and neutrons %
the average value of the fraction of absorbed energy with respect to the initial one is plotted against the size.
In terms of materials, the BGO crystal performs better than the plastic scintillator as expected, %
absorbing almost the entirety of photons and leaving the neutrons mainly unaffected.
Apart from not being as efficient as scintillator, the hydrogen in the plastic thermalises the neutrons more than BGO.
This would impact the capture time in a water Cherenkov detector.
In terms of sizes and shapes, a BGO cube of $4\sim6$\,cm sides seems to maximise photon absorption and minimise %
the energy loss of neutrons.
This is in line with the choice for the calibrating device for SK, which is a 5\,cm BGO cube.
Both the cylindrical shapes are optimal when the height is $7\sim11$\,cm, but more crystal would be required %
increasing the cost of the device.
As far as the plastic scintillator is concerned, it is more difficult to define an optimal size/shape figure:
a cubic scintillator is more efficient in collecting photons, but being the form with the largest volume %
per given size, it is also more effective in slowing down neutrons.
The two cylinder shapes have similar performances, proportionally to their volumes.

From the simulation studies, the current design used with the Am-Be source seems to be ideal %
even for \tapi{252}Cf.
Simply replacing the neutron source in SK and adopting the Time Series Analysis could give a more accurate calibration.
It is important to realise that the same calibrating procedure and device can be equally applied %
to the gadolinium phase of SK and Hyper-Kamiokande in the future.
Further R\&D is anyway needed to build an optimal device for neutron calibration with a californium source.

%To verify neutron tagging efficiency given above, experimental tests were conducted with an Am/Be source embedded in a bismuth germanite (BGO) scintillator. The prompt and delayed event-pair is generated via: α + 9Be →12
%C∗+n; 12C∗ →12 C+γ(prompt); n+p → d+γ(delayed).
%The scintillation light induced by 4.43 MeV deexcitation γ
%serves as the primary event. Note that the reaction to the
%ground state of 12C also exists, where no 4.43 MeV deexcitation γ is emitted. The experimental apparatus was
%deployed at the center of the tank, during which the trigger
%gate to catch 2.2 MeV γ was temporarily enlarged to 800
%μs in order to obtain a complete neutron capture time spectrum. To estimate source related background (e.g. ground
%transition neutron), 10 Hz 800μs random trigger data was
%also taken.
%The final N10 distribution after all cuts applied is shown in
%Fig. 2, where for Am/Be data all backgrounds are subtracted according to random trigger data. Signal efficiencies for
%MC and data are (19.2±0.1)% and (19.0±0.2)%, respectively. Data is in good agreement with MC. Fig. 3 shows
%the distribution of time difference (ΔT) between delayed
%neutron signal and prompt event. The neutron lifetime in
%pure water is measured to be (201.8 ± 4.7)μs using a unbinned maximum likelihood fitting as shown in Fig. 3.
%
%The Am-Be source is embedded in a 5 cm cube of BGO scintillator (see figure 8.27)
%to amplify the light released by the 4.4 MeV γ-ray, such that it will activate the SHE
%trigger in the detector. Upon triggering SHE stores 35µs of data, and then an extended
%AFT trigger is activated to store an additional 800µs, and grant a more complete view
%the neutron capture time spectrum.
%Figure 8.27: Am-Be crystal embedded in a 5 cm cube of BGO scintillator. This is
%held in an acrylic case.
%This configuration was set up in 3 different locations around the tank: the centre (35.3,
%-70.7, 0) cm (Centre), near the side of the barrel (35.3, -1201.9, 0) cm (Y12), and near
%the top of the tank (35.3, -70.7, 1500.0) cm (Z15). Random data was also taken with
%the apparatus in the centre, using a 10Hz trigger, to study the irreducible background
%from the ground-state transition.
%
%
%Neutron tagging is then performed on the remaining Am-Be dataset, and a 2.2 MeV
%γ-ray MC. There are some differences between this study and the atmospheric neutrino study discussed up until now. Neutrons released by Beryllium have an energy
%ranging from 2-10 MeV, much less than the average neutron energy following atmospheric neutrino interactions. Thus, we can make the assumption that the location of
%the Am-Be apparatus is roughly the same as the neutron capture vertex. This raises
%the expected neutron detection efficiency in accordance with figure 8.23. In addition,
%Chapter 8. Neutron Tagging 140
%the PMT after-pulse observed following high energy atmospheric neutrino events (figure 8.5) is not produced to the same extent after Am-Be events, so neutron searching
%is started at 5µs (previously > 18µs) after the primary trigger. The AFT trigger for
%Am-Be is also extended, allowing neutron-searching until 835µs after the initial trigger.
%The efficiency is calculated by fitting the timing distribution of neutron candidates to a
%constant background + exponentially decaying signal representing the neutron capture
%lifetime.




\section{Gadolinium neutron capture}
\label{sec:gadolinium}

%In the next phase of the Super-Kamiokande experiment, a gadolinium salt compound Gd\tped{2}(SO\tped{4})\tped{3} %
%will be added to the detector~\cite{}, which will improve the ability of the detector to identify neutrons.
%Neutrons are emitted in inverse beta decay (IBD) process, $\cj{\nu}_e + p \to e^+ + n$.
%The current technique to tag neutrons in SK is by detecting capture on hydrogen nuclei.
%Capture of a thermal neutron on hydrogen has a cross-section of $\np{332.6}\pm\np{0.7}$~\,b, %
%characteristic time of $\sim200$\,\textmu s in water, with the release of $\np{2.2}$\,MeV gammas.
%After the prompt Cherenkov radiation from the charged lepton, the secondary signal from an electron, %
%compton-scattered by the 2.2\,MeV photon, is looked for in a long time window.
%However, in water what matters are the electrons Compton-scattered above the Cerenkov threshold by relatively hard gammas.
%
%The detectable light following neutron capture on Gd (in thin foils) in possible discrete counters was carefully simulated %
%for the SNO heavy water Cerenkov detector.
%The equivalent single electron energy was found to peak at about 5 MeV, and range over 3--8\,MeV.
%This spread reflects the intrinsic variation in the gamma cascades and the detector energy
%resolution (the simulation had 5 photoelectrons per MeV of electron energy, compared to 6 in SK-I)~\cite{Beacom:2003nk}.
%The energy threshold for SK is 5\,MeV, and the search of hydrogen-tagged neutrons is not efficient (20\%).
%a long time window in which to look for the hydrogen signal, 
%Capture on gadolinium has a cross-section of $\np{49e3}$~b, which de-excites emitting a $\sim\np{8}$~MeV gamma-ray cascade.
%By adding 0.2\,\% by mass of the Gd salt as Gd\tped{2}(SO\tped{4})\tped{3}, the characteristic time of capture is %
%$\sim30$~\textmu s and an efficiency of 90\,\% can be achieved.
%The isotopes \tapi{155}Gd and \tapi{157}Gd present very high cross-sections to thermal neutron capture, %
%respectively \np{6.074e4}\,b and \np{2.537e5}\,b.
%Hydrogen alone has a cross-section of \np{3.321e-1}\,b.


Gadolinium-157 has the highest thermal neutron capture cross-section among any stable nuclides.
It is estimated to be \np{2.537E5}\,b and its natural abundance is around 15.65\,\%.
Another isotope of gadolinium with similar abundance is \tapi{155}Gd, at 14.80\,\%, %
which also presents a very high neutron capture cross-section \np{6.074E4}\,b.
Dissolving gadolinium compounds in water could therefore considerably increase the neutron capture %
probability, as proposed for the first time in \refref{Beacom:2003nk}.
Gd-doped water enhances the capture cross-section, with an effective cross-section of \np{49e3}\,b for a 0.2\,\% concentration, %
compared to pure water with $\sim$0.3\,b on a free proton.
Upon capturing a neutron, the Gd nucleus emits three to four gamma rays having a total energy of about 8\,MeV.
Such energetic photons can produce Cherenkov light via Compton-scattering and therefore they can be 
reliably detected in a large detector volume.
The neutron in gadolinium-loaded water thermalises more quickly than in just pure water, with a %
characteristic time of $\sim$30\,\textmu s, and it can be captured by a Gd nucleus with an %
estimated efficiency of 90\,\%~\cite{Beacom:2003nk} for a 0.2\,\% Gd solution (see \reffig{fig:gd_conc}).
%The time and spatial correlation of the positron and neutron capture events ($20~\mu$s and 4~cm) %
%can significantly reduce the backgrounds, and hence enhance the nu e signal events.
%Even moderately energetic neutrons ranging from tens to hundreds of MeV will quickly lose energy %
%by collisions with free protons and oxygen nuclei in water. 
%Once thermalised, the neutrons undergo radiative capture, combining with a nearby nucleus to %
%produce a more tightly bound final state, with excess energy released in a gamma-ray ( ) cascade. 

%\iffalse
\begin{figure}
	\begin{minipage}[t]{0.48\textwidth}
		\centering
		\resizebox{\linewidth}{!}{\input{pics/gdconcentration.tex}}
		\captionof{figure}[Dependency of neutron capture efficiency with respect to gadolinium concentration]%
		{Estimated dependency of neutron capture efficiency on gadolinium with respect to the gadolinium %
		concentration in water. Super-Kamiokande will start with a 0.02\,\% Gd-load, %
		increasing up to 0.2\,\% to reach an expected $\sim$90\,\% capture efficiency.}
		\label{fig:gd_conc}
	\end{minipage}
	\hfill
	\begin{minipage}[t]{0.5\textwidth}
		\centering
		\raisebox{0.8em}{\includegraphics[width=\linewidth]{pics/bandpass.pdf}}
		\captionof{figure}[Schematic of the band-pass filtration system and the fast recirculation loop in EGADS]%
		{Schematic of the band-pass filtration system and the fast recirculation loop in EGADS, %
		where the main components of the system are highlighted: anion-exchange resins (AE), %
		ultraviolet lamp (UV), total organic compound lamp (TOC), deionising system (DI), %
		reverse osmosis stage (RO). See text and \refref{Ikeda:2019pcm} for details. }
		\label{fig:egads}
	\end{minipage}
\end{figure}
%\fi


In the next phase of the Super-Kamiokande experiment, called SK-Gd, a gadolinium salt compound %
will be dissolved in the detector to improve the ability to identify neutrons.
The gadolinium phase is supposed to start in late 2020.
This will be possible thanks to extensive R\&D performed by the EGADS experiment~\cite{Ikeda:2019pcm}.
In \refref{Beacom:2003nk}, gadolinium trichloride GdCl\tped{3} was proposed, %
but the current full-scale plan for SK and EGADS has settled on using gadolinium sulphate Gd\tped{2}(SO\tped{4})\tped{3}.
%---another candidate being 
This choice was determined by a few requirements by which a Gd compound candidate must comply.
In addition to the aforementioned salts, a third option could have been gadolinium nitrate Gd(NO\tped{3})\tped{3}.
The candidate compound must be water soluble, also in large amounts, but all of the three gadolinium salts %
easily dissolve in water.
%The above three candidates can all be dissolved fairly easily, with the GdCl3 and Gd(NO3)3 only
%needing stirring to fully dissolve, while the Gd2(SO4)3, can be forced into solution with the addition of %
%a small amount of sulfuric acid, about 380 ml of acid for 28 kg of Gd2(SO4)3 in 14 tons of water (test done at EGADS).
%So, this solubility requirement does not immediately rule out any of these three compounds.
If used in very large quantities, as it will be for SK-Gd, the compound must be safe for %
%Next, if the compound is to be used inside a very large water Cherenkov detector, such as SK, it must be safe for
the detectors components and it should be nontoxic. %it may be easily put into existing detectors.
None of the above salts are toxic, but GdCl\tped{3} is corrosive and not suitable for a full scale test.
Soak tests in a Gd solution at $25^\circ$C showed that the rubber friction pads used to hold the inner detector PMTs %
are susceptible to the sulphate, even though they seem to be affected also by pure water.
In over twenty years of operation of SK, the effect has never been noticed, meaning the filtration system %
removes the impurities from this material with great success.
Taking in account the different scales between the detector and the test and that the typical temperature %
of the water in SK is around $13^\circ$C, the case was deemed not to be a potential problem.
%they show different corrosive properties, due to the different anions of the molecule.
%The nitrate and the sulfate turn out to be non-corrosive, and do not seem to affect detectors components much, if at all.
%An extensive soak test study was carried out in Japan, with each of the 31 different materials inside the SK detector %
%being soaked in both pure water and Gd2(SO4)3 solution.
%Only the which it is in the case of SK.
%However, the chloride is corrosive, so for this reason the GdCl3 has been ruled out as a full scale test candidate.
It is also necessary that the gadolinium solution maintains a high level of optical transparency, %
so that optical photons of Cherenkov radiation can propagate inside the tank without attenuation.
The Gd(NO\tped{3})\tped{3} is opaque in the UV region of the electromagnetic spectrum, for wavelengths less than 350\,nm, %
and this unfortunately is where a large fraction of Cherenkov light is detected by SK PMTs, %
so it must be rejected as a good full scale candidate.
This leaves Gd2(SO4)3 as the only choice possible, having good transparency in the UV and optical regions.
%This is in fact the choice that has been made by the EGADS group, and the remaining studies described in this paper %
%have been done using, or under the assumption that, Gd2(SO4)3 will be used.
%One last exceptionally nice quality about this compound is today it costs only 5 US dollars per kilogram, %
%meaning it would only be about 500,000 US dollars to dope SK, a price which is on the scale of such a major %
%experiment~\cite{1201.1017}[conference].

The most challenging aspect of a gadolinium-loaded water Cherenkov detector is the filtration system.
The SK water purification system produces ultra pure water, close to the theoretical maximum, %
with a resistivity around $18$\,M$\Omega\cdot$cm.
This high purity is achieved thanks to several filtration stages, including microfilters, %
ultrafilters discarding particles with a size less than 0.1\,\textmu m, UV lamps for bacterial growth, %
reverse osmosis, vacuum and membrane degasifier, and anion-exchange resins.
The current process, without modifications, would completely remove dissolved gadolinium.
Therefore, in order to keep good water quality and high transparency, it is necessary to adapt the purification system %
such that it can remove all impurities, ions included, except Gd\tapi{3+} and its anionic partner (SO\tped{4})\tapi{2-}.
A new filtration system, called ``band-pass filtration system'', has been devised for EGADS and runs in parallel to %
the fast-recirculation loop~\cite{Ikeda:2019pcm}.
The schematic is illustrated in \reffig{fig:egads}.
The water from the water Cherenkov tank is first cleaned by microfilters and UV light.
The Gd solution is then cooled down, as the sulphate dissolves better at lower temperatures, %
before passing through an ultafilter.
At this stage, a series of nanofilters splits up the water line into Gd-enriched water and %
water with less than one part-per-million of gadolinium.
The concentrated line goes directly back to a buffer collection tank, whereas the gadolinium-free line %
is first cleaned by deionisation and reverse osmosis.
In the collection tank, dissolved air is removed by a membrane degasifier and resins are used %
for the final purification.
After full loading up to a concentration of 0.2\,\% of Gd, the water transparency in EGADS is found to be within typical SK values %
thanks to the water filtration system which can maintain good water quality and minimise gadolinium losses.

The water system for the gadolinium phase of SK will be essentially a scaled version of the EGADS filtration system.
During the first stage, a solution of 0.02\,\% in Gd\tped{2}(SO\tped{4})\tped{3} mass will be loaded in SK, %
to progressively increase it up to 0.2\,\% after the first commissioning.
When the experiment will be decommissioned in the future, the gadolinium will be extracted from the water system %
to avoid dispersing it in the environment.
This process was also tested successfully by the EGADS experiment.

\section{Monitoring gadolinium concentration}
\label{sec:gad}

The concentration of Gd in water affects the efficiency and timing of neutron captures, %
which must be known for an accurate measurement of the antineutrino rate.
It is fundamental, therefore, to measure the concentration regularly, as this can change in time inside the detector.
On large scales like for SK-Gd, it is not a trivial task to predict the temperature and the flow dependency %
of the dissolved Gd.
Currently, the EGADS experiment keeps track of the Gd concentration with a Zeeman atomic absorption spectrometer (AAS), %
located near the site of the experiment.
Water samples are collected monthly from the water tank, diluted and atomised inside the AAS machine.
The amount of Gd is then compared to known samples of Gd loaded water to determine the concentration.
This method reaches currently an accuracy of $\sim3$\,\%.

An alternative method for monitoring gadolinium concentration is under study and proposed here.
The new concept still uses atomic absorption lines of Gd, but in solution in water, %
allowing for a more frequent measurement.
Gadolinium presents strong emission/absorption lines in the UV region~\cite{Morton_2000}.
Using a UV source and a spectrometer it is possible to measure the absorption by gadolinium dissolved in water.
The absorbance is directly proportional to the amount of gadolinium and therefore to the concentration, %
where the absorbance is defined as
\begin{equation}
	\label{eq:abs}
	\mathcal{A}(x) = \log_{10} \frac{I_0 (x)}{I_\text{Gd}(x)}\ .
\end{equation}
The quantity $I_0$ is the intensity of a luminous source at a wavelength $x$ filtering through %
a reference sample of pure water, whereas $I_\text{Gd}$ is the intensity of the same source at the same wavelength %
for a gadolinium loaded water sample.
%The two intesnities spectra are then compared at a given wavelength.
The gadolinium absorption spectrum presents a series of lines in the region between 270\,nm and 275\,nm and
the height of each peak is related to the Gd concentration, thanks to the exponential law %
%as the intensity of light surviving the sample %
%follows
\begin{equation}
	\label{eq:expo}
	I_\text{Gd}(x) = I_0(x) e^{-\flatfrac{\ell}{\lambda}}\ ,
\end{equation}
with $\lambda$ the attenuation length and $\ell$ the length of the sample.
For a fixed cross-section $S$ and a fixed length, the attenuation length will %
decrease with the absolute quantity of Gd dissolved in the solution, $m_\text{Gd}$, or rather with the concentration $\rho_\text{Gd}$, %
since the volume is constant between the pure water and Gd-loaded water measurements:
\begin{equation}
	\rho_\text{Gd} = \frac{m_\text{Gd}}{S\,\ell}\ .
\end{equation}
The Beer-Lambert law relates the optical attenuation of a physical material, containing a single attenuating species %
of uniform concentration, to the optical path length via a material-specific constant~\cite{Beer_1852}
\begin{equation}
	\label{eq:beerlambert}
	\mathcal{A} = \varepsilon\, \ell\, \rho \ ,
\end{equation}
where $\varepsilon$ is the molar attenuation coefficient, or \emph{absorptivity} of the attenuating species.
By comparing \refeq{eq:beerlambert} with \refeqs{eq:abs}{eq:expo}, the attenuation length $\lambda$ can be related %
to concentration and absorptivity
\begin{equation}
	\lambda = \frac{\log_{10} (e)}{\varepsilon\, \rho }\ .
\end{equation}
The Beer-Lambert law agrees with the expectation of decreasing attenuation length with increasing concentration.

\begin{figure}
	\centering
	\resizebox{0.49\linewidth}{!}{\input{pics/shimadzu_20.tex}}
	\resizebox{0.49\linewidth}{!}{% GNUPLOT: LaTeX picture with Postscript
\begingroup
  \makeatletter
  \providecommand\color[2][]{%
    \GenericError{(gnuplot) \space\space\space\@spaces}{%
      Package color not loaded in conjunction with
      terminal option `colourtext'%
    }{See the gnuplot documentation for explanation.%
    }{Either use 'blacktext' in gnuplot or load the package
      color.sty in LaTeX.}%
    \renewcommand\color[2][]{}%
  }%
  \providecommand\includegraphics[2][]{%
    \GenericError{(gnuplot) \space\space\space\@spaces}{%
      Package graphicx or graphics not loaded%
    }{See the gnuplot documentation for explanation.%
    }{The gnuplot epslatex terminal needs graphicx.sty or graphics.sty.}%
    \renewcommand\includegraphics[2][]{}%
  }%
  \providecommand\rotatebox[2]{#2}%
  \@ifundefined{ifGPcolor}{%
    \newif\ifGPcolor
    \GPcolortrue
  }{}%
  \@ifundefined{ifGPblacktext}{%
    \newif\ifGPblacktext
    \GPblacktexttrue
  }{}%
  % define a \g@addto@macro without @ in the name:
  \let\gplgaddtomacro\g@addto@macro
  % define empty templates for all commands taking text:
  \gdef\gplbacktext{}%
  \gdef\gplfronttext{}%
  \makeatother
  \ifGPblacktext
    % no textcolor at all
    \def\colorrgb#1{}%
    \def\colorgray#1{}%
  \else
    % gray or color?
    \ifGPcolor
      \def\colorrgb#1{\color[rgb]{#1}}%
      \def\colorgray#1{\color[gray]{#1}}%
      \expandafter\def\csname LTw\endcsname{\color{white}}%
      \expandafter\def\csname LTb\endcsname{\color{black}}%
      \expandafter\def\csname LTa\endcsname{\color{black}}%
      \expandafter\def\csname LT0\endcsname{\color[rgb]{1,0,0}}%
      \expandafter\def\csname LT1\endcsname{\color[rgb]{0,1,0}}%
      \expandafter\def\csname LT2\endcsname{\color[rgb]{0,0,1}}%
      \expandafter\def\csname LT3\endcsname{\color[rgb]{1,0,1}}%
      \expandafter\def\csname LT4\endcsname{\color[rgb]{0,1,1}}%
      \expandafter\def\csname LT5\endcsname{\color[rgb]{1,1,0}}%
      \expandafter\def\csname LT6\endcsname{\color[rgb]{0,0,0}}%
      \expandafter\def\csname LT7\endcsname{\color[rgb]{1,0.3,0}}%
      \expandafter\def\csname LT8\endcsname{\color[rgb]{0.5,0.5,0.5}}%
    \else
      % gray
      \def\colorrgb#1{\color{black}}%
      \def\colorgray#1{\color[gray]{#1}}%
      \expandafter\def\csname LTw\endcsname{\color{white}}%
      \expandafter\def\csname LTb\endcsname{\color{black}}%
      \expandafter\def\csname LTa\endcsname{\color{black}}%
      \expandafter\def\csname LT0\endcsname{\color{black}}%
      \expandafter\def\csname LT1\endcsname{\color{black}}%
      \expandafter\def\csname LT2\endcsname{\color{black}}%
      \expandafter\def\csname LT3\endcsname{\color{black}}%
      \expandafter\def\csname LT4\endcsname{\color{black}}%
      \expandafter\def\csname LT5\endcsname{\color{black}}%
      \expandafter\def\csname LT6\endcsname{\color{black}}%
      \expandafter\def\csname LT7\endcsname{\color{black}}%
      \expandafter\def\csname LT8\endcsname{\color{black}}%
    \fi
  \fi
    \setlength{\unitlength}{0.0500bp}%
    \ifx\gptboxheight\undefined%
      \newlength{\gptboxheight}%
      \newlength{\gptboxwidth}%
      \newsavebox{\gptboxtext}%
    \fi%
    \setlength{\fboxrule}{0.5pt}%
    \setlength{\fboxsep}{1pt}%
\begin{picture}(7200.00,5040.00)%
    \gplgaddtomacro\gplbacktext{%
      \csname LTb\endcsname%%
      \put(951,595){\makebox(0,0)[r]{\strut{}$-0.001$}}%
      \csname LTb\endcsname%%
      \put(951,1369){\makebox(0,0)[r]{\strut{}$0.001$}}%
      \csname LTb\endcsname%%
      \put(951,2143){\makebox(0,0)[r]{\strut{}$0.003$}}%
      \csname LTb\endcsname%%
      \put(951,2918){\makebox(0,0)[r]{\strut{}$0.005$}}%
      \csname LTb\endcsname%%
      \put(951,3692){\makebox(0,0)[r]{\strut{}$0.007$}}%
      \csname LTb\endcsname%%
      \put(951,4466){\makebox(0,0)[r]{\strut{}$0.009$}}%
      \csname LTb\endcsname%%
      \put(1053,409){\makebox(0,0){\strut{}$268$}}%
      \csname LTb\endcsname%%
      \put(1540,409){\makebox(0,0){\strut{}$269$}}%
      \csname LTb\endcsname%%
      \put(2026,409){\makebox(0,0){\strut{}$270$}}%
      \csname LTb\endcsname%%
      \put(2513,409){\makebox(0,0){\strut{}$271$}}%
      \csname LTb\endcsname%%
      \put(3000,409){\makebox(0,0){\strut{}$272$}}%
      \csname LTb\endcsname%%
      \put(3486,409){\makebox(0,0){\strut{}$273$}}%
      \csname LTb\endcsname%%
      \put(3973,409){\makebox(0,0){\strut{}$274$}}%
      \csname LTb\endcsname%%
      \put(4460,409){\makebox(0,0){\strut{}$275$}}%
      \csname LTb\endcsname%%
      \put(4946,409){\makebox(0,0){\strut{}$276$}}%
      \csname LTb\endcsname%%
      \put(5433,409){\makebox(0,0){\strut{}$277$}}%
      \csname LTb\endcsname%%
      \put(5920,409){\makebox(0,0){\strut{}$278$}}%
      \csname LTb\endcsname%%
      \put(6406,409){\makebox(0,0){\strut{}$279$}}%
      \csname LTb\endcsname%%
      \put(6893,409){\makebox(0,0){\strut{}$280$}}%
      \csname LTb\endcsname%%
      \put(3000,4079){\makebox(0,0)[r]{\strut{}peak \#1}}%
      \csname LTb\endcsname%%
      \put(4646,3418){\makebox(0,0)[l]{\strut{}peak \#2}}%
    }%
    \gplgaddtomacro\gplfronttext{%
      \csname LTb\endcsname%%
      \put(255,2724){\rotatebox{-270}{\makebox(0,0){\strut{}$\mathcal{A}$ / 1 cm}}}%
      \csname LTb\endcsname%%
      \put(3973,130){\makebox(0,0){\strut{}Wavelength (nm)}}%
      \csname LTb\endcsname%%
      \put(5975,4686){\makebox(0,0)[l]{\strut{}0.2032\%}}%
      \csname LTb\endcsname%%
      \put(5975,4500){\makebox(0,0)[l]{\strut{}0.1847\%}}%
      \csname LTb\endcsname%%
      \put(5975,4314){\makebox(0,0)[l]{\strut{}0.1693\%}}%
      \csname LTb\endcsname%%
      \put(5975,4128){\makebox(0,0)[l]{\strut{}0.1563\%}}%
      \csname LTb\endcsname%%
      \put(5975,3942){\makebox(0,0)[l]{\strut{}0.1451\%}}%
      \csname LTb\endcsname%%
      \put(5975,3756){\makebox(0,0)[l]{\strut{}0.1355\%}}%
      \csname LTb\endcsname%%
      \put(5975,3570){\makebox(0,0)[l]{\strut{}0.1270\%}}%
      \csname LTb\endcsname%%
      \put(5975,3384){\makebox(0,0)[l]{\strut{}0.1195\%}}%
      \csname LTb\endcsname%%
      \put(5975,3198){\makebox(0,0)[l]{\strut{}0.1129\%}}%
      \csname LTb\endcsname%%
      \put(5975,3012){\makebox(0,0)[l]{\strut{}0.1070\%}}%
      \csname LTb\endcsname%%
      \put(5975,2826){\makebox(0,0)[l]{\strut{}0.1016\%}}%
      \csname LTb\endcsname%%
      \put(5975,2640){\makebox(0,0)[l]{\strut{}0.0200\%}}%
    }%
    \gplbacktext
    \put(0,0){\includegraphics{pics/hdxabs}}%
    \gplfronttext
  \end{picture}%
\endgroup
}
	\caption[Gadolinium absorption spectra]%
	{Gadolinium absorption spectra between 270\,nm and 280\,nm.
	On the left, the spectrum is taken with a high-resolution integrated spectrophotometer %
	Shimadzu UV-2600, revealing numerous peaks in the region of interest for an aqueous solution %
	of 0.2\,\% gadolinium sulphate.
	On the right, the spectrum is recorded with a fast spectrometer Ocean HDX, varying the concentration 
	from 0.2\,\% down to 0.02\,\%: the height of the peaks scales linearly with the concentration.
	In both figures, the absorbance is normalised to the length of the sample.}
	\label{fig:gad_lines}
\end{figure}


\begin{figure}
	\centering
	\resizebox{0.6\linewidth}{!}{% GNUPLOT: LaTeX picture with Postscript
\begingroup
  \makeatletter
  \providecommand\color[2][]{%
    \GenericError{(gnuplot) \space\space\space\@spaces}{%
      Package color not loaded in conjunction with
      terminal option `colourtext'%
    }{See the gnuplot documentation for explanation.%
    }{Either use 'blacktext' in gnuplot or load the package
      color.sty in LaTeX.}%
    \renewcommand\color[2][]{}%
  }%
  \providecommand\includegraphics[2][]{%
    \GenericError{(gnuplot) \space\space\space\@spaces}{%
      Package graphicx or graphics not loaded%
    }{See the gnuplot documentation for explanation.%
    }{The gnuplot epslatex terminal needs graphicx.sty or graphics.sty.}%
    \renewcommand\includegraphics[2][]{}%
  }%
  \providecommand\rotatebox[2]{#2}%
  \@ifundefined{ifGPcolor}{%
    \newif\ifGPcolor
    \GPcolortrue
  }{}%
  \@ifundefined{ifGPblacktext}{%
    \newif\ifGPblacktext
    \GPblacktexttrue
  }{}%
  % define a \g@addto@macro without @ in the name:
  \let\gplgaddtomacro\g@addto@macro
  % define empty templates for all commands taking text:
  \gdef\gplbacktext{}%
  \gdef\gplfronttext{}%
  \makeatother
  \ifGPblacktext
    % no textcolor at all
    \def\colorrgb#1{}%
    \def\colorgray#1{}%
  \else
    % gray or color?
    \ifGPcolor
      \def\colorrgb#1{\color[rgb]{#1}}%
      \def\colorgray#1{\color[gray]{#1}}%
      \expandafter\def\csname LTw\endcsname{\color{white}}%
      \expandafter\def\csname LTb\endcsname{\color{black}}%
      \expandafter\def\csname LTa\endcsname{\color{black}}%
      \expandafter\def\csname LT0\endcsname{\color[rgb]{1,0,0}}%
      \expandafter\def\csname LT1\endcsname{\color[rgb]{0,1,0}}%
      \expandafter\def\csname LT2\endcsname{\color[rgb]{0,0,1}}%
      \expandafter\def\csname LT3\endcsname{\color[rgb]{1,0,1}}%
      \expandafter\def\csname LT4\endcsname{\color[rgb]{0,1,1}}%
      \expandafter\def\csname LT5\endcsname{\color[rgb]{1,1,0}}%
      \expandafter\def\csname LT6\endcsname{\color[rgb]{0,0,0}}%
      \expandafter\def\csname LT7\endcsname{\color[rgb]{1,0.3,0}}%
      \expandafter\def\csname LT8\endcsname{\color[rgb]{0.5,0.5,0.5}}%
    \else
      % gray
      \def\colorrgb#1{\color{black}}%
      \def\colorgray#1{\color[gray]{#1}}%
      \expandafter\def\csname LTw\endcsname{\color{white}}%
      \expandafter\def\csname LTb\endcsname{\color{black}}%
      \expandafter\def\csname LTa\endcsname{\color{black}}%
      \expandafter\def\csname LT0\endcsname{\color{black}}%
      \expandafter\def\csname LT1\endcsname{\color{black}}%
      \expandafter\def\csname LT2\endcsname{\color{black}}%
      \expandafter\def\csname LT3\endcsname{\color{black}}%
      \expandafter\def\csname LT4\endcsname{\color{black}}%
      \expandafter\def\csname LT5\endcsname{\color{black}}%
      \expandafter\def\csname LT6\endcsname{\color{black}}%
      \expandafter\def\csname LT7\endcsname{\color{black}}%
      \expandafter\def\csname LT8\endcsname{\color{black}}%
    \fi
  \fi
    \setlength{\unitlength}{0.0500bp}%
    \ifx\gptboxheight\undefined%
      \newlength{\gptboxheight}%
      \newlength{\gptboxwidth}%
      \newsavebox{\gptboxtext}%
    \fi%
    \setlength{\fboxrule}{0.5pt}%
    \setlength{\fboxsep}{1pt}%
\begin{picture}(7200.00,5040.00)%
    \gplgaddtomacro\gplbacktext{%
      \csname LTb\endcsname%%
      \put(849,595){\makebox(0,0)[r]{\strut{}$0$}}%
      \csname LTb\endcsname%%
      \put(849,1465){\makebox(0,0)[r]{\strut{}$5000$}}%
      \csname LTb\endcsname%%
      \put(849,2335){\makebox(0,0)[r]{\strut{}$10000$}}%
      \csname LTb\endcsname%%
      \put(849,3206){\makebox(0,0)[r]{\strut{}$15000$}}%
      \csname LTb\endcsname%%
      \put(849,4076){\makebox(0,0)[r]{\strut{}$20000$}}%
      \csname LTb\endcsname%%
      \put(849,4946){\makebox(0,0)[r]{\strut{}$25000$}}%
      \csname LTb\endcsname%%
      \put(951,409){\makebox(0,0){\strut{}$260$}}%
      \csname LTb\endcsname%%
      \put(1607,409){\makebox(0,0){\strut{}$265$}}%
      \csname LTb\endcsname%%
      \put(2263,409){\makebox(0,0){\strut{}$270$}}%
      \csname LTb\endcsname%%
      \put(2918,409){\makebox(0,0){\strut{}$275$}}%
      \csname LTb\endcsname%%
      \put(3574,409){\makebox(0,0){\strut{}$280$}}%
      \csname LTb\endcsname%%
      \put(4230,409){\makebox(0,0){\strut{}$285$}}%
      \csname LTb\endcsname%%
      \put(4886,409){\makebox(0,0){\strut{}$290$}}%
      \csname LTb\endcsname%%
      \put(5541,409){\makebox(0,0){\strut{}$295$}}%
      \csname LTb\endcsname%%
      \put(6197,409){\makebox(0,0){\strut{}$300$}}%
      \csname LTb\endcsname%%
      \put(6299,595){\makebox(0,0)[l]{\strut{}$0$}}%
      \csname LTb\endcsname%%
      \put(6299,1030){\makebox(0,0)[l]{\strut{}$0.1$}}%
      \csname LTb\endcsname%%
      \put(6299,1465){\makebox(0,0)[l]{\strut{}$0.2$}}%
      \csname LTb\endcsname%%
      \put(6299,1900){\makebox(0,0)[l]{\strut{}$0.3$}}%
      \csname LTb\endcsname%%
      \put(6299,2335){\makebox(0,0)[l]{\strut{}$0.4$}}%
      \csname LTb\endcsname%%
      \put(6299,2771){\makebox(0,0)[l]{\strut{}$0.5$}}%
      \csname LTb\endcsname%%
      \put(6299,3206){\makebox(0,0)[l]{\strut{}$0.6$}}%
      \csname LTb\endcsname%%
      \put(6299,3641){\makebox(0,0)[l]{\strut{}$0.7$}}%
      \csname LTb\endcsname%%
      \put(6299,4076){\makebox(0,0)[l]{\strut{}$0.8$}}%
      \csname LTb\endcsname%%
      \put(6299,4511){\makebox(0,0)[l]{\strut{}$0.9$}}%
      \csname LTb\endcsname%%
      \put(6299,4946){\makebox(0,0)[l]{\strut{}$1$}}%
    }%
    \gplgaddtomacro\gplfronttext{%
      \csname LTb\endcsname%%
      \put(153,2770){\rotatebox{-270}{\makebox(0,0){\strut{}Intensity}}}%
      \csname LTb\endcsname%%
      \put(6911,2770){\rotatebox{-270}{\makebox(0,0){\strut{}Absorbance}}}%
      \csname LTb\endcsname%%
      \put(3574,130){\makebox(0,0){\strut{}Wavelength (nm)}}%
      \csname LTb\endcsname%%
      \put(5613,4779){\makebox(0,0)[r]{\strut{}Pure water}}%
      \csname LTb\endcsname%%
      \put(5613,4593){\makebox(0,0)[r]{\strut{}Gd water}}%
      \csname LTb\endcsname%%
      \put(5613,4407){\makebox(0,0)[r]{\strut{}Absorbance}}%
    }%
    \gplbacktext
    \put(0,0){\includegraphics{pics/abs_example}}%
    \gplfronttext
  \end{picture}%
\endgroup
}
	\caption[Ocean HDX spectra with a 10\,cm water sample]%
		{Gadolinium absorption measurement taken with the Ocean HDX spectrometer and a 10\,cm water sample.
		The profile of the UV LED is shown when pure water (blue) and Gd-loaded water (black) are used.
		The absorbance (red) is then extracted from the two intensity measurement.}
	\label{fig:uv_led}
\end{figure}

The concept of tracking gadolinium concentration in water using the absorbance spectrum %
was proven using a 10\,cm cell, the measurements of which are reported in \reffig{fig:gad_lines}.
The spectra were taken with a high-resolution spectrophotometer Shimadzu UV2600 %
and a fast spectrometer Ocean HDX. % are shown at different concentration levels.
The Shimadzu spectrophotometer uses two grating systems to select very narrow windows of the electromagnetic spectrum %
both at the source and at the detection, which is performed by a photomultiplier.
A deuterium lamp, integrated in the spectrometer, generates light at UV wavelengths.
This machine can reach very high resolutions in wavelengths, however it scans over each wavelengths %
taking one measurement at the time and resulting in a slow process.
%This process can be quite slow and the gain from the resolution is not so relevant.
The Ocean HDX spectrometer does not have an integrated UV source, %
but the grating system refracts the light input on a linear CCD, recording the whole %
electromagnetic spectrum simultaneously.
The speed of the measurement with such a spectrometer is only limited by the integration time %
and reading rate of the CCD.
A set of consecutive traces is taken and the error on the measurement is estimated by taking the mean.
One of the advantage of the Ocean HDX spectrometer is that it can reach a very high light throughput %
thanks to a toroidal mirroring system which reduces stray light and maximises the dynamic range.
These are the deciding factors for monitoring gadolinium concentration with this technique, %
in addition to the speed of the measurement.
Wavelength resolution is not as important, because the relative position of the lines is sufficient %
to identify them and to compute the correct absorbance.
A UV-LED, with emission centred at 275\,nm, is used as an illuminating source.
The profiles of the LED in pure water and in Gd solution are reported in \reffig{fig:uv_led}, %
together with the resulting absorbance.

In an absorbance measurement, however, there could be other sources of absorption, %
since a light source and a photo sensor are needed, in addition to optical interfaces with the %
sample to measure; the solvent of the sample can also contribute to the overall absorbance.
The Beer-Lambert law can be generalised to describe a generic number of attenuating elements, %
as 
\begin{equation}
	\label{eq:linear_bl}
	\mathcal{A} = \sum_{i = 1}^N \varepsilon_i \int_0^\ell \dd{z} \rho_i (x) = %
	\ell\,\varepsilon_0\,\rho_0 +  \sum_{i = 2}^N \varepsilon_i \int_0^\ell \dd{z} \rho_i (x) = a + b \, \rho_0\ ,
\end{equation}
where the attenuating species of interest, labelled ``0'', has been isolated: %
if the other elements are constant, there is a linear law with measured absorbance and concentration.
The absorbance will also depend on the purity of the solvent---water in this case---and other factors, %
such as the optical interfaces.
It follows that absorption measurements taken with different setup cannot be compared unless the %
effect of other absorbing elements is well understood.
Namely this would require to know the factors $a$ and $b$ of \refeq{eq:linear_bl}, %
and so calibration of the measurement apparatus is required.
The absorption can also be biased by the presence of wavelength-independent factors, %
or at least independent in the region of the spectrum of interest.
Examples of these factors are microbubbles or other impurities in the samples which %
can block or scatter the light, thus contaminating the measurement. % ??
An unbiased estimation of the absorbance can therefore be achieved by using two absorption lines %
and taking the difference of the peaks: %
\begin{equation}
	\Delta \mathcal{A} = \mathcal{A}(x_1) - \mathcal{A}(x_2) = %
	\log_{10} \frac{I_0 (x_1)}{I(x_1)} - \log_{10} \frac{I_0 (x_2)}{I(x_2)}\ .
\end{equation}
Any wavelength-independent effect is removed doing so.
Supposing the true intensity of the reference sample is modified by a factor $\eta$ %
and the intensity of the study sample is varied by a factor $\zeta$, the absorbance difference is
\begin{equation}
	\Delta \mathcal{A} = %
	\log_{10} \frac{\eta I_0 (x_1)}{\zeta I(x_1)} - \log_{10} \frac{\eta I_0 (x_2)}{\zeta I(x_2)} = %
	\log_{10} \frac{I_0 (x_1)}{I(x_1)} - \log_{10} \frac{I_0 (x_2)}{I(x_2)}\ .
\end{equation}
The drawback of this method is that two measurements are needed at the two wavelengths $x_1$ and $x_2$: 
the error on the measurement will increase by a factor~$\sim\sqrt{2}$.



\begin{figure}
	\centering
	\resizebox{0.8\textwidth}{!}{% GNUPLOT: LaTeX picture with Postscript
\begingroup
  \makeatletter
  \providecommand\color[2][]{%
    \GenericError{(gnuplot) \space\space\space\@spaces}{%
      Package color not loaded in conjunction with
      terminal option `colourtext'%
    }{See the gnuplot documentation for explanation.%
    }{Either use 'blacktext' in gnuplot or load the package
      color.sty in LaTeX.}%
    \renewcommand\color[2][]{}%
  }%
  \providecommand\includegraphics[2][]{%
    \GenericError{(gnuplot) \space\space\space\@spaces}{%
      Package graphicx or graphics not loaded%
    }{See the gnuplot documentation for explanation.%
    }{The gnuplot epslatex terminal needs graphicx.sty or graphics.sty.}%
    \renewcommand\includegraphics[2][]{}%
  }%
  \providecommand\rotatebox[2]{#2}%
  \@ifundefined{ifGPcolor}{%
    \newif\ifGPcolor
    \GPcolortrue
  }{}%
  \@ifundefined{ifGPblacktext}{%
    \newif\ifGPblacktext
    \GPblacktexttrue
  }{}%
  % define a \g@addto@macro without @ in the name:
  \let\gplgaddtomacro\g@addto@macro
  % define empty templates for all commands taking text:
  \gdef\gplbacktext{}%
  \gdef\gplfronttext{}%
  \makeatother
  \ifGPblacktext
    % no textcolor at all
    \def\colorrgb#1{}%
    \def\colorgray#1{}%
  \else
    % gray or color?
    \ifGPcolor
      \def\colorrgb#1{\color[rgb]{#1}}%
      \def\colorgray#1{\color[gray]{#1}}%
      \expandafter\def\csname LTw\endcsname{\color{white}}%
      \expandafter\def\csname LTb\endcsname{\color{black}}%
      \expandafter\def\csname LTa\endcsname{\color{black}}%
      \expandafter\def\csname LT0\endcsname{\color[rgb]{1,0,0}}%
      \expandafter\def\csname LT1\endcsname{\color[rgb]{0,1,0}}%
      \expandafter\def\csname LT2\endcsname{\color[rgb]{0,0,1}}%
      \expandafter\def\csname LT3\endcsname{\color[rgb]{1,0,1}}%
      \expandafter\def\csname LT4\endcsname{\color[rgb]{0,1,1}}%
      \expandafter\def\csname LT5\endcsname{\color[rgb]{1,1,0}}%
      \expandafter\def\csname LT6\endcsname{\color[rgb]{0,0,0}}%
      \expandafter\def\csname LT7\endcsname{\color[rgb]{1,0.3,0}}%
      \expandafter\def\csname LT8\endcsname{\color[rgb]{0.5,0.5,0.5}}%
    \else
      % gray
      \def\colorrgb#1{\color{black}}%
      \def\colorgray#1{\color[gray]{#1}}%
      \expandafter\def\csname LTw\endcsname{\color{white}}%
      \expandafter\def\csname LTb\endcsname{\color{black}}%
      \expandafter\def\csname LTa\endcsname{\color{black}}%
      \expandafter\def\csname LT0\endcsname{\color{black}}%
      \expandafter\def\csname LT1\endcsname{\color{black}}%
      \expandafter\def\csname LT2\endcsname{\color{black}}%
      \expandafter\def\csname LT3\endcsname{\color{black}}%
      \expandafter\def\csname LT4\endcsname{\color{black}}%
      \expandafter\def\csname LT5\endcsname{\color{black}}%
      \expandafter\def\csname LT6\endcsname{\color{black}}%
      \expandafter\def\csname LT7\endcsname{\color{black}}%
      \expandafter\def\csname LT8\endcsname{\color{black}}%
    \fi
  \fi
    \setlength{\unitlength}{0.0500bp}%
    \ifx\gptboxheight\undefined%
      \newlength{\gptboxheight}%
      \newlength{\gptboxwidth}%
      \newsavebox{\gptboxtext}%
    \fi%
    \setlength{\fboxrule}{0.5pt}%
    \setlength{\fboxsep}{1pt}%
\begin{picture}(8640.00,5760.00)%
    \gplgaddtomacro\gplbacktext{%
      \csname LTb\endcsname%%
      \put(762,576){\makebox(0,0)[r]{\strut{}0.00}}%
      \csname LTb\endcsname%%
      \put(762,1056){\makebox(0,0)[r]{\strut{}3.00}}%
      \csname LTb\endcsname%%
      \put(762,1536){\makebox(0,0)[r]{\strut{}6.00}}%
      \csname LTb\endcsname%%
      \put(1508,390){\makebox(0,0){\strut{}0.002}}%
      \csname LTb\endcsname%%
      \put(3117,390){\makebox(0,0){\strut{}0.0025}}%
      \csname LTb\endcsname%%
      \put(4726,390){\makebox(0,0){\strut{}0.003}}%
      \csname LTb\endcsname%%
      \put(6335,390){\makebox(0,0){\strut{}0.0035}}%
      \csname LTb\endcsname%%
      \put(7944,390){\makebox(0,0){\strut{}0.004}}%
    }%
    \gplgaddtomacro\gplfronttext{%
      \csname LTb\endcsname%%
      \put(168,1296){\rotatebox{-270}{\makebox(0,0){\strut{}Error (\%)}}}%
      \csname LTb\endcsname%%
      \put(4726,111){\makebox(0,0){\strut{}Absorbance / 1 cm}}%
    }%
    \gplgaddtomacro\gplbacktext{%
      \csname LTb\endcsname%%
      \put(762,2016){\makebox(0,0)[r]{\strut{}0.05}}%
      \csname LTb\endcsname%%
      \put(762,2929){\makebox(0,0)[r]{\strut{}0.10}}%
      \csname LTb\endcsname%%
      \put(762,3841){\makebox(0,0)[r]{\strut{}0.15}}%
      \csname LTb\endcsname%%
      \put(762,4754){\makebox(0,0)[r]{\strut{}0.20}}%
      \csname LTb\endcsname%%
      \put(762,5666){\makebox(0,0)[r]{\strut{}0.25}}%
      \csname LTb\endcsname%%
      \put(1508,1830){\makebox(0,0){\strut{}}}%
      \csname LTb\endcsname%%
      \put(3117,1830){\makebox(0,0){\strut{}}}%
      \csname LTb\endcsname%%
      \put(4726,1830){\makebox(0,0){\strut{}}}%
      \csname LTb\endcsname%%
      \put(6335,1830){\makebox(0,0){\strut{}}}%
      \csname LTb\endcsname%%
      \put(7944,1830){\makebox(0,0){\strut{}}}%
      \csname LTb\endcsname%%
      \put(1250,4936){\makebox(0,0)[l]{\strut{}$a = 0.00013 \pm 0.00007$}}%
      \csname LTb\endcsname%%
      \put(1250,4571){\makebox(0,0)[l]{\strut{}$b = 0.0189 \pm 0.0005$}}%
      \csname LTb\endcsname%%
      \put(1250,4206){\makebox(0,0)[l]{\strut{}$\chi^2$ / DOF = 390.17288 / 9}}%
    }%
    \gplgaddtomacro\gplfronttext{%
      \csname LTb\endcsname%%
      \put(168,3841){\rotatebox{-270}{\makebox(0,0){\strut{}Concentration (\%)}}}%
      \csname LTb\endcsname%%
      \put(1658,5301){\makebox(0,0)[r]{\strut{}Data}}%
    }%
    \gplbacktext
    \put(0,0){\includegraphics{pics/gad_10cm}}%
    \gplfronttext
  \end{picture}%
\endgroup
}
	\caption[Linear fit between Gd concentration and absorption with a 10\,cm water sample]%
		{The inverse linear fit (red) between concentration and absorption for a 10\,cm water sample %
		is shown together with data (blue) on the top panel.
		The absolute values of the errors are roughly constant with the concentration, %
		as it is mostly dominated by the uncertainties on the linear fit.
		The relative error (bottom panel) therefore increases with lower concentrations.
		The rightmost point corresponds to a 0.2\,\% concentration, the relative error %
		of which is around~3\,\%.}
	\label{fig:gad_10cm}
\end{figure}


In order to estimate the accuracy of the gadolinium concentration measurement, %
a prototype based on a 10\,cm cell was tested, starting with a 0.2\,\% gadolinium sulphate concentration %
which is diluted down to achieve lower concentrations.
At each known value of the concentration, the absorption is calculated by averaging one thousands spectra %
taken with the Gd-loaded water and the pure water samples.
In this way, the error on the absorbance is defined as the standard error of the mean.
Following \refeq{eq:linear_bl}, a linear interpolation is performed between the difference of the absorption peaks %
$\Delta \mathcal{A}$ and the concentration~$\rho_\text{Gb}$: 
\begin{equation}
	\Delta \mathcal{A} = a + b\,\rho_\text{Gd}\ .
\end{equation}
The relation above is then inverted, such that uncertainties from the absorbance measurement %
can be propagated to the concentration value, as
\begin{equation}
	\rho_\text{Gd} = \frac{\Delta \mathcal{A} - a}{b}\ .
\end{equation}
The result of the fit is shown in \reffig{fig:gad_10cm}, from which a relative error of $\sim3$\,\% %
is achieved on the concentration measurement of 0.2\,\%.
According to \reffig{fig:gd_conc}, this uncertainty translates to a $\sim1$\,\% error on the neutron capture efficiency.
The method employing the 10\,cm water sample performs similarly to the AAS spectrometry employed in EGADS.


\begin{figure}
	\centering
	\includegraphics[width=0.7\linewidth]{pics/Device.pdf}
	\caption[Schematic of the device used for testing the 100\,cm water sample]%
		{Schematic of the device used for testing the 100\,cm water sample in laboratory.
		The diaphragm pump circulates the water in the cell and interfaces the device with the main water system.
		During the test, the pump was leaving a considerable amount of microbubbles in the sample.
		Waiting 15 minutes between measurement was found to be sufficient to restore the intensity of the light at %
		the spectrometer.
		The pump, however, might not be needed when the device will be installed in EGADS or SK as the %
		pressure of the water system will be enough to circulate water and fill the 100\,cm cell.
		This scenario is favourable since the amount of undesired microbubbles is minimised and the %
		measurement can be taken more frequently.}
	\label{fig:gad_device}
\end{figure}

In the first stage of SK-Gd, a concentration of 0.02\,\% gadolinium sulphate %
will be loaded, and therefore a good sensitivity on this low value is also required.
To achieve better sensitivities, a longer water sample can be used to increase absorbance by gadolinium.
The resolution should improve with the absolute quantity of Gd, or the length in this case, %
since more light is absorbed and the peaks are more distinct.
Thanks to the high light throughput of the Ocean HDX spectrometer, %
the absorbance of a 100\,cm water sample was successfully studied.
In order to deal with a larger water volume, an automated device was developed to fill the water sample, %
activate the LED and operate the spectrometer.
The diaphragm pump used to circulate the water produced a considerable amount of microbubbles during measurements.
It was found that letting the water settle for 15 minutes between measurement was sufficient %
to remove the majority of microbubble and so clear the light path.
A schematic of the prototype used in laboratory is shown in \reffig{fig:gad_device}.

\begin{figure}
	\centering
	\resizebox{\linewidth}{!}{% GNUPLOT: LaTeX picture with Postscript
\begingroup
  \makeatletter
  \providecommand\color[2][]{%
    \GenericError{(gnuplot) \space\space\space\@spaces}{%
      Package color not loaded in conjunction with
      terminal option `colourtext'%
    }{See the gnuplot documentation for explanation.%
    }{Either use 'blacktext' in gnuplot or load the package
      color.sty in LaTeX.}%
    \renewcommand\color[2][]{}%
  }%
  \providecommand\includegraphics[2][]{%
    \GenericError{(gnuplot) \space\space\space\@spaces}{%
      Package graphicx or graphics not loaded%
    }{See the gnuplot documentation for explanation.%
    }{The gnuplot epslatex terminal needs graphicx.sty or graphics.sty.}%
    \renewcommand\includegraphics[2][]{}%
  }%
  \providecommand\rotatebox[2]{#2}%
  \@ifundefined{ifGPcolor}{%
    \newif\ifGPcolor
    \GPcolortrue
  }{}%
  \@ifundefined{ifGPblacktext}{%
    \newif\ifGPblacktext
    \GPblacktexttrue
  }{}%
  % define a \g@addto@macro without @ in the name:
  \let\gplgaddtomacro\g@addto@macro
  % define empty templates for all commands taking text:
  \gdef\gplbacktext{}%
  \gdef\gplfronttext{}%
  \makeatother
  \ifGPblacktext
    % no textcolor at all
    \def\colorrgb#1{}%
    \def\colorgray#1{}%
  \else
    % gray or color?
    \ifGPcolor
      \def\colorrgb#1{\color[rgb]{#1}}%
      \def\colorgray#1{\color[gray]{#1}}%
      \expandafter\def\csname LTw\endcsname{\color{white}}%
      \expandafter\def\csname LTb\endcsname{\color{black}}%
      \expandafter\def\csname LTa\endcsname{\color{black}}%
      \expandafter\def\csname LT0\endcsname{\color[rgb]{1,0,0}}%
      \expandafter\def\csname LT1\endcsname{\color[rgb]{0,1,0}}%
      \expandafter\def\csname LT2\endcsname{\color[rgb]{0,0,1}}%
      \expandafter\def\csname LT3\endcsname{\color[rgb]{1,0,1}}%
      \expandafter\def\csname LT4\endcsname{\color[rgb]{0,1,1}}%
      \expandafter\def\csname LT5\endcsname{\color[rgb]{1,1,0}}%
      \expandafter\def\csname LT6\endcsname{\color[rgb]{0,0,0}}%
      \expandafter\def\csname LT7\endcsname{\color[rgb]{1,0.3,0}}%
      \expandafter\def\csname LT8\endcsname{\color[rgb]{0.5,0.5,0.5}}%
    \else
      % gray
      \def\colorrgb#1{\color{black}}%
      \def\colorgray#1{\color[gray]{#1}}%
      \expandafter\def\csname LTw\endcsname{\color{white}}%
      \expandafter\def\csname LTb\endcsname{\color{black}}%
      \expandafter\def\csname LTa\endcsname{\color{black}}%
      \expandafter\def\csname LT0\endcsname{\color{black}}%
      \expandafter\def\csname LT1\endcsname{\color{black}}%
      \expandafter\def\csname LT2\endcsname{\color{black}}%
      \expandafter\def\csname LT3\endcsname{\color{black}}%
      \expandafter\def\csname LT4\endcsname{\color{black}}%
      \expandafter\def\csname LT5\endcsname{\color{black}}%
      \expandafter\def\csname LT6\endcsname{\color{black}}%
      \expandafter\def\csname LT7\endcsname{\color{black}}%
      \expandafter\def\csname LT8\endcsname{\color{black}}%
    \fi
  \fi
    \setlength{\unitlength}{0.0500bp}%
    \ifx\gptboxheight\undefined%
      \newlength{\gptboxheight}%
      \newlength{\gptboxwidth}%
      \newsavebox{\gptboxtext}%
    \fi%
    \setlength{\fboxrule}{0.5pt}%
    \setlength{\fboxsep}{1pt}%
\begin{picture}(14400.00,5760.00)%
    \gplgaddtomacro\gplbacktext{%
      \csname LTb\endcsname%%
      \put(618,624){\makebox(0,0)[r]{\strut{}0}}%
      \csname LTb\endcsname%%
      \put(618,1296){\makebox(0,0)[r]{\strut{}0.2}}%
      \csname LTb\endcsname%%
      \put(618,1969){\makebox(0,0)[r]{\strut{}0.4}}%
      \csname LTb\endcsname%%
      \put(618,2641){\makebox(0,0)[r]{\strut{}0.6}}%
      \csname LTb\endcsname%%
      \put(618,3313){\makebox(0,0)[r]{\strut{}0.8}}%
      \csname LTb\endcsname%%
      \put(618,3985){\makebox(0,0)[r]{\strut{}1}}%
      \csname LTb\endcsname%%
      \put(618,4658){\makebox(0,0)[r]{\strut{}1.2}}%
      \csname LTb\endcsname%%
      \put(618,5330){\makebox(0,0)[r]{\strut{}1.4}}%
      \csname LTb\endcsname%%
      \put(720,102){\makebox(0,0){\strut{}$240$}}%
      \csname LTb\endcsname%%
      \put(1697,102){\makebox(0,0){\strut{}$250$}}%
      \csname LTb\endcsname%%
      \put(2674,102){\makebox(0,0){\strut{}$260$}}%
      \csname LTb\endcsname%%
      \put(3651,102){\makebox(0,0){\strut{}$270$}}%
      \csname LTb\endcsname%%
      \put(4628,102){\makebox(0,0){\strut{}$280$}}%
      \csname LTb\endcsname%%
      \put(5605,102){\makebox(0,0){\strut{}$290$}}%
      \csname LTb\endcsname%%
      \put(6582,102){\makebox(0,0){\strut{}$300$}}%
    }%
    \gplgaddtomacro\gplfronttext{%
      \csname LTb\endcsname%%
      \put(126,2977){\rotatebox{-270}{\makebox(0,0){\strut{}Absorbance}}}%
      \csname LTb\endcsname%%
      \put(6975,3453){\makebox(0,0)[r]{\strut{}0.016\%}}%
      \csname LTb\endcsname%%
      \put(6975,3639){\makebox(0,0)[r]{\strut{}0.018\%}}%
      \csname LTb\endcsname%%
      \put(6975,3825){\makebox(0,0)[r]{\strut{}0.020\%}}%
      \csname LTb\endcsname%%
      \put(6975,4011){\makebox(0,0)[r]{\strut{}0.022\%}}%
      \csname LTb\endcsname%%
      \put(6975,4197){\makebox(0,0)[r]{\strut{}0.024\%}}%
      \csname LTb\endcsname%%
      \put(6975,4383){\makebox(0,0)[r]{\strut{}0.030\%}}%
      \csname LTb\endcsname%%
      \put(6975,4569){\makebox(0,0)[r]{\strut{}0.050\%}}%
      \csname LTb\endcsname%%
      \put(6975,4755){\makebox(0,0)[r]{\strut{}0.100\%}}%
      \csname LTb\endcsname%%
      \put(6975,4941){\makebox(0,0)[r]{\strut{}0.160\%}}%
      \csname LTb\endcsname%%
      \put(6975,5127){\makebox(0,0)[r]{\strut{}0.180\%}}%
      \csname LTb\endcsname%%
      \put(6975,5313){\makebox(0,0)[r]{\strut{}0.200\%}}%
      \csname LTb\endcsname%%
      \put(6975,5499){\makebox(0,0)[r]{\strut{}0.220\%}}%
    }%
    \gplgaddtomacro\gplbacktext{%
      \csname LTb\endcsname%%
      \put(7457,624){\makebox(0,0)[r]{\strut{}}}%
      \csname LTb\endcsname%%
      \put(7457,1296){\makebox(0,0)[r]{\strut{}}}%
      \csname LTb\endcsname%%
      \put(7457,1969){\makebox(0,0)[r]{\strut{}}}%
      \csname LTb\endcsname%%
      \put(7457,2641){\makebox(0,0)[r]{\strut{}}}%
      \csname LTb\endcsname%%
      \put(7457,3313){\makebox(0,0)[r]{\strut{}}}%
      \csname LTb\endcsname%%
      \put(7457,3985){\makebox(0,0)[r]{\strut{}}}%
      \csname LTb\endcsname%%
      \put(7457,4658){\makebox(0,0)[r]{\strut{}}}%
      \csname LTb\endcsname%%
      \put(7457,5330){\makebox(0,0)[r]{\strut{}}}%
      \csname LTb\endcsname%%
      \put(7559,102){\makebox(0,0){\strut{}$240$}}%
      \csname LTb\endcsname%%
      \put(8529,102){\makebox(0,0){\strut{}$250$}}%
      \csname LTb\endcsname%%
      \put(9499,102){\makebox(0,0){\strut{}$260$}}%
      \csname LTb\endcsname%%
      \put(10469,102){\makebox(0,0){\strut{}$270$}}%
      \csname LTb\endcsname%%
      \put(11438,102){\makebox(0,0){\strut{}$280$}}%
      \csname LTb\endcsname%%
      \put(12408,102){\makebox(0,0){\strut{}$290$}}%
      \csname LTb\endcsname%%
      \put(13378,102){\makebox(0,0){\strut{}$300$}}%
      \csname LTb\endcsname%%
      \put(14348,102){\makebox(0,0){\strut{}$310$}}%
      \csname LTb\endcsname%%
      \put(10954,4658){\makebox(0,0){\strut{}\shortstack{background \\ subtraction}}}%
    }%
    \gplgaddtomacro\gplfronttext{%
      \csname LTb\endcsname%%
      \put(7457,2977){\rotatebox{-270}{\makebox(0,0){\strut{}}}}%
    }%
    \gplbacktext
    \put(0,0){\includegraphics{pics/polyfit}}%
    \gplfronttext
  \end{picture}%
\endgroup
}
	\caption[Gadolinium absorption spectrum of the 100\,cm water sample]%
		{Gadolinium absorption lines (right) taken with the automated device on the 100\,cm water sample.
		The baseline of the absorption spectrum appears to be changing with the concentration.
		This could be due to some contaminant in the Gd-enriched water, which is diluted with pure water %
		at each step.
		The measurements are also taken at different times and drifts in the electronics or optics might be a %
		concurrent cause.
		Removing the baseline with a polynomial fit (right) makes the absorption lines comparable.}
	\label{fig:gad_fit}
\end{figure}

At this long scale, effects from optics, LED and spectrometer alignment, and water purity become important.
As it can be seen from \reffig{fig:gad_fit}, measurements of the absorption at different concentration levels %
present a nonflat baseline that can bias the concentration estimation.
A five-degree polynomial is employed to fit the baseline, which is then removed in order to keep the measurement consistent over time.
The same analysis of the 10\,cm water sample is performed on the 100\,cm measurement %
and the dependency of concentration on absorbance is reported in \reffig{fig:gad_1m}.
The relative error on the measurement is improved tenfold compared to the 10\,cm test.
For a concentration of 0.2\,\% the error is estimated to be roughly 0.3\,\%, %
whereas for a 0.02\,\% concentration the error is at worst around 1.5\,\%. 	% ??
The calibration procedure, however, needs further investigation since %
the concentration seems to follow a quadratic law with the absorbance.
Nonlinear effects might be originated from the optics, the alignment, or the LED temperature drift.
It is expected that this behaviour should be lessened in the final prototype.
This device will be eventually installed in EGADS before and SK-Gd later to monitor the gadolinium concentration.
Once connected to the water system of these experiments, the pressure of the water flow will %
be sufficient to fill the 100\,cm water cell.
This will vastly reduce the amount of microbubbles and allow for an almost continuous measurement.
The water will be also more pure than the one used in laboratory testing, being continuously filtered.
Overall, the 100\,cm demonstrator proved that the absorption UV--spectroscopy technique successfully works %
with a better accuracy than the current technique AAS technique.



\begin{figure}
	\centering
	\resizebox{0.9\textwidth}{!}{% GNUPLOT: LaTeX picture with Postscript
\begingroup
  \makeatletter
  \providecommand\color[2][]{%
    \GenericError{(gnuplot) \space\space\space\@spaces}{%
      Package color not loaded in conjunction with
      terminal option `colourtext'%
    }{See the gnuplot documentation for explanation.%
    }{Either use 'blacktext' in gnuplot or load the package
      color.sty in LaTeX.}%
    \renewcommand\color[2][]{}%
  }%
  \providecommand\includegraphics[2][]{%
    \GenericError{(gnuplot) \space\space\space\@spaces}{%
      Package graphicx or graphics not loaded%
    }{See the gnuplot documentation for explanation.%
    }{The gnuplot epslatex terminal needs graphicx.sty or graphics.sty.}%
    \renewcommand\includegraphics[2][]{}%
  }%
  \providecommand\rotatebox[2]{#2}%
  \@ifundefined{ifGPcolor}{%
    \newif\ifGPcolor
    \GPcolortrue
  }{}%
  \@ifundefined{ifGPblacktext}{%
    \newif\ifGPblacktext
    \GPblacktexttrue
  }{}%
  % define a \g@addto@macro without @ in the name:
  \let\gplgaddtomacro\g@addto@macro
  % define empty templates for all commands taking text:
  \gdef\gplbacktext{}%
  \gdef\gplfronttext{}%
  \makeatother
  \ifGPblacktext
    % no textcolor at all
    \def\colorrgb#1{}%
    \def\colorgray#1{}%
  \else
    % gray or color?
    \ifGPcolor
      \def\colorrgb#1{\color[rgb]{#1}}%
      \def\colorgray#1{\color[gray]{#1}}%
      \expandafter\def\csname LTw\endcsname{\color{white}}%
      \expandafter\def\csname LTb\endcsname{\color{black}}%
      \expandafter\def\csname LTa\endcsname{\color{black}}%
      \expandafter\def\csname LT0\endcsname{\color[rgb]{1,0,0}}%
      \expandafter\def\csname LT1\endcsname{\color[rgb]{0,1,0}}%
      \expandafter\def\csname LT2\endcsname{\color[rgb]{0,0,1}}%
      \expandafter\def\csname LT3\endcsname{\color[rgb]{1,0,1}}%
      \expandafter\def\csname LT4\endcsname{\color[rgb]{0,1,1}}%
      \expandafter\def\csname LT5\endcsname{\color[rgb]{1,1,0}}%
      \expandafter\def\csname LT6\endcsname{\color[rgb]{0,0,0}}%
      \expandafter\def\csname LT7\endcsname{\color[rgb]{1,0.3,0}}%
      \expandafter\def\csname LT8\endcsname{\color[rgb]{0.5,0.5,0.5}}%
    \else
      % gray
      \def\colorrgb#1{\color{black}}%
      \def\colorgray#1{\color[gray]{#1}}%
      \expandafter\def\csname LTw\endcsname{\color{white}}%
      \expandafter\def\csname LTb\endcsname{\color{black}}%
      \expandafter\def\csname LTa\endcsname{\color{black}}%
      \expandafter\def\csname LT0\endcsname{\color{black}}%
      \expandafter\def\csname LT1\endcsname{\color{black}}%
      \expandafter\def\csname LT2\endcsname{\color{black}}%
      \expandafter\def\csname LT3\endcsname{\color{black}}%
      \expandafter\def\csname LT4\endcsname{\color{black}}%
      \expandafter\def\csname LT5\endcsname{\color{black}}%
      \expandafter\def\csname LT6\endcsname{\color{black}}%
      \expandafter\def\csname LT7\endcsname{\color{black}}%
      \expandafter\def\csname LT8\endcsname{\color{black}}%
    \fi
  \fi
    \setlength{\unitlength}{0.0500bp}%
    \ifx\gptboxheight\undefined%
      \newlength{\gptboxheight}%
      \newlength{\gptboxwidth}%
      \newsavebox{\gptboxtext}%
    \fi%
    \setlength{\fboxrule}{0.5pt}%
    \setlength{\fboxsep}{1pt}%
\begin{picture}(11520.00,5760.00)%
    \gplgaddtomacro\gplbacktext{%
      \csname LTb\endcsname%%
      \put(1050,576){\makebox(0,0)[r]{\strut{}0.00}}%
      \csname LTb\endcsname%%
      \put(1050,936){\makebox(0,0)[r]{\strut{}0.50}}%
      \csname LTb\endcsname%%
      \put(1050,1296){\makebox(0,0)[r]{\strut{}1.00}}%
      \csname LTb\endcsname%%
      \put(1050,1656){\makebox(0,0)[r]{\strut{}1.50}}%
      \csname LTb\endcsname%%
      \put(2257,390){\makebox(0,0){\strut{}0.0005}}%
      \csname LTb\endcsname%%
      \put(4099,390){\makebox(0,0){\strut{}0.001}}%
      \csname LTb\endcsname%%
      \put(5942,390){\makebox(0,0){\strut{}0.0015}}%
      \csname LTb\endcsname%%
      \put(7784,390){\makebox(0,0){\strut{}0.002}}%
      \csname LTb\endcsname%%
      \put(9626,390){\makebox(0,0){\strut{}0.0025}}%
      \csname LTb\endcsname%%
      \put(11468,390){\makebox(0,0){\strut{}0.003}}%
    }%
    \gplgaddtomacro\gplfronttext{%
      \csname LTb\endcsname%%
      \put(456,1296){\rotatebox{-270}{\makebox(0,0){\strut{}Error (\%)}}}%
      \csname LTb\endcsname%%
      \put(6310,111){\makebox(0,0){\strut{}Absorbance / 1 cm}}%
    }%
    \gplgaddtomacro\gplbacktext{%
      \csname LTb\endcsname%%
      \put(1050,2016){\makebox(0,0)[r]{\strut{}0.00}}%
      \csname LTb\endcsname%%
      \put(1050,2746){\makebox(0,0)[r]{\strut{}0.05}}%
      \csname LTb\endcsname%%
      \put(1050,3476){\makebox(0,0)[r]{\strut{}0.10}}%
      \csname LTb\endcsname%%
      \put(1050,4206){\makebox(0,0)[r]{\strut{}0.15}}%
      \csname LTb\endcsname%%
      \put(1050,4936){\makebox(0,0)[r]{\strut{}0.20}}%
      \csname LTb\endcsname%%
      \put(1050,5666){\makebox(0,0)[r]{\strut{}0.25}}%
      \csname LTb\endcsname%%
      \put(2257,1830){\makebox(0,0){\strut{}}}%
      \csname LTb\endcsname%%
      \put(4099,1830){\makebox(0,0){\strut{}}}%
      \csname LTb\endcsname%%
      \put(5942,1830){\makebox(0,0){\strut{}}}%
      \csname LTb\endcsname%%
      \put(7784,1830){\makebox(0,0){\strut{}}}%
      \csname LTb\endcsname%%
      \put(9626,1830){\makebox(0,0){\strut{}}}%
      \csname LTb\endcsname%%
      \put(11468,1830){\makebox(0,0){\strut{}}}%
      \csname LTb\endcsname%%
      \put(1668,4936){\makebox(0,0)[l]{\strut{}a = 0.001816 +/- 0.000264}}%
      \csname LTb\endcsname%%
      \put(1668,4571){\makebox(0,0)[l]{\strut{}b = 1.771576 +/- 0.011811}}%
      \csname LTb\endcsname%%
      \put(1668,4206){\makebox(0,0)[l]{\strut{}$\chi^2$ / DOF = 0.58796 / 10}}%
    }%
    \gplgaddtomacro\gplfronttext{%
      \csname LTb\endcsname%%
      \put(456,3841){\rotatebox{-270}{\makebox(0,0){\strut{}Concentration (\%)}}}%
      \csname LTb\endcsname%%
      \put(2076,5301){\makebox(0,0)[r]{\strut{}Data}}%
    }%
    \gplbacktext
    \put(0,0){\includegraphics{pics/gad_1m}}%
    \gplfronttext
  \end{picture}%
\endgroup
}
	\caption[Linear fit between Gd concentration and absorption with a 100\,cm water sample]%
		{The inverse linear fit (red) between concentration and absorption %
		is shown together with data (blue) on the top panel for the 100\,cm water sample.
		The error bars on the top panel are enlarged by a factor of 20.
		The linear fit is done only with the six leftmost points, %
		since the behaviour appears to be more quadratic (green) than linear.
		The relative error (bottom panel) is well below the 3\,\% limit, %
		even for a concentration of 0.02\,\%.}
	\label{fig:gad_1m}
\end{figure}
